\eject
\iffulloutput
\vbox{\rightline{\pic[100mm]{agrippa.jpg}}
\kern-20mm
\eightit\baselineskip10pt\rightskip105mm\leftskip0mm\parindent1.05cm
Only for you, children of doctrine and learning, have we written this work.
Examine this book, ponder the meaning we have dispersed in various places
and gathered again;
what we have concealed in one place we have disclosed in another,
that it may be understood by your wisdom.\par
\kern-10pt
\rightline{\hfil\eightrm ---Henricus C. Agrippa ab Nettesheym, \eightmit 1533}
}
\fi

\chapter Creating data files.

\subchapter Basics.

The input files for Therion are in text format. There are a few rules about how
such a file should look:

\list
* There are two kinds of commands. One-line commands and
  multi-line commands.

* A one-line command is terminated by an end of
  line character. The syntax of these is

  |command arg1 ... argN [-option1 value1 -option2 value2 ...]|

  where {\it arg1 ... argN} are obligatory arguments, and pairs
  {\it -option value} are options, which you may freely omit.
  Which arguments and options are available depends on the particular command.
  An example may be

  |point 643.5 505.0 gradient -orientation 144.7|

  with three obligatory arguments and one optional option/value pair. Sometimes options
  have no or multiple values.

* Multi-line commands begin similarly to one line commands, but continue on
  subsequent lines until explicit command termination. These lines may contain
  either data or options, which are applied to subsequent data. If a data line
  starts with a word reserved for an option, you have to insert `|!|' in front
  of it. The syntax is

  |command arg1 ... argN [-option1 value1 -option2 value2 ...]
  ...
  optionX valueX
  data
  ...
endcommand|

Again, for better illustration, a real example follows:

|line wall -id walltobereferenced
  1174.0 744.5
  1194.0 756.5 1192.5 757.5 1176.0 791.0
  smooth off
  1205.5 788.0 1195.5 832.5 1173.5 879.0
endline|

This command |line| has one obligatory argument, a line type (passage wall in
this case), followed by one option. The next two lines contain data (coordinates of
B\'ezier curves to be drawn). The next line (``|smooth off|'') specifies an option which
applies to subsequent data (i.e. not for the whole line, unlike the option |-id|
in the first line) and the last line contains some more data.

* if the value of an option or argument contains spaces, you should enclose this
  value in \hbox{|" "|} or \hbox{|[ ]|}. If you want to put a double-quote |"| into
  text in \hbox{|" "|} you need to insert it twice. Quotes are used for strings;
  brackets for numerical values and keywords.

* each line ending with a backslash (|\|) is considered to continue on
  the next line, as if there was neither line-break nor backlash.

* everything that follows |#|, until the end of line---even inside a command---is
  considered to be a comment, and is ignored.

* \NEW{5.4}multiline comments inside |comment| ... |endcomment| block are allowed in
  data and configuration files

\endlist


\subchapter Data types.

Therion uses following data types:

\list
* {\it keyword} = a sequence of |A-Z|, |a-z|, |0-9| and |_-/| characters
             (not starting with `|-|').
* {\it ext\_keyword} = keyword that can also contain |+*.,'| characters,
            but not on the first position.
* {\it restr\_keyword} = keyword limited to |A-Z|, |a-z| and |_| characters because
            of \MP\ limitations
* {\it date} = a date (or a time interval) specification in the format\hfil\break
          |YYYY[.MM[.DD[@HH[:MM[:SS[.SS]]]]]] [- YYYY[.MM[.DD[@HH[:MM[:SS[.SS]]]]]]]| or `|-|'
          to leave a date unspecified.
* {\it person} = a person's first name and surname separated by whitespace characters.
            Use `|/|' to separate first name and surname if there are
            more names.
* {\it string} = a sequence of any characters.
    \NEW{5.3}Strings may contain special tag |<lang:XX>| to separate
    translations.
    In multilingual strings only the text between |<lang:XX>|
    (where |XX| is the language selected in initialization
    or configuration file) and the next |<lang:YY>|
    tag is displayed on the output.
    If no match is found, everything before any occurrence of
    |<lang:ZZ>| tag is displayed.
* {\it units} = length units supported:
           meter[s], centimeter[s], inch[es], feet[s], yard[s]
           (also m, cm, in, ft, yd).
           Angle units supported: degree[s], minute[s] (also deg, min),
	   grad[s], mil[s], percent[age] (clino only).
           A degree value may be entered in decimal notation
	   ($x.y$) or in a special notation for degrees, minutes and seconds
	   ($deg[{:}min[{:}sec]]$).
\endlist



\subchapter Coordinate systems.

Therion supports coordinate transformations in geodetic coordinate systems.
You can specify |cs| option in |centreline|, |surface|, |import| and
|layout| objects and then enter XY data in given system.
You can also specify output |cs| in configuration file.

If you do not specify any |cs| in your dataset,
it is assumed you are working in |local| coordinate system, and no
conversions are done. If you specify |cs| anywhere in the data, you have to
specify it for all location data (|fix|, |origin| in |layout| etc.).

|cs| applies to all subsequent location data until other |cs| is specified or
until the end of the current object, whichever comes first.

Following coordinate systems are supported:

\list
* |UTM1| -- |UTM60| = Universal Transverse Mercator in northern hemisphere
  and given zone, WGS84 datum. Equivalent to EPSG:32601--EPSG:32660.
* |UTM1N| -- |UTM60N| = same as |UTM1| -- |UTM60|
* |UTM1S| -- |UTM60S| = UTM in southern hemisphere, WGS84 datum.\hfil\break
  Equivalent to EPSG:32701--EPSG:32760.
* |lat-long|, |long-lat| = latitude (N positive, S negative) and
  longitude (E positive, W negative) in given order in degrees
  ($deg[{:}min[{:}sec]]$ allowed), WGS84 datum. Not
  supported on output. Equivalent to EPSG:4326.
* |EPSG:<number>| = Most of EPSG coordinate systems. Almost every
  coordinate system used worldwide has its own EPSG number. To find
  the number of your system, see \www{https://epsg.org}.
* |ESRI:<number>| = Similar to EPSG, but ESRI standard.
* |ETRS| = European Terrestrial Reference System 1989 (ETRS89); long-lat order,
  not supported on output. Equivalent to EPSG:4258.
* |ETRS28| -- |ETRS37| = ETRS89 zones in UTM projection; east-north order.
  Equivalent to EPSG:25828--EPSG:25837.
%* |EUR79Z30| = UTM zone 30, EUR79 datum.  % ad-hoc addition not worth mentioning??
* |JTSK|, |iJTSK| = Czechoslovak S-JTSK system used since 1920s with south and west
  axis (JTSK) and its modified version with axis
  pointing east and north and negative numbers (iJTSK). JTSK is
  not supported on output (iJTSK is).
* |JTSK03|, |iJTSK03| = new S-JTSK realisation introduced in Slovakia in 2011.
* |OSGB:<H, N, O, S or T><A-Z except I>| = British Ordnance Survey National
  Grid.\NEW{5.4}
* |S-MERC| = the spherical Mercator projection, as used by various online mapping sites.
  Equivalent to EPSG:3857.
\endlist

\subchapter Magnetic declination.

Therion contains built-in IGRF\[See \www{https://www.ngdc.noaa.gov/IAGA/vmod/}]
Earth geomagnetic field model valid for period
1900--2025\NEW{5.4.5}. It is automatically used if the cave is located in space with a
|fix| station using any of the supported geodetic coordinate systems and in time with the
centerlines |date| command. The computed declination is listed in the LOG file for
information.

If the user specified a |declination| in the centerline, that value takes
precedence over the automatic calculation.

\subchapter Data format.

The syntax of input files is explained in the description of
individual commands. Studying the example files distributed with
Therion will help you understand. See also an example in the {\it Appendix}.

Each of the following sections describes
one Therion command using the following structure:

{\it Description:} notes concerning this command.

{\it Syntax:} schematic syntax description.

{\it Context:} specifies the context in which is this command allowed.
The {\it survey} context means that the command must be enclosed by
|survey ... endsurvey| pair. The {\it scrap} context means that the command must be
enclosed within |scrap ... endscrap| pair. Context {\it all} means that
the command may be used anywhere.

{\it Arguments:} a list of the obligatory arguments with explanations.

{\it Options:} a list of the available options.

{\it Command-like options:} options for multi-line commands, which can be specified
   among the data lines.

%%%%%%%%%%%%%%%%%%%%%%%%%%%%%%%%%%%%%%%%


\subsubchapter `encoding'.

\description
  sets the encoding of input file. This allows the use of non-ASCII characters
  in input files.
\enddescription

\syntax
  |encoding <encoding-name>|
\endsyntax

\context
  It should be the very first command in the file.
\endcontext

\arguments
* |<encoding-name>| = to see a list of all the supported encoding names, run Therion with
  |--print-encodings| option. `UTF-8' (Unicode) and `ASCII' (7\,bit) encodings
  are always supported.
\endarguments

%%%%%%%%%%%%%%%%%%%%%%%%%%%%%%%%%%%%%%%%


\subsubchapter `input'.

\description
  inserts the contents of a file in place of
  the command. Default extension is `.th' and may be omitted. For greatest
  portability use relative paths and Unix slashes `|/|', not Windows
  backslashes `|\|', as
  directory separators.
\enddescription

\syntax
  |input <file-name>|
\endsyntax

\context
  all
\endcontext

\arguments
*  |<file-name>|
\endarguments

%%%%%%%%%%%%%%%%%%%%%%%%%%%%%%%%%%%%%%%%


\subsubchapter `survey'.

\description
  Survey is the main data structure.
 % Each data object must belong to a   survey.
  Surveys may be nested---this allows a hierarchical structure
  to be built. Usually some level of this hierarchical structure
  survey represents caves, higher levels karst areas and lower levels e.g.~passages.

  Each survey has its own namespace specified by its |<id>| argument. Objects
  (like survey stations or scraps; see below) which belong to a subsurvey of
  the current survey are referenced as

  |<object-id>@<subsurvey-id>|,

  or, if there are more nesting levels

  |<object-id>@<subsubsurvey-id>.<subsurvey-id>|.\[Note: it's not possible to
  reference any object from the higher-level surveys.]

  This means, that object identifiers must be unique only in the scope of one
  survey. For instance, survey stations names can be the same if they are
  in different surveys. This allows stations to be numbered from 0 in each survey or
  the joining of two caves into one cave system without renaming survey stations.

\enddescription

\syntax
      |survey <id> [OPTIONS]
       ... other therion objects ...
       endsurvey [<id>]|
\endsyntax

\context
  none, survey
\endcontext

\arguments
*|<id>| = survey identifier
\endarguments

\options
* |namespace <on/off>| = specify whether survey creates namespace (|on|
  by default)
* |declination <specification>| = set the default declination for
  all data objects in this survey (which can be overridden by
  declination definitions in subsurveys). The |<specification>|
  has three forms:

  1. |[]| an empty string. This will reset the declination definition.

  2. |[<value> <units>]| will set a single value (also for undated surveys).

  3. |[<date1> <value1> [<date2> <value2> ... ] <units>]|
     will set declination for several dates. Then the declination
     of each shot will be set according to the date specification
     of the data object. If you want to explicitly set the declination
     for undated survey data, use `|-|' instead of date.

  If no declination is specified and some geodetic coordinate system is
  defined, the declination is automatically computed using built-in geomagnetic
  model.

  N.B.: The declination is positive when the magnetic north is east of true north.

* |person-rename <old name> <new name>| = rename a person whose name has been
  changed

* |title <string>| = description of the object

* |entrance <station-name>| = specifies the main entrance to the cave represented
  by this survey. If not specified and there is exactly one station marked entrance
  in this survey, it is considered to represent a cave also. This information is
  used for |cave-list| export.
\endoptions


%%%%%%%%%%%%%%%%%%%%%%%%%%%%%%%%%%%%%%%%


\subsubchapter `centreline'.

\description
  Survey data (centreline) specification. The syntax is borrowed from Survex
  with minor modifications; the Survex manual may be useful as an additional
  reference for the user. A synonym term `centerline' may be used.
\enddescription

\syntax
      |centreline [OPTIONS]
          date <date>
          team <person> [<roles>]
          explo-date <date>
          explo-team <person>
          instrument <quantity list> <description>
          calibrate <quantity list> <zero error> [<scale>]
          units <quantity list> [<factor>] <units>
          sd <quantity list> <value> <units>
          grade <grade list>
          declination <value> <units>
          grid-angle <value> <units>
          infer <what> <on/off>
          mark <type>
          flags <shot flags>
          station <station> <comment> [<flags>]
          cs <coordinate system>
          fix <station> [<x> <y> <z> [<std x> <std y> <std z>]]
          equate <station list>
          data <style> <readings order>
          break
          group
          endgroup
          walls <auto/on/off>
          vthreshold <number> <units>
          extend <spec> [<station> [<station>]]
          station-names <prefix> <suffix>
          ...
          [SURVEY DATA]
          ...
        endcentreline|
\endsyntax

\context
  none, survey
\endcontext

\options
  * |id <ext_keyword>| = id of the object
  * |author <date> <person>| = author of the data and its creation date
  * |copyright <date> <string>| = copyright date and name
  * |title <string>| = description of the object
\endoptions


\comopt
  * |date <date>| = survey date. If multiple dates are specified,
    a time interval is created.
  * |explo-date <date>| = discovery date. If multiple dates are specified,
    a time interval is created.
  * |team <person> [<roles>]| = a survey team member. The first argument
    is his/her name, the others describe the roles of the person in
    the team (optional---currently not used). The following role keywords are
    supported: station, [back]length, [back]tape, [back]compass, [back]bearing, [back]clino,
    [back]gradient, counter, depth, station, position, notes, pictures, pics,
    instruments (insts), assistant (dog).
  * |explo-team <person>| = a discovery team member.
  * |instrument <quantity list> <description>| = description
    of the instrument that was used to survey the given quantities (same
    keywords as team person's role)
  * |infer <what> <on/off>| = `|infer plumbs on|' tells the
    program to interpret gradients $\pm90\,^\circ$
    as UP/DOWN (this means
    no clino corrections are applied). `|infer equates on|' will case program to
    interpret shots with 0 length as equate commands (which means that no
    tape corrections are applied)
  * |declination <value> <units>| = sets the declination for subsequent
    shots $$true\ bearing = measured\ bearing + declination.$$
    The declination is positive when the magnetic north is east of true north.
    If no declination is specified, or the declination is reset (|-|),
    then a valid declination specification is searched for in all surveys
    the data object is in. See declination option of survey command.
  * |grid-angle <value> <units>| = specifies the magnetic
    grid angle (declination against grid north).
  * |sd <quantity list> <value> <units>| = sets the
    standard deviation for the given measurements. The Quantity list can
    contain the following keywords: length, tape, bearing, compass,
    gradient, clino, counter, depth, x, y, z, position, easting, dx,
    northing, dy, altitude, dz.
  * |grade <grade list>| = sets standard deviations according to the
    survey grade specification (see grade command). All previously
    specified standard deviations or grades are lost. If you want
    to change an SD, use the sd option after this command. If multiple
    grades are specified, only the last one applies. You can specify
    grades only for position or only for surveys. If you want to
    combine them, you must use them in one grade line.
  * |units <quantity list> [<factor>] <units>| = set the units
    for given measurements (same quantities as for sd).
  * |calibrate <quantity list> <zero error> [<scale>]| = set the
    instrument calibration. The measured value is calculated using the
    following formula:
    $measured\ value = (read\ value - zero\ error) \times scale$.
    The supported quantities are the same as sd.
  * |break| = can be used with interleaved data to separate two traverses
  * |mark [<station list>] <type>| =
    set the type of named stations. |<type>| is one of: fixed,
    painted and temporary (default). If there is no station list,
    all subsequent stations are marked.
  * |flags <shot flags>| = set flags for following shots. The supported
    flags are: |surface| (for surface measurements), |duplicate| (for
    duplicated surveys), |splay| (for short side legs that are hidden
    in maps and models by default). These are excluded from length
    calculations.

    All shots having one of the stations named either `|.|' or `|-|' are
    |splay| shots by default (see also |data| command).

    If flag is set to |approx[imate]|, it is included to total length
    calculations, but also displayed separately in survey statistics.
    It should be used for shots, that were not surveyed properly and need
    to be resurveyed.

    Also ``|not|'' is allowed before a flag.
  * |station <station> <comment> [<flags>]| = set the station comment
    and its flags. If |""| is specified as a comment, it is ignored.

    Supported flags: |entrance|, |continuation|, |air-draught[:winter/summer]|,
    |sink|, |spring|, |doline|, |dig|, \NEW{5.3}|arch|, |overhang|. Also |not| is allowed before a flag,
    to remove previously added flag.

    You can also specify custom attributes to the station using |attr| flag
    followed by attribute name and value. Example:\hfil\break
    |station 4 "pit to explore" continuation attr code "V"|

    If there is a passage, that was explored, but not surveyed yet, estimated
    explored length of this passage can be added to the station with
    |continuation| flag. Just add |explored <explored-length>| to the
    station flags. Explored lengths are a part of survey/cave statistics,
    displayed separately. Example:\hfil\break
    |station 40 "ugly crollway" continuation explored 100m|

  * |cs <coordinate system>| = coordinate system for stations with
    fixed coordinates
  * |fix <station> [<x> <y> <z> [<std x> <std y> <std z>]]|
    = fix station coordinates (with specified errors---only
    the units transformation, not calibration, is applied to them).
  * |equate <station list>| = set points that are equivalent
  * |data <style> <readings order>| = set data style (normal, topofil,
    diving, cartesian, cylpolar, dimensions, nosurvey) and readings order. Reading
    is one of the following keywords: station, from, to, [back]tape/[back]length,
    [back]compass/[back]bearing, [back]clino/\penalty0[back]gradient,
    depth, fromdepth, todepth, depthchange, counter,
    fromcount, tocount, northing, easting, altitude,
    up/ceiling\[dimension may be specified as a pair |[<from> <to>\char"5D|,
    meaning the size at the beginning and end of the shot],
    down/floor, left, right, ignore, ignoreall.

    See Survex manual for details.

    For interleaved data both newline and direction keywords
    are supported. If backward and forward compass or clino
    reading are given, the average of them is computed.

\NEW{5.3}    If one of the shot stations is named either `|.|' or `|-|', the shot has
    |splay| attribute set. {\it Dot} should be used for shots ending on
    features inside passage, {\it dash} for shots ending on passage walls, floor
    or ceiling. Although Therion makes no distinction between them yet, it
    should be used to improve 3D modeling in the future.

  * |group|
  * |endgroup| = |group/endgroup| pair enables the user to make
    temporary changes
    in almost any setting (calibrate, units, sd, data, flags...)
  * |walls <auto/on/off>| = turn on/off passage shape generation from
    LRUD data for subsequent shots. If set |auto|, passage is generated
    only if there is no scrap referencing given centreline.
  * |vthreshold <number> <units>| = threshold for interpreting LRUD readings
    as left-right-front-back reading perpendicular to the shot.

    If passages are horizontal (|inclination < vthreshold|),
    LR is perpendicular to the shot and UD is vertical.

    If passages are more or less vertical (|inclination > vthreshold|), even UD
    becomes perpendicular to the shot -- otherwise passages would not look very
    good. In the case of vertical shots, UD is interpreted as north-south
    dimension from the station to allow tube-like modeling of verticals.

  * |extend <spec> [<station> [<station>]]| =
    control how the centerline is extended. |<spec>| is one of the following

    |normal/reverse| = extend given and following stations
     to the same/reverse direction as previous station. If two
     stations are given---direction is applied only to given shot.

    |left/right| = same as above, but direction is specified explicitly.

    |vertical| = do not move station (shot) in $X$ direction, use only $Z$
    component of the shot

    |start| = specify starting station (shot)

    |ignore| = ignore specified station (shot), continue extended elevation
    with other station (shot) if possible

    |hide| = do not show specified station (shot) in extended elevation


    If no stations are specified, |<spec>| is valid for following shots
    specified.

  * |station-names <prefix> <suffix>| = adds given prefix/suffix to
    all survey stations in the current centreline. Saves some typing.

\endcomopt



%%%%%%%%%%%%%%%%%%%%%%%%%%%%%%%%%%%%%%%%


\subsubchapter `scrap'.

\description
  Scrap is a piece of 2D map, which doesn't contain overlapping passages
  (i.e.~all the passages may be drawn on the paper without overlapping).
  For small and simple caves, the whole cave may belong to one scrap.
  In complicated systems, a scrap is usually one chamber or one passage.
  Ideally, a scrap contains about 100~m of the cave.\[If necessary,
  scraps may be much smaller---just to display a few meters of the cave.
  {\it When deciding about scrap size please take into account the following:}
  Using small scraps may take more time for cartographer to optimize
  scrap joins. On the other hand smaller scraps will probably be less distorted
  by map warping algorithms than larger scraps.
  Using too large scraps may exhaust \MP's memory if passage fills
  are frequently used and the map editor in XTherion is much less
  responsive when editing huge scraps.]
  Each scrap is processed
  separately by \MP; scraps which are too large may exceed \MP's memory and
  cause errors.

  Scrap consists of point, line and area map symbols. See chapter {\it How
  the map is put together} for explanation how and in which order are they
  displayed.

  Scrap border consists of lines with the |-outline out| or |-outline in|
  options (passage walls have |-outline out| by default). These lines shouldn't
  intersect---otherwise Therion (\MP) can't determine the interior of the scrap
  and \MP\ issues a warning message ``|scrap outline intersects itself|''.

  Each scrap has its own local cartesian coordinate system, which usually
  corresponds with the millimeter paper (if you measure the coordinates of map
  symbols by hand) or pixels of the scanned image (if you use XTherion).
  Therion does the transformation from this local coordinate system to the
  real coordinates using the positions of survey stations, which are
  specified both in the scrap as point map symbols and in centreline data.
  If the scrap doesn't contain at least two survey stations with the |-name|
  reference, you have
  to use the |-scale| option for calibrating the scrap. (This is usual for
  cross sections.)

  The transformation consists of the  following steps:
    \list
    * Linear transformation (shifting, scaling and rotation) which `best' fits
      stations drawn in the scrap to real ones. `Best' means that the sum of
      squared distances between corresponding stations before and after
      transformation is minimal. The result is displayed red if |debug| option
      of the |layout| command is set |on|.
    * Non-linear transformation of the scrap which (1) moves survey stations
      to their correct position, (2) is continuous. Displayed blue in the |debug|
      mode.
    * Non-linear transformation of the scrap which (1) moves joined points
      together, (2) doesn't move survey stations, (3) is continuous.
      Finally the position of curves' control points is adjusted to preserve
      smoothness. The result is final map.
    \endlist

\enddescription

\syntax |scrap <id> [OPTIONS]
       ... point, line and area commands ...
       endscrap [<id>]|
\endsyntax

\context
  none, survey
\endcontext

\arguments
  *|<id>| = scrap identifier
\endarguments

\options
  * |projection <specification>| = specifies the drawing projection.
    Each projection is identified by a type and optionally by an index
    in the form |type[:index]|. The index can be any keyword. The following
    projection types are supported:

    1. |none| = no projection, used for cross sections or maps that
       are independent of survey data (e.g.~digitization of old maps where
       no centreline data are available).
       No index is allowed for this projection.

    2. |plan| = basic plan projection (default).

    3. |elevation| = orthogonal projection (a.k.a.\ projected profile) which optionally takes a view
       direction as an argument (e.g.~|[elevation 10]| or |[elevation 10 deg]|).

    4. |extended| = extended elevation (a.k.a.\ extended profile).

  * |scale <specification>| =
    is used to pre-scale (convert coordinates from pixels to
    meters) the scrap data. If scrap projection is none, this is the only
    transformation that is done with coordinates.
    The |<specification>| has four forms:

    1. |<number>| = |<number>| meters per drawing unit.

    2. |[<number> <length units>]| = |<number> <length units>| per
       drawing unit.

    3. |[<num1> <num2> <length units>]| = |<num1>| drawing units
       corresponds to |<num2> <length units>| in reality.

    4. |[<num1> ... <num8> [<length units>]]| = this is the most
       general format, where you specify, in order, the $x$ and $y$ coordinates
       of two points in the scrap and two points in reality.
       Optionally, you can also specify units for the coordinates of the `points in reality'.
       This form allows you to apply both scaling and rotation to the scrap.
  * |cs <coordinate system>| = assumes that (calibrated) local scrap coordinates are given
    in specified coordinate system. It is useful for absolute placing of
    imported sketches where no survey stations are specified.\[If there are
    some survey stations in the scrap, the |cs| specification is ignored.]
  * |stations <list of station names>| = stations you want to plot
    to the scrap, but which are not used for scrap transformation. You don't have
    to specify (draw) them with the |point station| command.
  * |sketch <filename> <x> <y>| = underlying sketch bitmap specification
    (lower left corner coordinates).
  * |walls <on/off/auto>| = specify if the scrap should be used in 3D model
    reconstruction
  * |flip (none)/horizontal/vertical| = flips the scrap after
    scale transformation
  * |station-names <prefix> <suffix>| = adds given prefix/suffix to
    all survey stations in the current scrap. Saves some typing.

  * |author <date> <person>| = author of the data and its creation date
  * |copyright <date> <string>| = copyright date and name
  * |title <string>| = description of the object
\endoptions




\subsubchapter `point'.

\description
Point is a command for drawing a point map symbol.
\enddescription

\syntax
  |point <x> <y> <type> [OPTIONS]|
\endsyntax

\context
  scrap
\endcontext

\arguments
  * |<x>| and |<y>| are the drawing coordinates of an object.
  * |<type>| determines the type of an object. The following
    types are supported:

    {\it special objects:}
    |dimensions|\[Use |-value| option to
      specify passage dimensions above/below centerline
      plane used while creating 3D model.],
    |section|\[|section| is an anchor for placing the cross-section at this
      point.  This symbol has no visual representation. The cross section
      must be in the separate scrap with `none' projection specified.
      You can specify it through the |-scrap| option.],
    |station|\[Survey station. For each scrap (with the exception of scraps
      in `none' projection) at least one station with station reference
      (|-name| option) has to be specified.];

    {\it labels:}
    |altitude|\[General altitude label.
    All altitudes are exported as a difference against grid $Z$ origin
   (which is 0 by default).
    To display altitude on the passage
    wall, use |altitude| option for any line point of the passage wall],
    |date|\[Set date's value with the |-value| option],
    |height|\[Height of formations inside of the passage (like pit etc.);
    see below for details.],
    |label|,
    |passage-height|\[Height of the passage; see below for details.],
    |remark|,
    |station-name|\[If no text is specified, the name of the nearest
      station is used.]; \penalty-100

    {\it symbolic passage fills:}\[Unlike other point symbols, these are
      clipped by the scrap border. See the chapter {\it How the map is
      put together}.]
    |bedrock|,
    |blocks|,
    |clay|,
    |debris|,
    |guano|,
    |ice|,
    |mudcrack|,
    |mud|,
    |pebbles|,
    |raft|,
    |sand|,
    |snow|,
    |water|;

    {\it speleothems:}
    |anastomosis|,
    |aragonite|,
    |cave-pearl|,
    \NEW{5.4}|clay-tree|,
    |crystal|,
    |curtains|,
    |curtain|,
    |disc-pillar|,
    |disc-stalactite|,
    |disc-stalagmite|,
    |disc-pillars|,
    |disc-stalactites|,
    |disc-stalagmites|,
    |disk|,
    |flowstone|,
    |flute|,
    |gypsum-flower|,
    |gypsum|,
    |helictites|,
    |helictite|,
    |karren|,
    |moonmilk|,
    |pendant|,
    |pillar-with-curtains|,
    |pillars-with-curtains|,
    |pillar|,
    |popcorn|,
    |raft-cone|,
    |rimstone-dam|,
    |rimstone-pool|,
    |scallop|,
    |soda-straw|,
    |stalactite-stalagmite|,
    |stalactites-stalagmites|,
    |stalactite|,
    |stalactites|,
    |stalagmite|,
    |stalagmites|,
    |volcano|,
    |wall-calcite|;

    {\it equipment:}
    |anchor|,
    |bridge|,
    |camp|,
    |fixed-ladder|,
    |gate|,
    |handrail|,
    |masonry|,
    |nameplate|,
    |no-equipment|,
    |no-wheelchair|,
    |rope-ladder|,
    |rope|,
    |steps|,
    |traverse|,
    |via-ferrata|,
    |walkway|,
    |wheelchair|;

    {\it passage ends:}
    |breakdown-choke|,
    \NEW{5.4}|clay-choke|,
    |continuation|,
    |entrance|,
    |flowstone-choke|,
    |low-end|,
    |narrow-end|;

    {\it others:}
    |air-draught|\[Number of ticks is set according to |-scale| option],
    |altar|,
    |archeo-excavation|,
    |archeo-material|,
    |audio|,
    |bat|,
    |bones|,
    |danger|,
    |dig|,
    |electric-light|,
    |ex-voto|,
    |extra|\[Additional morphing point. See |-dist| and |-value| options. ],
    |gradient|,
    |human-bones|,
    |ice-pillar|,
    \NEW{5.4}|ice-stalactite|,
    |ice-stalagmite|,
    |map-connection|\[Virtual point, used to indicate connection between shifted maps (extended elevation, map offset).],
    %|minus|,
    |paleo-material|,
    |photo|,
    %|plus-minus|,
    %|plus|,
    |root|,
    |seed-germination|,
    |sink|,
    |spring|\[Always use |spring| and |sink| symbols with a |water-flow| arrow.],
    |tree-trunk|,
    |u|\[For user defined point symbols.],
    |vegetable-debris|,
    |water-drip|,
    |water-flow|.

\endarguments


\options
  * |subtype <restr_keyword>| = determines the object's subtype. The following
    subtypes for given types are supported:

    {\it station:}\[if station subtype is not specified, Therion reads it from centreline,
   if it's specified there]
    |temporary| (default), |painted|, |natural|, |fixed|;

    {\it air-draught:} |winter|, |summer|, |undefined| (default);

    {\it water-flow:} |permanent| (default), |intermittent|, |paleo|.

    The subtype may be specified also directly in |<type>| specification using
    `|:|' as a separator.\[E.g.~|station:fixed|]

    Any subtype specification can be used with user defined type (|u|).
    In this case you need also to define corresponding metapost symbol
    (see the chapter {\it New map symbols}).

  * |orientation/orient <number>| = defines the orientation
    of the symbol. If not specified, it's oriented to north.
    0 $\le$ |number| $<$ 360.
  * |align| = alignment of the symbol or text. The following values
    are accepted: center, c, top, t, bottom, b, left, l, right, r,
    top-left, tl, top-right, tr, bottom-left, bl, bottom-right, br.
  * |scale| = symbol scale, can be:
    tiny (xs), small (s), normal (m), large (l), huge (xl) or a
    numeric value. Normal is default. Named sizes scale by $\sqrt 2$, so that
    $xs \equiv 0.5$, $s \equiv 0.707$, $m \equiv 1.0$, $l \equiv 1.414$ and
    $xl \equiv 2.0$.
  * |place <bottom/default/top>| = changes displaying order in the map.
  * |clip <on/off>| = specify whether a symbol is clipped by the scrap border.
    You cannot specify this option for the following symbols: station,
    station-name, label, remark, date, altitude, height, passage-height.
  * |visibility <on/off>| = displays/hides the symbol.
  * |context <point/line/area> <symbol-type>| = (to be used with |symbol-hide|
    and |symbol-show| layout options) symbol will be hidden/shown according
    to rules for specified |<symbol-type>|.\[Example: if you specify
    |-context point air-draught| to a label which displays the observation
    date, the |symbol-hide point air-draught| command would hide both
    air-draught arrow and the corresponding label.]
  * |id <ext_keyword>| = ID of the symbol.

    {\it Type-specific options:}\Nobreak

  * |dist <distance>| =  if the point type is extra, specifies the distance to the nearest
    station (or station specified using |-from| option. If not specified,
    appropriate value from LRUD data is used.
  * |from <station>| =  if the point type is extra, specifies reference station.
  * |name <reference>| = if the point type is station, this
    option gives the reference to the real survey station.
  * |extend [prev[ious] <station>]| = if the point type is station and scrap
    projection is extended elevation, you can
    adjust the extension of the centreline using this option.
  * |scrap <reference>| = if the point type is section, this is a
    reference to a cross-section scrap.
  * |explored <length>| = if the point type is continuation, you can specify
    length of passages explored but not surveyed yet. This value is afterwards
    displayed in survey/cave statistics.
  * |text| = text of the label, remark or continuation. It may contain
    following formatting keywords:\[For SVG output, only |<br>|, |<thsp>|,
    |<it>|, |<bf>|, |<rm>| and |<lang:XX>| keywords are taken into account;
    all others are silently ignored.]

    |<br>| = line break

    |<center>|/|<centre>|, |<left>|, |<right>| = line alignment for multi-line labels.
    Ignored if there is no |<br>| tag.

    |<thsp>| = thin space

    |<rm>|, |<it>|, |<bf>|, |<ss>|, |<si>| = font switches

\NEW{5.3}    |<rtl>| and |</rtl>| = marks beginning and end of a right-to-left written
    text

\NEW{5.3}|<lang:XX>| = creates multilingual label (see |string| type for
   detailed description)

    \NEW{6.1.0}|<size:N>| = specify the font size in points; |N| should be
    an integer between 1 and 127.

    \NEW{6.1.1}|<size:N%>| = specify the font size as a percentage of the native
    font size of the given label; |N| should be between 1 and 999.\[For practical
    reasons, the values are currently used in the increments of 10, so both 46 and 53
    are interpreted as 50~\% size.]

    \NEW{6.1.1}|<size:S>| = specify the font size using predefined scales;
    |S| can be one of |xs|, |s|, |m|, |l|, |xl|.

  * |value| = value of height, passage-height, altitude, dimensions or date

      {\it height:} according to the sign of the value (positive, negative or
      unsigned), this type of symbol represents chimney height, pit depth
      or step height in general. The numeric value can be optionally followed by `|?|',
      if the value is presumed and units can be added
      (e.g.~|-value [40? ft]|).

      {\it passage-height:} the following four forms of value are supported:
      |+<number>| (the height of the ceiling), |-<number>| (the depth of the
      floor or water depth), |<number>| (the distance between floor
      and ceiling) and |[+<number> -<number>]| (the distance to ceiling and
      distance to floor).

      {\it altitude:} the value specified is the altitude difference from
      the nearest station. The value will be set to 0 if defined as `|-|', `|.|',
      `|nan|', `|NAN|' or `|NaN|'. If the altitude value is prefixed by `|fix|'
      (e.g. |-value [fix 1300]|), this value is used as an absolute altitude.
      The value can optionally be followed by length units.

      {\it dimensions:}
      |-value [<above> <below> [<units>]]|
      specifies passage dimensions a\-bo\-ve/below centerline
      plane used in 3D model.

      {\it date:}
      |-value <date>|
      sets the date for the date point.

\endoptions


\subsubchapter `line'.

\description
Line is a command for drawing a line symbol on the map. Each line symbol is
oriented and its visualization may depend on its orientation (e.g.~pitch edge
ticks). The general rule is that the free space is on the left, rock on the
right. Examples: the lower side of a pitch, higher side of a chimney and
interior of a passage are on the left side of pitch, chimney or wall symbols,
respectively.
\enddescription

\syntax
  |line <type> [OPTIONS]
         [OPTIONS]
         ...
         [LINE DATA]
         ...
         [OPTIONS]
         ...
         [LINE DATA]
         ...
       endline|
\endsyntax

\context
  scrap
\endcontext

\arguments
   * |<type>| is a keyword that determines the type of line.
     The following types are supported:

     {\it passages:} |wall|, |contour|,
     |slope|\[Slope line marks upper border of the sloping area. It's
       necessary to specify |l-size| in at least one point. Gradient lines
       length and orientation is an average of specified |l-size|s and
       |orientation|s in the nearest points. If there is no orientation
       specification, gradient marks are perpendicular to the slope line.],
     |floor-step|, 
     |pit|,
     |pitch| (synonym of pit),
     |ceiling-step|, 
     |chimney|, 
     |overhang|, 
     |ceiling-meander|,
     |floor-meander|,
     |low-ceiling|,
     |pit-chimney|;

     {\it passage fills:} |flowstone|, |moonmilk|,
     |rock-border|\[Outer outline of large boulders. If the line is closed,
       it is filled with the background colour.],
     |rock-edge|\[Inner edges of large boulders.],
     |water-flow|,
     |abyss-entrance|,
     |dripline|,
     |fault|,
     |gradient|,
     |joint|,
     |rimstone-dam|,
     |rimstone-pool|;

    {\it equipment:}
    |fixed-ladder|,
    |handrail|,
    |rope|,
    |rope-ladder|,
    |steps|,
    |via-ferrata|,
    |walkway|;

     {\it labels:} |label|;

     {\it special:} |border|, |arrow|,
     |section|\[Line showing cross-section position.\NEW{5.3}\
       If both control points (red dots) of a B\'ezier curve (grey line) are given
       then the section line (blue) is drawn up to the perpendicular projection (dotted) of
       the first control point and from the projection (dotted) of the section control
       point. No section curve is displayed.\hfil\break\MPpic{xsect.1}],
     |survey|\[Survey line is automatically drawn by Therion.],
     |map-connection|\[Used to indicate connection between maps (in offset,
     or the same points in extended elevation).], |u|\[For user defined line symbols.].
\endarguments


\comopt
    Most of these options are accepted both as a |line| command option and as a
    |[LINE DATA]| option despite representing a 'whole line' option. The last 
    one appearing prevails. The exceptions are |adjust|, |altitude|, |l-size|,
    |mark|, |orientation|, |size| and |smooth| that are only accepted as a 
    |[LINE DATA]| option.

    The |direction| and |gradient| options are only accepted as a |[LINE DATA]| 
    command option when set to |point|. Otherwise they are accepted as a |line| 
    command option.
       * |subtype <restr_keyword>| = determines line subtype. The following
         subtypes are supported for given types:

         {\it wall:} |invisible|, |bedrock| (default), |sand|, |clay|,
         |pebbles|, |debris|, |blocks|, |ice|, |underlying|, \NEW{5.4}|overlying|, |unsurveyed|,
         |presumed|, |pit|\[Usually open to surface.], |flowstone|, |moonmilk|;

         {\it border:} |visible| (default), |invisible|, |temporary|,
         |presumed|;

         {\it water-flow:} |permanent| (default), |conjectural|, |intermittent|;

	 {\it survey:} |cave| (default), |surface| (default if centreline has
	    surface flag).

    The subtype may be specified also directly in |<type>| specification using
    `|:|' as a separator.\[E.g.~|border:invisible|]

    Any subtype specification can be used with user defined type (|u|).
    In this case you need also to define corresponding metapost symbol
    (see the chapter {\it New map symbols}).

       * |[LINE DATA]| specify either the coordinates of a line segment
         |<x> <y>|, or coordinates of a B\'ezier curve arc
         |<c1x> <c1y> <c2x> <c2y> <x> <y>|, where |c| indicates the control
         point.
       * |close <on/off/auto>| = determines whether a line is closed
         or not
       * |mark <keyword>| = is used to mark the point on the line (see
         |join| command).
       * |outline <in/out/none>| = determines whether the line serves as
         a border line for a scrap. Default value is `|out|' for
         walls, `|none|' for all other lines. Use |-outline in| for
         large pillars etc.
       * |reverse <on/off>| = whether points are given in reverse order.
       * |smooth <on/off/auto>| = whether the line is smooth at the given point.
         Auto is default.
    * |adjust <horizontal/vertical>| = shifts the line point to be aligned
    horizontally/ver\-ti\-cally with the previous point. It can't be set on the 
    first point. The result is a horizontal/vertical line segment.
    This option is not allowed in the |plan| projection.
  * |place <bottom/default/top>| = changes displaying order in the map.
  * |clip <on/off>| = specify whether a symbol is clipped by the scrap border.
  * |visibility <on/off>| = displays/hides the symbol.
  * |context <point/line/area> <symbol-type>| = (to be used with |symbol-hide|
    and |symbol-show| layout options) symbol will be hidden/shown according
    to rules for specified |<symbol-type>|.

    {\it Type-specific options:}\Nobreak

       * |altitude <value>| = can be specified only with the wall type.
         This option creates an altitude label on the wall.
    All altitudes are exported as a difference against grid $Z$ origin
   (which is 0 by default).
         If the value is specified, it
         gives the altitude difference of the point on the wall
         relative to the nearest station. The value will be set to 0 if defined 
         as "-", ".", "nan", "NAN" or "NaN". The value can be prefixed
         by a keyword ``|fix|'', then no nearest station is taken into
         consideration; the absolute given value is used instead.
         Units can follow the value. Examples: |+4|, |[+4 m]|,
         |[fix 1510 m]|.
       * |anchors <on/off>| = this option can be specified only with
         the `rope' line type. Default is |on|.
       * |border <on/off>| = this option can be specified only with
         the `slope' symbol type. It switches on/off the border line of
         the slope.
       * |direction <begin/end/both/none/point>| = can be used only
         with the section type. It indicates where to put
         a direction arrow on the section line. None is default. The |point| 
         option must be used inside |[LINE DATA]|. The others can (and should)
         be used as a line option.
       * |gradient <none/center/point>| = can be used only with the contour
         type and indicates where to put a gradient mark on the contour line.
         If there is no gradient specification, behaviour is symbol-set
         dependent (e.g. no tick in UIS, tick in the middle in SKBB). The |point| 
         option must be used inside |[LINE DATA]|. The others can (and should)
         be used as a line option.
       * |head <begin/end/both/none>| = can be used only with the arrow
         type and indicates where to put an arrow head. End is default.
       * |l-size <number>| = Size of the line (to the left). Only valid on and 
         required for |slope| type.
       * |orientation/orient <number>| = orientation of the symbols on the line.
         Can be used only with slope type. If not specified, it's perpendicular 
         to the line on its left side. 0 $\le$ |number| $<$ 360.
       * |rebelays <on/off>| = this option can be specified only with
         the `rope' line type. Default is |on|.
       * |scale| = scale affects only the text on label lines, can be:
         tiny (xs), small (s), normal (m), large (l), huge (xl) or a
         numeric value. Normal is default. Named sizes scale by $\sqrt 2$, so that
         $xs \equiv 0.5$, $s \equiv 0.707$, $m \equiv 1.0$, $l \equiv 1.414$ and
         $xl \equiv 2.0$. Absolute font sizes (in points) can be assigned to named sizes
         using |fonts-setup| in the |layout| configuration section.
       * |size <number>| = synonym of l-size
       * |text <string>| = valid only for label lines.
       * \NEW{5.4}|height <value>| = height of pit or wall:pit; available in
         \MP\ as a numeric variable |ATTR__height|.
\endcomopt

\options
  * |id <ext_keyword>| = ID of the symbol.
\endoptions


\subsubchapter `area'.

\description
Area is specified by surrounding border lines. They may be of any type, but
must be listed in order and each pair of consecutive lines must intersect.
In order to be sure that lines intersect even after scrap transformation
you may e.g.~continue a lake border 1\,cm behind a passage wall---these
overlaps will be automatically clipped by scrap border. You may use 
invisible border to achieve this inside of the passage. When defined by more
than one line, the actual area will be the intersection of the areas defined 
by each separate line.
\enddescription

\syntax
  |area <type>
         place <bottom/default/top>
         clip <on/off>
         visibility <on/off>
       ... border line references ...
       endarea|
\endsyntax

\context
  scrap
\endcontext

\arguments
  * |<type>| is one of following: |water|, |sump|, |sand|, |debris|,
    |blocks|, |flowstone|, |moonmilk|, |snow|, |ice|, |clay|, |pebbles|,
    |bedrock|\[An empty area which
    can be used to clean the background.], |u|\[For user defined area symbols,
    may be followed by arbitrary subtype.],
    |mudcrack|,
    |pillar|,
    |pillar-with-curtains|,
    |pillars|,
    |pillars-with-curtains|,
    |stalactite|,
    |stalactite-stalagmite|,
    |stalagmite|.
\endarguments

\comopt
All options can appear as command-like options or as |area| options, i.e., on
the same line as the |area| command. The last one appearing prevails.
  * the data lines consist of border line references (IDs)
  * |place <bottom/default/top>| = changes displaying order in the map.
  * |clip <on/off>| = specify whether a symbol is clipped by the scrap border.
  * |visibility <on/off>| = displays/hides the symbol.
  * |context <point/line/area> <symbol-type>| = (to be used with |symbol-hide|
    and |symbol-show| layout options) symbol will be hidden/shown according
    to rules for specified |<symbol-type>|.
\endcomopt

\options
  * |id <ext_keyword>| = ID of the symbol.
\endoptions


\subsubchapter `join'.

\description
  Join works in two modes: it joins either two scraps or two or more points
  or lines in a map together.

  When joining more than two points or lines, use one join command for
  all of them, not a sequence of join commands for pairs.\[E.g. use
  |join a b c|, not |join a b| followed by |join b c|.]

  When joining scraps, only passage walls are joined.
  It's a good idea to place a scrap join in the passage which is as simple
  as possible, otherwise you have to specify join for each pair of objects
  which should be joined.
  \[If you want some object which is clipped by a scrap boundary to continue
  to a neighbouring scrap, use |-clip off| option for that object.]

  When joining more than two scraps at the same scrap border, a manual
  join must be performed where the connection points must be entered
  in one join statement.
  \[Like |join origScrapLineWest:end upperScrapLineWest:0 lowerScrapLineWest:0| % line was too long
    and another similar join command for the three east wall lines.]

\enddescription

\syntax
  |join <point1> <point2> ... <pointN> [OPTIONS]|
\endsyntax

\context
  none, scrap, survey
\endcontext

\arguments
   * |<pointX>| can be an ID of a point or line symbol,
     optionally followed by a line point mark |<id>:<mark>|
     (e.g.~|podangl_l31@podangl:mark1|).
     |<mark>| can be also `|end|' (end of the line) or line point index
     (where 0 is the first point).

     A special case is when |<point1>| and |<point2>| are scrap
     IDs---than the closest scrap ends are joined together.
\endarguments

\options
  * |smooth <on/off>| indicates whether two lines are to be connected
    smoothly.
  * |count <N>| (when used with scraps) = Therion will try to join scraps
    which connect in |N| locations/passages.
\endoptions


\subsubchapter `equate'.

\description
  Sets the survey stations equivalence.
\enddescription

\syntax
  |equate <station list>|
\endsyntax

\context
none, survey
\endcontext


\subsubchapter `map'.

\description
  A map is a collection of either scraps or other maps of the same projection
  type. It's possible to include survey in the map---this will display
  centreline in the map.
  Map object simplifies the data management when selecting data for output.
  See the chapter {\it How the map is put together} for more thorough
  explanation.

  (Note: |break| only changes level of maps of scraps and has no function when
   used with maps of maps, as they will cause a |break| implicitly)
\enddescription

\syntax
  |map <id> [OPTIONS]
        ... scrap, survey or other map references ...
        break
        ... next level scrap, survey or other map references ...
        preview <above/below> <other map id>
      endmap|
\endsyntax

\context
  none, survey
\endcontext

\arguments
  *|<id>| = scrap identifier
\endarguments

\comopt
  * the data lines consist of scrap or map references. Note that
    you can not mix them together.
  * if you refer to map, you can specify offset at which this
    sub-map will be displayed together with preview type of its
    original position. Syntax is following:\hfil\break
    |<map reference> [<offset X> <offset Y> <units>] <above/below/none>|
  * scraps following the |break| will be placed on another level (only
    applies to maps consisting of scraps)
  * |preview <above/below> <other map id>| will put the outline of
    the other map in the specified preview position relative to the
    current map.

    Preview is displayed only if the map is in the |map-level| level as
    specified by the |select| command.

    Use the revise command if you want to add maps from higher levels to the
    preview.
  * |colo[u]r <color>| = set the map colour; this option overrides the automatic
    choice when the layout specifies |colour map-fg [map]|.
\endcomopt

\options
  * |projection/proj <plan/elevation/extended/none>| = required if the map
    contains survey.
  * |title <string>| = description of the object
  * |survey <id>| =\NEW{5.4} associate a survey with map
    (e.g.\ all surveying statistics from this survey will be used when
    this map is selected for output).
\endoptions

\subsubchapter `surface'.

\description
Surface (terrain) specification. It is possible to display it in two ways: as a
scanned topographical map (both in 2D map and 3D model\[You need to enter
elevation data in order to display the topographical map in 3D model. Currently
only JPEG maps are supported in 3D.])
or surface grid -- digital elevation model (in 3D model only).
\enddescription

\syntax
|surface [<name>]
   cs <coordinate system>
   bitmap <filename> <calibration>
   grid-units <units>
   grid <origin x> <origin y> <x spacing> <y spacing> <x count> <y count>
   grid-flip (none)/vertical/horizontal
   [grid data]
endsurface|
\endsyntax

\context
  none, survey
\endcontext

\comopt
* |cs <coordinate system>| = coordinate system for bitmap calibration
  and grid origin specification
* |bitmap <filename> <calibration>| = scanned topographical map.

  |calibration| may have two forms:

  1. |[X1 Y1 x1 y1 X2 Y2 x2 y2 [units]]|, where upper case X/Y variables
  are picture coordinates (pixels; lower-left corner is |0 0|), lower-case
  x/y variables are real coordinates. Optional units apply to real coordinates
  (metres by default).

  2. |[X1 Y1 station1 X2 Y2 station2]|, where upper case X/Y variables
  are picture coordinates and |station1| and |station2| are survey
  stations names.

* |grid-units <units>| = units in which grid is specified. Metres by default.

* |grid <origin x> <origin y> <x spacing> <y spacing> <x count> <y count>|

  |<origin x> <origin y>| = specify coordinates of the lower-left (S-W)
  corner of the grid

  |<x spacing> <y spacing>| = distance between grid nodes in W-E and S-N
  directions

  |<x count> <y count>| = number of nodes in the row and number of rows
  which form the grid (see below).

* |[grid data]| = a stream of numbers giving the altitude a.s.l.~in grid nodes.
  It starts in the grid-origin and fills the grid in rows
  (in the row from W to E; rows from S to N).

 * |grid-flip (none)/vertical/horizontal| = useful if your grid (exported
   from other program) needs to be flipped

\endcomopt


\subsubchapter `import'.

\description
  Reads survey data in different formats (currently processed centreline in
  *.3d, *.plt, *.xyz formats). Survey stations may be referenced in scraps
  etc. When importing a Survex 3D file, stations are inserted in the survey
  hierarchy if there exists an identical hierarchy to that in 3D file.
\enddescription

\syntax
  |import <file-name> [OPTIONS]|
\endsyntax

\context
survey / all\[only with .3d files, where survey structure is specified]
\endcontext

\options
  * |filter <prefix>| = if specified, only stations with given prefix
    and shots between  them will be imported. Prefix will be removed
    from station names.
  * |surveys (create)/use/ignore| =
     specifies how to import survey structure (works only with .3d files).

     |create| = split stations into subsurveys, if subsurveys do not
                exist, create them

     |use| = split stations into existing subsurveys

     |ignore| = do not split stations into sub-surveys
  * |cs <coordinate system>| = coordinate system for stations with
    fixed coordinates
  * |calibrate [<x> <y> <z> <X> <Y> <Z>]| = coordinates in the imported
    file are shifted from lower-case coordinates to upper-case coordinates.
\endoptions


\subsubchapter `grade'.

\description
   This command is used to store predefined precisions of centreline data.
   Built in grades are: BCRA\[see \www{http://bcra.org.uk/surveying/};
     syntax is: |BCRA{\it n}|, where |{\it n}| may be |3| or |5|]
   and UISv1\[see \www{http://www.uisic.uis-speleo.org/UISmappingGrades.pdf};
     syntax is: |UISv1\_{\it n}|\NEW{5.4}, where |{\it n}| is |-1| to |6| or |X|; whereas
     |-1| to |2| are only declaratory and |X| requires |sd| data in |centerline|)].

   See |sd| option description for |centreline| command to define your own grades.
\enddescription

\syntax:
  |grade <id>
        ...
        [<quantity list> <value> <units>]
        ...
        endgrade|
\endsyntax

\context
  all
\endcontext



\subsubchapter `revise'.

\description
  This command is used to set or change properties of an already
  existing object.
\enddescription

\syntax
  The syntax of this command for
  object created with ``single line'' command is

  |revise id [-option1 value1 -option2 value2 ...]|

  For objects created with ``multi line'' commands is syntax following

|revise id [-option1 value1 -option2 value2 ...]
  ...
  optionX valueX
  data
  ...
endrevise|
\endsyntax

\context
  all
\endcontext

\arguments
  The id stands for object identifier (the id of an object you want to
  revise must always be specified).
\endarguments

%%%%%%%%%%%%%%%%%%%%%%%%%%%%%%%%%%%%%%%%


\subsubchapter `require'.

\description
  Checks that the version of Therion in use is at least |<version>| and stops
  with an error if not.
\enddescription

\syntax
  |require <version>|
\endsyntax

\context
  all
\endcontext

\arguments
* |<version>| = The minimum supported Therion version in the form
  |major[.minor[.patch]]|, e.g.~|6.3|.
\endarguments

%%%%%%%%%%%%%%%%%%%%%%%%%%%%%%%%%%%%%%%%


\subchapter Custom attributes.

Objects {\it survey}, {\it centreline}, {\it scrap}, {\it point}, {\it
line}, {\it area}, {\it map} and {\it surface} can contain user-defined
attributes in a form |-attr <name> <value>|. |<name>| may contain
alphanumeric characters, |<value>| is a string.

The custom attributes are used in map exports depending on the output
format:
\list
* in {\it shapefile} export they are written directly to the associated
  dbf file,
* in maps generated using \MP\ (PDF, SVG) the attributes are written in
  the \MP\ source file as strings (named like |ATTR_<name>|) and can be
  evaluated and used by the user to define symbols in macros.

  You can test presence of such a variable using |if known ATTR_<name>: ... fi|.

\endlist


\subchapter XTherion.

XTherion is a GUI (Graphical User Interface) for Therion.
It helps a lot with creating input data files. Currently it works in
three main modes: text editor, map editor and compiler.
\[Here we're concerned with creating data, so only the first two modes are
described in this section. For compiler features see the chapter
{\it Processing data}.]

It is not necessary for Therion itself---you may edit input files in your
favourite text editor and run Therion from the command line. XTherion is also
not the only GUI which may be used with Therion. It is possible to
write a better one, which would be more user friendly, more WYSIWYG, faster,
more robust and easier to use. Any volunteers?

This manual does not describe such familiar things as `if you want to save a file, go to
menu File and select Save, or press Ctrl-s'. Browse the top menu for a minute
to get the feeling of XTherion.

For each mode of operation, there is an additional
right or left menu. The submenus may be packed; you may unpack them by
clicking on the menu button. For most of the menus and buttons, there is a short
(translated) description in the status line, so it's not hard to guess the meaning of each one.
The order of submenus on the side may be customized by the user. Right-click on
the menu button and select in the menu which of the other menus it should
be swapped with.


\subsubchapter XTherion---text editor.

XTherion's text editor offers some interesting features which may help with
creating text input files: support for Unicode encoding and ability to open
multiple files.\[File encoding is specified on the first line of the file. This
line is hidden by XTherion and may be accessed only indirectly using the right-hand
menu.]

To make entering data easy, it supports table formatting of centreline data.
There is a menu {\it Data table} for typing the data. It may be customized to the
user's data order by pressing a {\it Scan data format} button when the cursor
is below the data order specification (`data' option in the `centreline'
command).


\subsubchapter XTherion---map editor.

Map editor allows you to draw and edit maps fully interactively.
But don't expect too much. XTherion is not a truly WYSIWYG editor. It
displays only the position, not the actual shape, of drawn point or line
symbols. Visually there is no difference between a helictite and a text
label---both are rendered as simple dots. The type and other attributes of any
object are specified only in the {\it Point control} and {\it Line control} menus.


\ifx\pdfoutput\undefined\else
\leavevmode\llap{\smash{\raise10pt\hbox{\pdfannot width 6cm height 0cm depth  4cm
{/Subtype /Text
 /Name /Help
 /Contents (Hints: 1. What does loop closure do?
            2. Why do we use MetaPost?)
}}} \qquad}\fi
{\it Exercise:} Find two substantial reasons, why the map drawn in XTherion can't be
identical with Therion output. (If you answer this, you'll know, why XTherion
will never be a true WYSIWYG editor. Authors' laziness is not the correct
answer.)

Let's begin by describing typical use of the map editor. First, you have
to decide which part of the cave (which scrap) you'll draw.\[It's possible to
draw more than one scrap in each file, in which case all inactive scraps are rendered
yellow.]

After creating a new file in the map editor, you may load one or more
{\bf images}---scanned survey sketches from the cave\[XTherion can't scale nor
rotate individual images, so use the same orientation, scale and DPI for all
images used in the same scrap.]---as a background for
the drawing. Click on the {\it Insert} button in {\it Background images} menu.
Unfortunately, as a limitation of Tcl/Tk language, only GIF, PNM and PPM
(plus PNG and JPEG if you installed tkImg extension) images are supported.
Additionally XTherion supports XVI (XTherion vector image) format, which
displays centreline and LRUD information on the background, and PocketTopo
data exported in Therion format (see below).
All opened images are placed in the upper-left corner of
the working area. Move them by double clicking and dragging with the right
mouse button or through a menu. For better performance on slower computers,
it's possible to temporarily unload a currently unused image from memory
by unchecking its {\it Visibility} check-box. It's possible to open an existing
file without loading background images using {\it Open XP} menu.
\[{\it Note:} Therion doesn't use background images in any way unless you
assign them to some scrap using |-sketch| option.]

The size and zoom setting of the {\bf drawing area} is adjusted in the
corresponding menu. {\it Auto adjust} calculates optimal size of the working
area according to the sizes and positions of loaded background images.

After these preparation steps, you're ready for drawing, or, more
precisely, for {\bf creating a map data file}. It's important to remember,
that you're actually creating a text file which should conform to the syntax
described in the chapter {\it Data format}. Actually, only a subset of the
Therion commands are used in the Map editor: multi-line |scrap ... endscrap|
commands which may contain |point|, |line| and |area| commands. (Cf.~chapter
{\it Data format}). This corresponds with a section of hand-drawn maps, which are
built up from points, lines and filled areas.

So, the first step is defining the {\bf scrap} by a |scrap ... endscrap|
multi-line command.  In the {\it File commands} menu click on the {\it Action}
submenu and select {\it Insert scrap}. This changes the {\it Action button} to
{\it Insert scrap} if it had any other value. After pressing this button a
new scrap will be inserted in the beginning of the file. You should see lines

|scrap - scrap1
endscrap
end of file|

in the preview window above the {\it Insert scrap} button. This window is a
simplified outline of the text file, which will be saved by XTherion. Only
the command (|scrap|, |point|, |line|, |text|---why text see below) and its
type (for |point| and |line|) or ID (for |scrap|) are shown.

The full contents of any command are displayed in the {\it Command preview}
menu.

For modifying previously-created commands, there are additional
menus---e.g.~{\it Scrap control} for the |scrap| command. Here you can
change the ID (very important!) and other options.
For details see chapter {\it Data format}.

Now it's possible to insert some {\bf point symbols}. As with scrap
insertion, go to the {\it File commands} menu, click on the {\it Action}
submenu and select {\it Insert point}; than press newly renamed {\it Insert
point} button. A shortcut for all this is Ctrl-p. Than click on the desired
spot in the working area and you'll see a blue dot representing a point
symbol. Its attributes can be adjusted in the {\it Point control} menu.
You'll stay in `insert' mode---each click on the working area adds a new
point symbol. Take care not to click twice on the same place---you would insert
two point symbols in the same place!
To escape from `insert' to `select' mode, press {\it Esc} key
on the keyboard or {\it Select} button in the {\it File commands} menu.

What will be the order of commands in the output file? Exactly the same as in the
outline in the {\it File commands} menu. Newly created point, line and text objects
are added before the currently marked line in the outline. It is possible to
change the order by selecting a line and pressing {\it Move down},
{\it Move up} or {\it Move to} buttons in the {\it File commands} menu. This way
you can also move objects between scraps.

{\bf Drawing lines} is similar to drawing in other
vector editing programs, which work with B\'ezier curves.
(Guess how to enter the line insertion mode, other than
using the shortcut Ctrl-l.) Click where the first point should be, then drag the
mouse with pressed left button and release it where the first control point
should be. Then click somewhere else (this point will be the second point of
the curve) and drag the mouse (adjusting the second control point of the
previous arc and the first control point of the next one simultaneously.) If
this explanation sounds too obscure, you can
get some practise working in some of the standard vector editors with
comprehensive documentation. The line will be finished after escaping from the
insertion mode. Beginning and orientation of the line is marked by a small
orange tick to the left at the first point.

For line symbols, there are two control menus: {\it Line control} and {\it Line
point control}. First one sets attributes for the whole curve, like type or
name. The check-box {\it reverse} is important: Therion requires oriented
curves and it is not unusual that you begin to draw from the wrong end.
The {\it Line point control} menu enables you to adjust the attributes of any selected
point on the line, such as the curve being smooth at this point (which is on
by default), or the presence of neighbouring control points (`|<<|' and `|>>|'
check-boxes).

{\bf Areas} are specified by their surrounding lines. Click on {\it Insert area}
and then click on the lines surrounding the desired area. They are
automatically inserted in the {\it Area control} and named (if not already
named). An alternate way is to insert them as a |text|%
\[CAUTION! The command |text| is not a Therion command! It's only a nickname for a
block of arbitrary text in
XTherion. In the file saved by XTherion, there'll only be
whatever you type into the {\it Text editor} or see in the {\it Command preview}.
It may be an area definition or whatever you want, such as a comment beginning
with a `|\#|' character.]
command, the contents of which (entered in the {\it Text editor} menu of the Map editor)
is usual |area ... endarea| multi-line command (see the chapter {\it Data
format}.)

If you draw some scraps with |none| projection, it's necessary to
{\bf calibrate} the
drawing area. The scale can be defined only one way in XTherion---using
coordinates of two points (specified both in the picture coordinate system and
in the `real' coordinate system).

After selecting a scrap (click on its header in the
{\it File commands} menu) two small red squares connected by a red arrow
will appear (by default, they'll be in the lower corners of drawing area).
You have to drag them to points with known coordinates---usually intersections
of mm grid lines on the scanned drawing. If you can not see these points,
you can either
\list
* press {\it Scale} button in the {\it Scraps} menu and click on two different
places on the image where the endpoints of calibration arrow should be, or
* move the mouse pointer to the desired position, read pointer coordinates from
the status bar and enter these coordinates into {\it picture scale points}
boxes in the {\it Scraps} control. After filling X1,Y1 and
X2,Y2 coordinate pairs the calibration arrow will be moved correspondingly.
\endlist
Then you have to enter real coordinates of these points (X1, Y1, X2, Y2).

In the {\bf selection mode} you can select existing line or point objects and
set their attributes in the corresponding menus, move them, or delete them (Ctrl-d or
{\it Action button} in {\it File commands menu} after setting {\it Action} to
{\it Delete}).

% [adding or deleting a point on the curve]

There is a {\it Search and select} menu which makes it easy to
switch between objects and visualize things you can't see at the first
look at the picture. For example, if you enter expression `station' and
press {\it Show All}, all stations on the picture will become red.

XTherion doesn't do any syntax checking; it only writes drawn objects with their
attributes to a text file. Any errors are detected only when you process these
files with Therion.

TIP: Entering symbols of the same type at once saves you a lot of time
because you need not to change the symbol type and fill options for each new symbol.
{\it Options} box preserves the old value and it's enough to change a few
characters. \[In the case of survey stations, XTherion automatically
increases the station number for the next symbol inserted.]
It is a good idea to start with drawing all survey stations (don't forget to
give them names according to the real names in the centreline command), than all
passage walls followed by all other point symbols, lines and areas. Finally,
draw cross-sections.



\subsubchapter Additional tools.

\NEW{5.3}{\bf Help/Calibrate bitmap} produces OziExplorer-compatible MAP file based
on georeferencing data included in PDF maps produced by Therion versions 5.3--6.2.\NEW{6.3}\[Calibration information for
nine distinct points is present if the centreline contains
station(s) fixed using geodetic coordinate system(s).]\,\[Therion 6.3+ uses standard
geospatial PDF files where such a calibration is not needed.]

If the map in PDF format has been converted to raster using an external program,
the converter uses raster image {\it and} pdf map with the same base name
located in the same directory to calculate the calibration data.

If the PDF file is used directly, you have to set the DPI and output format
before automatic conversion\[|ghostscript| and |convert| should be installed
on your system. Note, that Windows installation does not include |ghostscript|.]
to a raster format.

\NEW{5.3}{\bf PocketTopo data} exported in Therion format\[This is a special text
format which needs to be imported using XTherion and can not be processed
by Therion directly.]
from PocketTopo application can be imported in the text editor as well as in the map
editor ({\it File $\to$ Import $\to$ PocketTopo Therion export} and
{\it Background Images $\to$ Insert $\to$ PocketTopo Therion export}). The
same file is used for both imports. Importing the sketch does not create scrap
data directly. The drawing is just displayed on the background like scanned
bitmaps and should be digitized manually.



\subsubchapter Keyboard and mouse shortcuts in the Map editor.
{\it General}
\list
 * Ctrl+Z = undo
 * Ctrl+Y = redo
 * F9 = compile current project
 * to select an object in the listbox using the keyboard:
    switch using `Tab' into the desired listbox;
    move with the underlined cursor to the desired object;
    press `Space'
 * PageUp/PageDown = scroll up/down in the side panel
 * Shift+PageUp/PageDown = scroll up/down in the file commands window
\endlist

{\it Drawing area and background images}
\list
 * RightClick = scroll the drawing area
 * Double RightClick on the image = move the image
\endlist

{\it Inserting a scrap}
\list
 * Ctrl+R = insert scrap
\endlist

{\it Inserting a line}
\list
 * Crtl+L = insert a new line and enter an `insert line point' mode
 * LeftClick = insert a line point (without control points)
 * Ctrl+LeftClick = insert a line point very close to the existing point
    (normally it's inserted right above closest existing point)
 * LeftClick + drag = insert line point (with control points)
 * hold Ctrl while dragging = fix the distance of the previous control point
 * LeftClick + drag on the control point = move its position
 * RightClick on one of the previous points = select the previous point while
    in insert mode (useful if you want to change also the direction of
    the previous control point)
 * Esc or LeftClick on the last point = end the line insertion
 * LeftClick on the first line point = close the line and end the line insertion
\endlist

{\it Editing a line}
\list
 * LeftClick + drag = move the line point
 * Ctrl+LeftClick + drag = move the line point close to the existing
    point (normally it is moved right above the closest existing point)
 * LeftClick on control point + drag = move the control point
\endlist

{\it Adding a line point}
\list
  * select the point before which you want to insert points;
    insert required points;
    press Esc or left-click on the point you selected at the beginning
\endlist

{\it Deleting a line point}
\list
  * select the point you want to delete;
    press {\it Edit line} $\to$ {\it Delete point} in the {\it Line control}
    panel
\endlist

{\it Splitting a line}
\list
 * select the point at which you want to split the line;
    press {\it Edit line} $\to$ {\it Split line} in the {\it Line control}
    panel
\endlist


{\it Inserting a point}
\list
 * Ctrl+P = switch to `insert point' mode
 * LeftClick = insert point at a given position
 * Ctrl+LeftClick = insert point very close to the existing point (normally it
    will be inserted right above the closest point)
 * Esc = escape from the `inset point' mode
\endlist

{\it Editing a point}
\list
 * LeftClick + drag = move the point
 * Ctrl+LeftClick + drag = move the point close to the existing
    point (normally it is moved right above the closest existing point)
 * LeftClick + drag on the point arrow = change point orientation or
    size (according to the given switches in the {\it Point control} panel)
\endlist

{\it Inserting an area}
\list
 * press Ctrl+A or {\it File commands} $\to$ {\it Insert} $\to$ {\it area}
  to switch to the `insert area border' mode
 * RightClick on the lines, that surround the desired area
 * Esc to finish the area border lines insertion
\endlist

{\it Editing an area}
\list
 * select the area you want to edit
 * press `Insert' in the {\it Area control} to insert other border lines
    at the current cursor position
 * press `Insert ID' to insert a border with a given ID at the current cursor position
 * press `Delete' to remove the selected area border line
\endlist


{\it Selecting an existing object}
\list
 * LeftClick = select the object at the top
 * RightClick = select the object right below the top object (useful when several
    points lie above each other)
\endlist



\subchapter Thinking in Therion.

Although everything (well, almost everything) about Therion input files has been explained, this chapter
offers some additional hints and tips.

\subsubchapter How to enter a centreline.

The basic building block is the |centreline| command.
If the cave is larger than a few meters it's a good idea to split the data into more
files and separate the centreline data from the map data.

We usually use one |*.th| file containing a centreline per survey trip.
It's handy to
start with an empty template file as shown below, where dots will be replaced
with appropriate texts.

|encoding ISO8859-1
survey ... -title "..."
  centreline
    team "..."
    team "..."
    date ...
    units clino compass grad
    data normal from to compass clino length
      ... ... ... ... ...
  endcentreline
endsurvey|

To create a unique namespace the |centreline| command is enclosed in
|survey| ... |endsurvey| command.
It's useful when the survey has the same name as the file which contains it.
\[E.g.~|survey entrance| in the file |entrance.th|.] The points will then be
referenced using the |@| character---see the |survey| command description.

For really large caves it's possible to build a hierarchical structure of
directories. In such a case we create one special file called |INDEX.th| which
includes all other |*.th| files from a given directory and contains |equate|
commands to define connections between surveys.

\subsubchapter How to draw maps.

The most important thing is to devise a division of the cave into scraps. Scrap is
the basic building block of the map.
It's almost always a {\it bad\/} idea to try to fit each scrap to corresponding
|*.th| file with a centreline from one survey trip. The reason is that
the connections between scraps should be as simple as possible.
Scraps in general are independent on the centreline hierarchy so try to forget
the survey hierarchy when drawing maps and choose the best scrap joins.

We usually insert maps in the last-but-one level in survey hierarchy.\[Remember
that surveys create namespaces, so you may reference only the objects in the given
survey and all subsurveys.] Each
scrap may than contain arbitrary parts of any survey in the last level of
the hierarchy. For example, there is a survey |main| which contains surveys |a|,
|b|, |c| and |d|. Surveys |a| -- |d| contain centreline data from four survey
trips and each of them is in a separate file. There is a map |main_map| which
contains scraps |s1| and |s2|. If the |main_map| is located in the |main|
survey, scrap |s1| may cover part of the centreline from survey |a|, complete
survey |b| and part of |c|; |s2| will cover part of the |a| and |c| surveys
and a complete |d| survey. The survey stations names will be referenced using
the |@| symbol (e.g.~|1@a|) in the scraps.\[If you include maps in the top-level
survey, you may reference any survey station in any scrap, which is very
flexible. On the other hand you then have to use longer names in station
references, like |3@dno.katakomby.jmn.dumbier|.]

Scraps are usually stored in |*.th2| files. Each file may contain more scraps.
To keep the data well organized, there are some naming conventions: in the file
|foo.th2| all scraps are named |foo_si|, where |i| is |1|, |2| an so on.
Cross-sections are named |foo_ci|, lines |foo_li| etc. This helps a lot with
large cave systems: if some scrap is referenced, you immediately know in which
file it has been defined.

Similar to |*.th| files, there may be one file |INDEX.th2| per directory which
includes all the |*.th2| files, and defines scrap joins and maps.

When drawing scraps you should check if the outline is properly defined: all
lines creating the outer border should have |-outline out| option; all lines
surrounding inner pillars \hbox{|-outline in|} option. Scrap outlines can't intersect
themselves---otherwise the inner side of the scrap can't be determined. There
are two simple tests that the scrap outline is correct:
\list
* there is no \MP\ warning ``|scrap outline intersects itself|''
* when you set a passage fill to any color (|color map-fg <number>| option in
|layout|), you may see what Therion considers to be inside the scrap.
\endlist

\subsubchapter How to create models.

The model is created from scrap outlines. The height and depth of the passage
are computed from |passage-height| and |dimensions| point map symbols.


\subchapter Therion in depth.

\subsubchapter How the map is put together.

This chapter explains how |-clip|, |-place|, |-visibility| and |-context|
options of |point|, |line| and |area| commands work exactly. It also gives an
explanation of |color|, |transparency|, |symbol-hide| and |symbol-show| options
of the |layout| command.

While exporting the map, Therion has to determine three attributes for each
point, line or area symbol: visibility, clipping and ordering.

(1) Symbol is visible if all of the following is true:

\list
* it has the |-visibility| option set |on| (all symbols by default),
* it hasn't been hidden by the |-symbol-hide| option in the layout,
* if its |-context| option is set, the corresponding symbol hasn't been
  hidden by the |-symbol-hide| option in the layout.
\endlist

Only the visible symbols are exported.

(2) Some symbols are clipped by the scrap outline. These are by default all of the
following:
\list
* {\it point symbols:} symbolic passage fills (bedrock\dots guano),
* {\it line symbols:} all line symbols which don't have the |-outline| option set
  with the exception of |section|, |arrow|, |label|, |gradient| and
  |water-flow|
* {\it area symbols:} all.
\endlist

The default setting may be changed using the |-clip option|, if this is allowed
for a particular symbol. All other symbols are not clipped by the scrap boundary.

(3) Ordering: Each symbol belongs to one of the following groups which are
drawn consecutively:

\list
* bottom = all symbols with the |-place bottom| option set
* default-bottom = all |area| symbols by default
* default = symbols which don't belong to any other group
* default-top = |ceiling-step| and |chimney| by default
* top = all symbols with |-place top| option set
\endlist

Ordering of symbols inside each group follows the order of commands in the
input file\[Or {\it File commands} menu in XTherion.]: symbols which come first
are drawn last (i.e.~they are displayed at the top of each group).

Now we are ready to describe how the map (or atlas chapter) is constructed:

\list\obeyspaces\obeylines
*map area is filled with |color map-bg|
*surface bitmaps are displayed if |surface| is set |bottom|
*FOR each scrap: outline is filled white
*grid is displayed if |grid| is set |bottom|
*preview below\[As specified using the |preview| option in the |map| command.]%
 is filled with |color preview-below|
*FOR each level\[Level is a collection of scraps not separated by a |break|%
 in the |map| command.]:
  BEGIN of transparency
    FOR each scrap: outline is filled with |color map-fg|
    FOR each scrap: |area| symbols are filled and clipped to scrap boundary
  END of transparency
  BEGIN of clipping by text labels (for all labels in this and upper levels)
    FOR each scrap:
      draw all symbols to be clipped (with the exception of |line survey|)
        ordered from bottom to top
      draw |line survey| symbols
      clip to scrap boundary
    FOR each scrap:
      draw all symbols not to be clipped (with the exception of |point station|
        and all labels) ordered from bottom to top
      draw |point station| symbols
  END of clipping by text labels
  FOR each scrap: draw all (point and line) labels (including |wall-altitude|)
*preview above is drawn with |color preview-above|
*surface bitmaps are displayed if |surface| is set |top|
*grid is displayed if |grid| is set |top|
\endlist




\endinput
