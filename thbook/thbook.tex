% The Therion Book
% Copyright (C) 2003, 2004 Martin Budaj, Stacho Mudrak
% Thanks to Wookey for corrections

\ifx\outputsize\undefined
  \def\outputsize{0}  % can't be defined using \newcount inside \ifx
  % 0 - A4 portrait
  % most images are omitted in smaller sizes:
  % 1 - small screen portrait (some lines and images don't fit)
  % 2 - small screen landscape
  % 3 - ebook reader
\fi

\ifx\directlua\undefined\else
  \let\pdfoutput\outputmode
\fi

\newdimen\tmpdimen
\ifx\pdfoutput\undefined
  \input epsf
  \gdef\outline#1{}
  \def\pdfstartlink#1 #2 #3 #4 #5 #6 #7\pdfendlink{#7}
  \let\red\relax \let\blue\relax \let\grey\relax 
  \let\black\relax \let\green\relax \let\POPcolor\relax
  \let\pdfpagewidth\tmpdimen
  \let\pdfpageheight\tmpdimen
\else
  \ifx\directlua\undefined
    \pdfsuppressptexinfo=-1\pdftrailerid{}% reproducibility settings, note that SOURCE_DATE_EPOCH should also be set
  \else
    \pdfvariable suppressoptionalinfo\numexpr1+2+4+8+512% reproducibility settings
    \let\pdflastximage     \lastsavedimageresourceindex
    \let\pdfpageheight     \pageheight
    \let\pdfpagewidth      \pagewidth
    \let\pdfrefximage      \useimageresource
    \let\pdfximage         \saveimageresource
    \def\pdfcolorstackinit        {\pdffeedback colorstackinit}
    \def\pdftexversion    {\numexpr\pdffeedback version\relax}
    \edef\pdfgentounicode             {\pdfvariable gentounicode}
    \edef\pdfminorversion             {\pdfvariable minorversion}
    \protected\def\pdfannot               {\pdfextension annot }
    \protected\def\pdfcatalog             {\pdfextension catalog }
    \protected\def\pdfcolorstack          {\pdfextension colorstack}
    \protected\def\pdfendlink             {\pdfextension endlink\relax}
    \protected\def\pdfglyphtounicode      {\pdfextension glyphtounicode }
    \protected\def\pdfinfo                {\pdfextension info }
    \protected\def\pdfoutline             {\pdfextension outline }
    \protected\def\pdfstartlink           {\pdfextension startlink }
  \fi
  \pdfminorversion 5
  \openin0=glyphtounicode.tex
  \ifeof0\message{No glyph to unicode mapping found!}\else\closein0\input glyphtounicode.tex\pdfgentounicode=1\fi
  \input etc/supp-mis
  \input etc/supp-pdf

  \openin0=version.tex
  \read 0 to \thversion
  \closein0

  \pdfinfo{/Title (The Therion Book \thversion)
           /Author (Martin Budaj, Stacho Mudr\string\341k)
	   /Subject (Therion documentation)
	   /Keywords (caves, surveying, cartography, computers)
	   /Creator (TeX)
  }
  \pdfcatalog{/ViewerPreferences << /DisplayDocTitle true >> }
  \gdef\outline#1{\pdfoutline goto page \the\pageno {/XYZ} {#1}}

  \ifnum\pdftexversion<140
    \let\red\relax \let\blue\relax \let\grey\relax 
    \let\black\relax \let\green\relax \let\POPcolor\relax
  \else
    \chardef\Color=\pdfcolorstackinit page direct{0 g}
    \def\red{\pdfcolorstack\Color push{0 0.89 0.94 0.28 k}}
    \def\blue{\pdfcolorstack\Color push{1 1 0 0 k}}
    \def\grey{\pdfcolorstack\Color push{0 0 0 0.50 k}}
    \def\green{\pdfcolorstack\Color push{0.91 0 0.88 0.12 k}}
    \def\black{\pdfcolorstack\Color push{0 0 0 1 k}}
    \def\POPcolor{\pdfcolorstack\Color pop}
  \fi

\fi

\newskip\footrightskip \footrightskip=0pt plus0pt minus0pt
\newif\iffulloutput \fulloutputtrue
\ifnum\outputsize=1
  \hsize=84mm
  \vsize=154mm
  \pdfpagewidth=90mm
  \pdfpageheight=160mm
  \hoffset=-22.4mm
  \voffset=-22.4mm
  \rightskip=0pt plus 60 mm
  \footrightskip=0pt plus 60 mm
  \fulloutputfalse
  \overfullrule=0pt
  \mag700
\else\ifnum\outputsize=2
  \hsize=154mm
  \vsize=84mm
  \pdfpagewidth=160mm
  \pdfpageheight=90mm
  \hoffset=-22.4mm
  \voffset=-22.4mm
  \rightskip=0pt plus 30 mm
  \footrightskip=0pt plus 30 mm
  \fulloutputfalse
  \overfullrule=0pt
  \mag700
\else\ifnum\outputsize=3
  \hsize=124mm
  \vsize=169mm
  \pdfpagewidth=130mm
  \pdfpageheight=175mm
  \hoffset=-22.4mm
  \voffset=-22.4mm
  \rightskip=0pt plus 60 mm
  \footrightskip=0pt plus 60 mm
  \fulloutputfalse
  \overfullrule=0pt
  \mag700
\else
  \hsize=159.2mm
  \vsize=239.2mm
  \pdfpagewidth=210mm
  \pdfpageheight=297mm
\fi\fi\fi

\catcode`@=11
\input etc/optarg
\input etc/verbatim
%\input etc/path.sty

\font\rm=cmr12
\font\it=cmti12
\font\mit=cmmi10 at 12pt
\font\bf=cmbx12
\font\tt=cmtt12
\font\tenrm=cmr10
\font\sy=cmsy10 at 12pt
\font\eightrm=cmr8
\font\eightit=cmti8
\font\eightmit=cmmi8
\font\tenitbx=cmbxti10

\textfont0=\rm
\textfont1=\mit
\textfont2=\sy


\font\manfnt=logo10 at 12pt

\font\chap=cmssdc10 at 16pt
\font\subchap=cmssdc10 at 12pt

\def\MP{{\manfnt META}\-{\manfnt POST}}    

\footline={\hss\black\tenrm\folio\POPcolor\hss}

\def\cvak{\ifdim\dimen0<6pt \dimen0=9pt \else \dimen0=3pt\fi}
\def\bodky{\leaders\hbox to12pt{\kern\dimen0.\hss}\hfil}

\def\chapter#1.{\outline{#1}
	\skiplines2\centerline{\chap #1}
	\nobreak   \write1{\string\cvak\string\penalty-50
                    \string\Line{\string\pdfstartlink\space
		    attr {/Border [0 0 0]} goto page \folio\space {/Fit}
		    \string\bf\space #1\string\bodky\ \folio\string\pdfendlink}}%
  \skiplines1}
\def\subchapter#1.{%\global\advance\kapnum by1\pictnum=1\relax
  \penalty-100\skiplines2\centerline{\subchap #1}\nobreak
  \write1{\string\cvak\string\Line{\string\quad\string\pdfstartlink\space
	    attr {/Border [0 0 0]} goto page \folio\space {/Fit}
            #1\string\bodky\ \folio\string\pdfendlink}}\nobreak
  %\skiplines1\nobreak
}
\def\subsubchapter#1.{%\global\advance\kapnum by1\pictnum=1\relax
  \penalty-100\skiplines2\leftline{\subchap #1}\nobreak
  \write1{\string\cvak\string\Line{\string\qquad\string\pdfstartlink\space 
	    attr {/Border [0 0 0]} goto page \folio\space {/Fit}
            #1\string\bodky\ \folio\string\pdfendlink}}\nobreak
}
\def\skiplines#1{\par\hbox{}\nobreak\vskip-\baselineskip
                 \vskip#1\baselineskip}

\let\,\thinspace
\def\Nobreak{\let\brnobr\nobreak\nobreak}
\def\eq{=}

{\catcode`\*=13\catcode`\=13
 \gdef\itemize{%
   \par\nobreak\advance\leftskip1em%\parskip0pt
   \let\brnobr\nobreak
   \catcode`\*13\def*{\par\brnobr\leavevmode\let\brnobr\relax
                      \llap{$\bullet$ }\ignorespaces}
   \catcode`\=13\def={\ifmmode\eq\else\unskip\
                      $\triangleright$ \ignorespaces\fi}
   }
}

\catcode`\|=13
\def|{\ifvmode\penalty-100\medskip\fi\blue\verbatim|}
\def\setupverbatim % see D.E.K., p. 381
    {\parskip=0pt %
     \tt %
     \spaceskip=0pt \xspaceskip=0pt % just in case...
     \catcode`\^^I=\active %
%     \catcode`\<=\active \catcode`\>=\active %
%    \catcode`\,=\active \catcode`\'=\active %
     \catcode`\`=\active %
\catcode`\*=12
\catcode`\"=12
\catcode`\'=12
     \def\par{\leavevmode\endgraf}% this causes that empty lines aren't 
                                  % skipped
     \obeylines \uncatcodespecials \obeyspaces %
     }%
\def\doverbatim#1{\def\next##1#1{##1\endgroup\POPcolor}\next}%


\def\MPpic#1{%\par\penalty-200
%  \midinsert
  \ifx\pdfoutput\undefined
    %\centerline{
    \epsfbox{mp/#1}%}
  \else
    %\centerline{
    \convertMPtoPDF{mp/#1}{1}{1}%}%
  \fi
%  \endinsert
}

\def\pic{\@getoptionalarg\@pic}
\def\@pic#1{%
  \ifx\pdfoutput\undefined[[PICTURE SKIPPED IN DVI OUTPUT]]\else
  \pdfximage\ifx\@optionalarg\empty
            \else width\@optionalarg \fi{pic/#1}\pdfrefximage\pdflastximage
  \fi
}

\def\fitpic#1{
  \ifx\pdfoutput\undefined[[PICTURE SKIPPED IN DVI OUTPUT]]\else
  \medskip
  \setbox0=\hbox{\pdfximage{#1}\pdfrefximage\pdflastximage}
  \centerline{\pdfximage
    \ifdim\wd0>\ht0 width \else height \fi 10cm{#1}\pdfrefximage\pdflastximage}
  \medskip
  \fi
}

\def\mapsymbol#1{%
  \ifx\pdfoutput\undefined
    \hbox{\epsfbox{sym/symlib.#1}}%
  \else
    \convertMPtoPDF{sym/symlib.#1}{1}{1}%
  \fi
}

%\def\future#1{%
%[This chapter will be finished hopefully in the near future. Please read a 
%chapter {\it Syntax summary (#1)} in the {\it Appendix}.]}

\def\list{\penalty50\bgroup\itemize}
\def\endlist{\par\egroup\penalty-50}

\def\syntax{\par{\it Syntax:} }
\def\endsyntax{}

\def\context{\par{\it Context:} }
\def\endcontext{}

\def\description{\par{\it Description:} }
\def\enddescription{}

\def\arguments{\penalty-40\par{\it Arguments:}\bgroup\nobreak\itemize}
\def\endarguments{\par\egroup}

\def\options{\penalty-40\par{\it Options:}\bgroup\nobreak\itemize}
\def\endoptions{\par\egroup}

\def\comopt{\penalty-40\par{\it Command-like options:}\bgroup\nobreak\itemize}
\def\endcomopt{\par\egroup}

\def\example{\penalty-40\par{\it Example:} }
\def\endexample{}

\def\notes{\par{\it Special notes:} }
\def\endnotes{}

\def\www{\@getoptionalarg\@www}
\def\@www#1{\ifx\pdfoutput\undefined
    \ifx\@optionalarg\empty{\tt#1}\else\@optionalarg\fi
  \else
    \pdfstartlink attr {/Border [0 0 0]} user{%
        /Subtype /Link
        /A << 
            /Type /Action 
            /S /URI 
            /URI (#1) 
        >>}%
    \green\ifx\@optionalarg\empty{\tt#1}\else\@optionalarg\fi\POPcolor
    \pdfendlink
  \fi}

% \versiontonum#1 converts a version number like 5.3 or 5.3.1 to a numeric value for comparison
\ifx\pdfmatch\undefined
  \ifx\directlua\undefined
    \def\versiontonum#1{0}   % just return a constant if \pdfmatch or \directlua are unavailable
  \else
    \begingroup
    \catcode`\%=12
    \gdef\versiontonum#1{
      \directlua{
        local s = 0
        local k = 1000000
        for num in string.gmatch('#1', '%d+') do
          s = s + k*num
          k = k / 1000
        end
        tex.sprint(string.format('%d', s))
      }
    }
    \endgroup
  \fi
\else
  \def\twotonum#1.#2 {\numexpr(#1*1000000+#2*1000)}
  \def\threetonum#1.#2.#3 {\numexpr(#1*1000000+#2*1000+#3)}
  \def\versiontonum#1{\ifnum\pdfmatch{[0-9]+[.][0-9]+[.][0-9]+}{#1}=1
         \threetonum#1 \else \twotonum#1 \fi}
\fi

\def\NEW#1{\iffulloutput\ifnum\versiontonum{#1}<\versiontonum{5.5}% don't display too old news
       \else
       \leavevmode\vadjust{%
          \vbox to 0pt{%
	    \vss
	    \line{%
	      \ifodd\pageno\hfil\fi
	      \raise2.3pt\ifodd\pageno\expandafter\rlap\else\expandafter\llap\fi
	      {\ifodd\pageno\quad\fi\grey\tenit #1\POPcolor\ifodd\pageno\else\quad\fi}%
	      \ifodd\pageno\else\hfil\fi
	    }%
	  }%
	}\fi\fi\ignorespaces
      }%


    
\newcount\footnum \footnum 0    
\long\def\[#1]{\unskip\global\advance\footnum by 1\footnote{$^{\the\footnum}$}{#1}}

\def\tenpoint{\def\rm{\fam0\tenrm}%
  \textfont0=\tenrm \scriptfont0=\sevenrm \scriptscriptfont0=\sevenrm
  \textfont1=\tenit %\scriptfont1=\sevenit \scriptscriptfont1=\sevenit
  \def\it{\fam\itfam\tenit}%
  \textfont\itfam=\tenit
  \def\bf{\fam\bffam\tenbf}%
  \textfont\bffam=\tenbf \scriptfont\bffam=\sevenbf
  \scriptscriptfont\bffam=\sevenbf
  \def\tt{\fam\ttfam\tentt}%
  \normalbaselineskip=12pt
  \setbox\strutbox=\hbox{\vrule height9.5pt depth4.5pt width0pt}%
  \normalbaselines\rm}

% new tenpoint def
\def\tenpoint{\def\rm{\fam0\tenrm}% switch to 10-point type
  \textfont0=\tenrm \scriptfont0=\sevenrm \scriptscriptfont0=\fiverm
  \textfont1=\teni \scriptfont1=\seveni \scriptscriptfont1=\fivei
  \textfont2=\tensy \scriptfont2=\sevensy \scriptscriptfont2=\fivesy
  \textfont3=\tenex \scriptfont3=\tenex   \scriptscriptfont3=\tenex
  \textfont\itfam=\tenit \def\it{\fam\itfam\tenit}%
  \textfont\slfam=\tensl \def\sl{\fam\slfam\tensl}%
  \textfont\ttfam=\tentt \def\tt{\fam\ttfam\tentt}%
  \textfont\bffam=\tenbf \scriptfont\bffam=\sevenbf
   \scriptscriptfont\bffam=\fivebf \def\bf{\fam\bffam\tenbf}%
  \normalbaselineskip=12pt
  \font\manfnt=logo10
  \setbox\strutbox=\hbox{\vrule height9.5pt depth4.5pt width0pt}%
  \normalbaselines\rm}

\catcode`@=11
\def\vfootnote#1{\insert\footins\bgroup\tenpoint
  \interlinepenalty\interfootnotelinepenalty
  \splittopskip\ht\strutbox % top baseline for broken footnotes
  \splitmaxdepth\dp\strutbox \floatingpenalty\@MM
  \leftskip=18pt \rightskip\footrightskip \spaceskip\z@skip \xspaceskip\z@skip
  \indent\black\llap{#1\thinspace}\ignorespaces
  \footstrut\futurelet\next\fo@t}
\count\footins=1000 % footnote magnification factor (1 to 1)
\def\@foot{\POPcolor\strut\egroup}

\catcode`@=12 % at signs are no longer letters

\def\twodigits#1{\ifnum#1<10 0\fi\number#1}

\parskip=\medskipamount
\parindent=0pt
\baselineskip=16pt
\raggedbottom
\widowpenalty10000
\clubpenalty10000

\language0
\rm

\input ch00
\input ch01
\input ch02
\input ch03
\input ch04
\input ch05
\input ch06
\input ch07

\end
