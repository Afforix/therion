\eject
\iffulloutput
\vbox{\hsize=93.2mm\baselineskip=12pt\leftskip3.5mm\parskip0pt
\eightit
\leavevmode\llap{1. }When a distinguished but elderly scientist states that something is possible, he is almost certainly right. When he states that something is impossible, he is very probably wrong.

\leavevmode\llap{2. }The only way of discovering the limits of the possible is to venture a little way past them into the impossible.

\leavevmode\llap{3. }Any sufficiently advanced technology is indistinguishable from magic.
\vskip 6pt
\rightline{\eightrm ---Arthur C. Clarke, {\eightmit 1973}}
}
\fi

\chapter Appendix.

\subchapter Compilation.

\subsubchapter Installing the dependencies.

If you want to compile Therion from source code and run it, you need
(first three are required only during compilation):
\list
* GNU C/C++ compiler
* GNU make
* Perl
* Python 3
* PROJ library (\www{https://proj.org/}). Supported versions are:
  v4: 4.9.3; v5: 5.1.0 and newer; v6: 6.2.1 and newer; v7: 7.0.0 and newer.
* Tcl/Tk 8.4.3 and newer (\www{https://www.tcl.tk}) with
  {\it BWidget} widget set  \hfil\break
  (\www{https://sourceforge.net/projects/tcllib/})
  and optionally {\it tkImg} extension  \hfil\break
  (\www{https://sourceforge.net/projects/tkimg/}).
%  {\it Tom} OpenGL extension (improved version is included in Therion source
%  distribution)
%  and
* \TeX\ distribution with at least \TeX\ with Plain format,
  recent pdf\TeX, and \MP\ (\www{https://www.tug.org}).
* LCDF Typetools package (\www{https://www.lcdf.org/type/})
* ImageMagick distribution with {\it convert} and {\it identify} utilities,
  if you want to use warping of survey sketches.
* {\it ghostscript} if you want to create calibrated images from georeferenced
  PDF maps.
\endlist

%Installing Tom:
%
%\list
%* if you use Windows, download a Tcl/Tk source distribution, |make| and
%  |make install| it under MSYS
%* |cd therion/thtom/linux| or |cd therion/thtom/win|
%* |make|
%* copy |Tom0.2| directory (which should contain |pkgIndex.tcl| and one of
%  |libtom.so| or |libtom.dll|)
%  to the |lib| subdirectory of your
%  Tcl/Tk distribution.
%\endlist

To compile Loch, you need

\list
* freetype 2 and newer; |pkg-config freetype2| must work
* wxWidgets 2.6 and newer; |wx-config| must work
* VTK 5.0 and newer
* libjpeg, libpng, zlib
\endlist

All programs (with the exception of BWidget and tkImg package) are usually
included in Linux, Unix or MacOS\,X distributions.
For Windows consider using MinGW and MSYS2 (\www{https://www.msys2.org/}).
It's a distribution of GNU utilities with GNU make and GCC.
(BTW, why not to use precompiled Windows version?)

Here is an example for installing dependencies to compile Therion
on Ubuntu 18.04 or 20.04:
  |sudo apt install|
  |bwidget|
  |cmake|
  |gcc|
  |ghostscript|
  |imagemagick|
  |lcdf-typetools|
  |libfreetype6-dev|
  |libjpeg-dev|
  |libpng-dev|
  |libproj-dev|
  |libtk-img-dev|
  |libvtk7-dev|
  |libwxgtk3.0-gtk3-dev|
  |tcl-dev|
  |texlive-binaries|
  |texlive-metapost|
  |zlib1g-dev|.

\subsubchapter Using CMake.

\NEW{6.0}Unpack the source distribution |therion-5.*.tar.gz| and create a separate
directory for the build, e.g.:
\list
* |cd therion && mkdir build && cd build|
* |cmake [parameters] ..|
* |make -j4|
\endlist

{\it CMake parameters}

Here is a selection of parameters which can be used with cmake:
\list
* |-G <generator>| = specify the generator, e.g.
  |-G Ninja| to use {\it Ninja} build system instead of make,
  or |-G "MSYS Makefiles"| to build using {\it make} under MSYS2
* {\catcode`\=12|-DUSE_BUNDLED_SHAPELIB=OFF|} = use the system Shapelib library
* {\catcode`\=12|-DUSE_BUNDLED_CATCH2=OFF|} = use the system Catch2 library
* {\catcode`\=12|-DUSE_BUNDLED_FMT=OFF|} = use the system fmt library
* {\catcode`\=12|-DECM_ENABLE_SANITIZERS=<option>|} = use runtime sanitizers, relevant options:

  |address| -- detects invalid memory accesses, use-after-free, double free, memory leaks, useful for debugging

  |undefined| -- detects undefined behavior, for example use of uninitialized values

* {\catcode`\=12|-DCMAKE_BUILD_TYPE=<option>|} = build type, relevant options:

  |Debug| -- compile with debug symbols and asserts

  |Release| -- compile with optimizations

  |RelWithDebInfo| -- compile with debug symbols and optimizations, useful for profiling and debugging

* {\catcode`\=12|-DTHBOOK_FORMAT=<option>|} = set the output size of the Therion book (most of the images are
  omitted in smaller sizes); relevant options:

  |0| -- A4 portrait

  |1| -- small screen portrait (some lines and images don't fit)

  |2| -- small screen landscape

  |3| -- ebook reader optimized

\endlist

The following cmake components can be used to selectively install a part of the package: th-runtime, loch-runtime,
th-docs, loch-docs.

\subsubchapter Legacy approach: using make.

\list
* unpack the source distribution |therion-5.*.tar.gz|
* |cd therion|
* |make config-macosx| or |make config-win32|, if you use MacOS~X or Windows,
  respectively
* |make|
* |sudo make install|
\endlist

{\it Make parameters}

Therion's {\it makefile} may take some optional parameters.

\list
* |config-linux|, |config-macosx|, |config-win32| = configure Therion for a
  specific platform. Linux is a default.
* |config-release|, |config-oxygen|, |config-ozone| = set optimization level
  for C++ compiler (none, |-O2| and |-O3|)
* |config-debug| = useful before debugging the program
* |install| = install Therion
* |clean| = delete all temporary files
\endlist

\subsubchapter Hacker's guide.

{\it Cross-compilation for Windows}\NEW{5.4}

Therion supports compilation of Win32/Win64 executables in Linux using MXE cross
compiler (\www{http://mxe.cc}).

\list
* install the following static/win32 packages (i686-w64-mingw32.static-\char42) 
or static/win64 packages (x86-64-w64-mingw32.static-\char42) to
the directory |/usr/lib/mxe/|: binutils, bzip2, expat, freetype-bootstrap,
gcc, gettext, glib, harfbuzz, jpeg, libiconv, libpng, proj, tiff, vtk, wxwidgets,
xz, zlib.
* modify PATH: {\catcode`\=12|export PATH=/usr/lib/mxe/usr/bin:$PATH|}
* use CMake or the legacy approach (|cd therion && make config-win32cross && make|)
  to build Therion
\endlist

See |therion/.github/workflows/| for detailed examples of building
Therion on multiple platforms.

{\it Adding new translations}

Therion supports translation of map labels.
Suppose you want to add a new language |xx|.

\list
* run `|perl process.pl export xx|' in the `thlang' Therion source subdirectory.
  This creates a file |texts_xx.txt|. This file is UTF-8 encoded.
* edit the |texts_xx.txt| file. Add your translations at lines beginning with
  `|xx:|'.
* run |make update|
* compile Therion
\endlist


{\it Adding new encodings}

Although UTF-8 Unicode encoding covers all characters which Therion is able to
process, it may be inconvenient to use it. In that case it's possible to add
support for any 8-bit encoding for text input files. Copy a translation file to
the |thchencdata| directory; add its name to `ifiles' hash in the beginning of
the Perl script |generate.pl|; run it and recompile Therion.

The translation file should contain two hexadecimal values of a character
(first one in the 8-bit encoding, second one in Unicode) in each line. Possible
comments follow the `|#|' character.

{\it Adding new \TeX\ encodings}

It's easy to add new encodings for 2D map output.%
\[\NEW{5.3}This section applies to old-style font selection using |tex-fonts|
command in the initialization file and is obsolete when using |pdf-fonts|
command.]
Copy an appropriate encoding
mapping file with an |*.enc| extension to the |texenc/encodings|, run the Perl
script |mktexenc.pl| located in the |texenc| directory and compile Therion.

Therion uses the same encoding files as |afm2tfm| program from the \TeX\
distribution, which has the same format as an encoding vector in a PostScript
font. You may find more details in the chapter {\it 6.3.1.5 Encoding file
format} in the documentation to Dvips program.


{\it Generating new \TeX\ and \MP\ headers}

Therion uses \TeX\ and \MP\ for 2D map visualization and typesetting.
Predefined macros are compiled into the Therion executable and are copied to
the working directory just before running \MP\ and \TeX\ (unless the
|--use-extern-libs| option is used). Layout command makes it possible to modify
some macros in the configuration file at the run-time.

However, it's possible to make permanent changes to the macro files. After
modifying the files in the |mpost| and |tex| directories it's necessary to run
Perl scripts |genmpost.pl| and |gentex.pl|, which generate C++ header files,
and compile Therion executable again.

{\it Updating the geomagnetic model}

Therion uses the IGRF model to calculate the magnetic declination. Download the
model in a txt format from \www{https://www.ngdc.noaa.gov/IAGA/vmod/igrf.html}
and save it in the |geomag/| directory (e.g. |igrfXYcoeffs.txt|).
Run |./igrf2c.py igrfXYcoeffs.txt| which creates |thgeomagdata.h| in the
Therion source directory and recompile Therion.

To test the model, extract the file |sample_out_IGRFXY.txt|, which is included
in the Geomag distribution available on the same web page. Put it into the
|geomag/test/| directory, run |./build.sh sample_out_IGRFXY.txt| and check the
lines with an exclamation mark in the output.



\subchapter Environment variables.

Therion reads following environment variables:

\list
* |THERION| = [not required] search path for (x)therion.ini file(s)
* |HOME| (|HOMEDRIVE| + |HOMEPATH| on WinXP) =
  [not required, but usually present on your system] search path
  for (x)therion.ini file(s)
* |TEMP|, |TMP| = system temporary directory, where Therion stores temporary
  files (in a directory named |th$PID$|, where |$PID$| is a process ID),
  unless |tmp-path| is specified in the initialization file.
\endlist

Consult the documentation of your OS how to set them.

\subchapter Initialization files.

Therion's and XTherion's system dependent settings are specified in the
file therion.ini or xtherion.ini, respectively.
They are searched for in the following directories:

\list
* on UNIX:
  |.|, |$THERION|, |$HOME/.therion|, |/etc|, |/usr/etc|, |/usr/local/etc|
* on Windows:
  |.|, |$THERION|, |$HOME\.therion|, |<Therion-installation-directory>|,
  |C:\WINDOWS|, |C:\WINNT|, |C:\Program Files\Therion|
\endlist

\subsubchapter Therion.

If no file is found Therion uses its default settings. If you want to list
them, use |--print-init-file| option. The initialization file is read
like any other therion file. (Empty lines or lines starting with `|#|' are
ignored; lines ending with a backslash continue on next line.) Currently
supported initialization commands follow.

\list
* |loop-closure <therion/survex>|

  By default, survex is used if present, otherwise therion.

* |encoding-default <encoding-name>|

  Set the default output encoding (currently unused).

* |encoding-sql <encoding-name>|

  Set the default output encoding for SQL export.

* |language <xx[_YY]>|

  Default output language. See the copyright page for
  the list of available languages.

* |units <metric/imperial>|

  Set default units.

* |mpost-path <file-path>|

  Set the full path to a \MP\ executable if Therion can't find it
  (``|mpost|'' is the default).

* |mpost-options <string>|

  Set \MP\ options.

* |pdftex-path <file-path>|

  Set the full path to a pdf\TeX\ executable if Therion can't find it
  (``|pdftex|'' is the default).

* |identify-path <file-path>|

  Set the full path to ImageMagick's identify executable if Therion
  can't find it (``|identify|'' is the default).

* |convert-path <file-path>|

  Set the full path to ImageMagick's convert executable if Therion
  can't find it (``|convert|'' is the default).

* |source-path <directory>|

  Path to data and configuration files. Used mostly for system-wide grades and
  layout definitions.

* |tmp-path <directory>|

  Path where temporary directory should be created.

* |tmp-remove <OS command>|

  System command to delete files from the temporary directory.

* |tex-env <on/off>|

  [Works on Windows only.]
  When set to |off| (default), Therion temporarily clears all environment
  variables related to \TeX. Useful if there is other \TeX distribution
  installed on your system which had set-up any environment variables,
  which could confuse \TeX\ and \MP\ programs supplied in Therion for Windows
  distribution.

  Set to |on| if you use other \TeX\ distribution for maps processing.

* |text <language ID> <therion text> <my text>|

  Using this option you can change any default therion text translation in output.
  For list of therion texts and available translations, see |thlang/texts.txt| file.

* |cs-def <id> <proj4def>|\NEW{5.4}  %[other options]

  Define a new coordinate system |<id>| using Proj4 syntax.

* |cs-trans <cs1> <cs2> <proj-pipeline>|\NEW{6.0.7}

  Define a transforamation pipeline between two coordinate systems.
  \[See \www{https://proj.org/usage/transformation.html} for details of pipelines
  definition.] Both |cs1| and |cs2| can be lists of aliases
  enclosed in brackets.

  Therion contains a built-in database of transformation pipelines.\[Mostly for
  JTSK to ETRS89 transformations. The definitions are in the
  |thcsdata.tcl| file in the source code, |proj\_transformations| variable.]
  You can override any built-in definition either by redefining the pipeline
  using |cs-trans| or make it ignored by using an empty string in the |cs-trans| definition.

  If the pipeline references transformations grids which are not installed
  locally, Therion attempts to download them from \www{cdn.proj.org} if
  |proj-missing-grid| option is set to |download| (otherwise it prints an error message
  and stops).

  This option is ignored if Therion is using PROJ v6 or older.

  Example: {\rightskip 0pt plus 3cm\catcode`\=12
  |cs-trans [jtsk epsg:5513] [etrs34 epsg:25834] "+proj=pipeline +step +inv +proj=krovak +axis=wsu +lat_0=49.5 +lon_0=24.8333333333333 +alpha=30.2881397527778 +k=0.9999 +x_0=0 +y_0=0 +ellps=bessel +step +inv +proj=hgridshift +grids=sk_gku_JTSK03_to_JTSK.tif +step +proj=push +v_3 +step +proj=cart +ellps=bessel +step +proj=helmert +x=485.021 +y=169.465 +z=483.839 +rx=-7.786342 +ry=-4.397554 +rz=-4.102655 +s=0 +convention=coordinate_frame +step +inv +proj=cart +ellps=GRS80 +step +proj=pop +v_3 +step +proj=utm +zone=34 +ellps=GRS80"|
  \par}


* |proj-auto <on/off>|\NEW{5.5}

  \penalty50
  If set |on|, let PROJ v6+ decide which transformation between coordinate systems
  is optimal.\[In this case
  the function |proj\_create\_crs\_to\_crs()| is used. Otherwise Therion calls
  |proj\_create()| function in the PROJ library.] The selected transformations
  are listed in the |log| file. It is recommended to specify coordinate systems
  using the EPSG codes directly (e.g.~EPSG:4258).

  If set |off| or if Therion uses PROJ older than v6, the source coordinates are
  first transformed to |wgs-84|, then from |wgs-84| to the target coordinate
  system. This might result in a suboptimal precision.

  The default setting is |on|.\NEW{6.0.7} This option is ignored for those pairs of coordinate
  systems which have a transformation pipeline defined (see |cs-trans|).

* |proj-missing-grid <ignore/warn/fail/cache/download>|\NEW{5.5.1}

  Set missing transformation grids handling if |proj-auto| is |on|
  or a custom transformation pipeline in |cs-trans| references such grids. The grids
  are used to achieve a better transformation precision between some
  coordinate systems.

  For |cs-trans| pipelines, only |download| tries to download the grid; all other options
  are equivalent to |fail|. If |proj-auto| is used to find the best transformation,
  the following applies:

  |ignore| silently lets Proj to choose other transformation which doesn't use the missing
  grid(s); this usually leads to decreased transformation accuracy
  (say metres instead of centimetres). See the |log| file for a list of
  the used transformations and their precisions.

  |warn| behaves like |ignore|, but prints warnings about missing grids (the
  download links are usually displayed as well).\[See
  \www{https://proj.org/resource\_files.html} for the instructions
  where to put the downloaded grids.]

  |fail| stops after the first missing grid is detected and displays
  the download link.

  |cache| enables the network connectivity and
  lets Proj to download the missing parts of grids from the
  Internet. The information is stored in a local
  cache.\[See \www{https://proj.org/usage/network.html}]
  As only parts of grids covering the current area are downloaded,
  it's potentially faster and less space consuming then |download|. The
  downside is that the local cache is not used by Proj if the network
  is disabled (in Therion you have to use the |cache| mode to
  use the local cache).

  |download| temporarily enables the network connectivity and
  lets Proj to download the missing grids from the
  Internet. The grids are used in {\it subsequent} runs (in any mode).

  The default setting is |download|.\NEW{6.0.7} The option |proj-missing-grid| is ignored if
  |proj-auto| is |off| or Proj v5 or older is used.
  If Proj v6 is used by Therion, the settings
  |cache| and |download| are equivalent to |fail|, as the network
  connectivity is a feature of Proj v7 and later.

* |pdf-fonts <rm> <it> <bf> <ss> <si>|\NEW{5.3}

  Set-up fonts to be used in PDF maps.
  The command has to be followed by paths specifying where regular, italic,
  bold, sans-serif and sans-serif oblique fonts are located in your system.
  TrueType and OpenType fonts are supported.

  Therion requires LCDF Typetools to be installed on your system to use this
  command. Example:

  |pdf-fonts  "/usr/share/fonts/Serif.ttf" \
           "/usr/share/fonts/Serif-Italic.ttf" \
           "/usr/share/fonts/Serif-Bold.ttf" \
           "/usr/share/fonts/Sans.ttf" \
           "/usr/share/fonts/Sans-Oblique.ttf"|

* |otf2pfb <on/off>|\NEW{5.3}

  When set to |on| (default), OpenType fonts used in |pdf-fonts| are
  converted to PFB fonts, if they are PostScript-based. Some information
  is lost in the PFB format, but there is advantage that pdf\TeX\ can embed
  subset of PFB fonts (in contrast with OpenType fonts which must be fully
  embedded).

* |tex-fonts <encoding> <rm> <it> <bf> <ss> <si>|

  Original and more complicated way to set-up fonts for PDF maps. You need
  to explicitly specify encoding (maximum 256 characters from the font
  that will be used). The list of currently supported
  encodings gives the |--print-tex-encodings| command line option.
  The same encoding must be used while generating \TeX\ metrics (|*.tfm| files)
  for those fonts (e.g.~with the afm2tfm program) and this encoding must be
  explicitly given also in the pdf\TeX's map file. The only exception is the
  base set of Computer Modern fonts, which use `raw' encoding. This encoding
  doesn't need to be specified in the pdf\TeX's map file.

  Encoding has to be followed by five font specifications for regular, italic,
  bold, sans-serif and sans-serif oblique styles.
  Default setting is |tex-fonts raw cmr10 cmti10 cmbx10 cmss10 cmssi10|

  Example how to use other fonts (e.g.~TrueType Palatino in xl2 (an encoding
  derived from ISO8859-2) encoding). Run:

  |ttf2afm -e xl2.enc -o palatino.afm palatino.ttf|

  |afm2tfm palatino.afm -u -v vpalatino -T xl2.enc|

  |vptovf vpalatino.vpl vpalatino.vf vpalatino.tfm|

  You get files |vpalatino.vf|, |vpalatino.tfm| and |palatino.tfm|. Add the line

  |palatino <xl2.enc <palatino.ttf|

  to the pdf\TeX's map file. The same should be done for the italic and bold
  faces and corresponding sans-serif and sans-serif-oblique fonts. If you're lazy
  try

  |tex-fonts xl2 palatino palatino palatino palatino palatino|

  (We should use actually virtual font |vpalatino| instead of |palatino|,
  which contains no kerning or ligatures, but
  pdf\TeX\ doesn't support |\pdfincludechars| command on virtual fonts.
  To be improved.)

  If you want to add some unsupported encodings,
  read the chapter {\it Compilation / Hacker's guide}.

* |tex-fonts-optional <encoding> <rm> <it> <bf> <ss> <si>|

  Similar to |tex-fonts|, but tests if the \TeX\ fonts are installed in the
  system. It does nothing if any of the specified fonts is not present.

  This setting is used by default for Czech/Slovak and cyrillic fonts
  to avoid \MP\ errors on systems without these fonts present.

  As the test takes some time (pdfTeX instance is run), you might
  disable the default behaviour completely by setting |tex-fonts| in the
  INI file.

* |tex-refs-registers <on/off>|

  Switch between using count registers and macros to store references to
  graphical objects in \TeX. Each approach has some advantages, see
  the section {\it Limitations}.

\endlist


\subsubchapter XTherion.

Initialization file for XTherion is actually a Tcl script evaluated when
XTherion starts. The file is commented; see the comments for details.


%\subsubchapter Speed optimization.
%
%[Optionally creating \MP\ and \TeX\ format files.]

\subchapter Limitations.

\list
*  scrap size = 

   \MP\ in the default (`scaled') mode: $\approx 2.8 \times 2.8$ m in the output scale

   \MP\ in the `double' mode:\[To run \MP\ in the `double' mode, set
   |mpost-options  "-numbersystem\eq double"| in the initialization file. It's not recommended
   to use arbitrary-precision modes `decimal' and `binary', as there are still bugs in their
   implementation and they are much slower than the `double' mode. You need to use \MP\ newer
   than 2.00 to use this mode without issues.]
   practically no limit\[It's high enough to be reached.]
*  page size =

   PDF map or atlas: $\approx 5 \times 5$ m (pdf\TeX\ limit)

   SVG map: limits depend on the viewing application
*  scraps count =

   \MP\ in the scaled mode: $4 (scraps + sections) < 4000$

   \MP\ in the `double' mode: practically no limit

   \TeX\ limit in registers mode:\[This is the default approach, in which the count registers
   are used to store referenced to the graphic objects.]
            $2\times pages + images + patterns + 6 (scraps + sections) < 32\,500$ when using pdf\TeX\
            (or approximately $65\,000$ when using Lua\TeX\[To use Lua\TeX,
            set |pdftex-path  "luatex"| in the initialization file.])

   \TeX\ limit in macro mode:\[Instead of using count registers, each reference is stored in
   a separate macro. This mode is activated by setting |tex-refs-registers off| in
   the initialization file.] limited only by memory available to \TeX\[Note, that you need to
   modify |texmf.cnf| configuration file in your \TeX\ distribution to change the limits.]
\endlist


\subchapter Example data.

Following simple example illustrates basic usage of Therion commands:

|encoding  utf-8

survey main -title "Test cave"

  survey first
    centreline
      units compass grad
      data normal from to compass clino length
                  1    2  100     -5    10
    endcentreline
  endsurvey

  survey second -declination [3 deg]
    centreline
      calibrate length 0 0.96
      data normal from to compass length clino
                  1    2  0       10     +10
    endcentreline
  endsurvey

  centreline
    equate 2@first 1@second
  endcentreline

  # scraps are usually in separate *.th2 files
  scrap s1 -author 2004 "Therion team"

    point 763 746 station -name 2@second
    point 702 430 station -name 2@first
    point 352 469 station -name 1@first
    point 675 585 air-draught -orientation 240 -scale large

    line wall -close on
      287 475
      281 354 687 331 755 367
      981 486 846 879 683 739
      476 561 293 611 287 475
    endline

  endscrap

  map m1 -title "Test map"
    s1
  endmap

endsurvey|

Corresponding configuration file could be:

|encoding  utf-8
source test

layout l1
  scale 1 100
  layers off
endlayout

select m1@main

export model -fmt survex
export map -layout l1|

If you save data file as `test.th' and configuration file as `thconfig' you may
process them with Therion.


%\subchapter Map symbols library.

%\input sym/symlib

\subchapter History.

\list
\everypar{\hangindent16pt\hangafter1}
* {\bf 1999}

  Oct: first concrete ideas

  Nov: start of programming (Perl scripts and \MP\ macros)

  Dec 27: Therion compiles simple map in PostScript format
       for the first time (32 kB of Perl and 7 kB of \MP\ and \TeX\ source code).
       The map warping model was substantially different
       from the current one (positions of features were relative to
       a particular survey shot, not to positions of all stations in a scrap).
       This version already included some interesting features
       such as {\it transformation functions} which allowed user specification
       of the input format for survey data, or splitting large maps to
       multiple sheets.

  Dec 30: the first web page (with data examples but without source
       code)

* {\bf 2000}\nobreak\par\nobreak
  Jan: xthedit (Tcl/Tk), a graphical front-end for Therion

  Feb 18: start of reprogramming (Perl)

  Apr 1: the first hyperlinked PDF cave map / atlas

  Aug: experiments with PDF, pdf\TeX\ and \MP

* {\bf 2001}

  Nov: start of reimplementation from scratch:
       Therion (C++ with some Perl scripts inherited from the previous version);
       notion of a scrap;
       interactive 2D map editor ThEdit as a replacement of xthedit (Delphi)

  Dec: ThEdit exports simple map for the first time

* {\bf 2002}

  Mar: Therion 0.1 ---
       Therion is able to process survey data (centreline) of the Cave of Dead Bats.
       XTherion, text editor designed for Therion (Tcl/Tk).

  Jul 27: Therion 0.2 ---
       Therion compiles simple map (consisting of two scraps)
       for the first time (800 kB of source code)

  Aug: XTherion extended to 2D map editor (as a replacement of ThEdit)

  Sep: Therion compiles first real and complex map of a cave. XTherion
       extended to compiler.

%  Oct 5: Public presentation of Therion, held on Trango\v{s}ka

* {\bf 2003}\nobreak

  Mar: the first version of The Therion Book finished

  Apr: Therion included in Debian GNU/Linux

  Jun: all Perl scripts rewritten in C++, Therion is one executable program
       now (although using Survex and \TeX)

%  Nov 10: first idea of the Labyrinth project

* {\bf 2004}\Nobreak

  Mar: Therion 0.3 --- Therion exports 3D model created from 2D maps.
  Loop closure algorithm included into Therion.

* {\bf 2006}\Nobreak

  Oct: Therion 0.4 --- New 3D viewer (Loch).

* {\bf 2007}\Nobreak

  Feb: Therion 0.5 --- Support for bitmap sketches morphing.

* {\bf 2016}\Nobreak

  Dec: GitHub repository.
\endlist


\subchapter Future.

Although Therion is already used for map production, there are a lot of
new features to be implemented:

\subsubchapter General.

\list
* loop closure information in SQL
\endlist

\subsubchapter 2D maps.

\list
* complete the predefined symbol sets
* generate registers for atlas
* use MPlib instead of \MP
\endlist


\subsubchapter 3D models.

\list
* improve passage walls modeling
\endlist

\subsubchapter XTherion.

\list
* improve 2D editing capabilities
\endlist

\subsubchapter Loch.

\list
* colour schemes
* survey tree for selecting sub-surveys to display
* spatial filtering (e.g.~clipping by planes)
* support for multiple surfaces
\endlist

\subsubchapter Labyrinth.

\list
* completely new GUI in the far future (see \www{https://labyrinth.speleo.sk})
\endlist



\endinput
