\chapter Appendix.

\subchapter Installation notes.

Therion runs under Unix, Linux, MacOS X and Win32 operating systems.

\subsubchapter Common requirements.
\list
* If you want to compile the therion source code, you need a C/C++ compiler 
   (for Windows MinGW (|http://www.mingw.org|) is highly recommended---a 
   distribution of GNU utilities with GNU make, gcc...)
   
* \TeX, pdf\TeX\ 1.0 and \MP. Plain should be the default format for \TeX; 
  Therion doesn't use La\TeX\ or other available formats. If you 
  have any doubts, install any full distribution of \TeX. (TeXlive for any 
  platform, teTeX for Unix, fpTeX or MikTeX for Windows.) If you're new
  to \TeX, visit the homepage of {\it \TeX\ Users Group} at 
  \path|http://www.tug.org|.

* Perl 5.5 (probably already installed on Unix; for Windows try ActivePerl
   from \path|http://www.activestate.com|).

* Tcl/Tk 8.4 (|http://www.tcl.tk|) with BWidget widget set 
  (\path|http://sourceforge.net/projects/tcllib|). Tcl/Tk is only required for 
  xtherion. For Windows try ActiveTcl from |http://www.activestate.com|.

* Survex 1.0 (|http://www.survex.com|)
%
%* Ghostscript 7.0 or Acrobat Reader 5.0 or other viewer of PDF files.
\endlist

Get all this working before installing therion. We don't provide any support 
concerning this. All these programs are well documented and accessible on the 
net.


\subsubchapter Compilation.

If you want to compile therion code yourself, follow these 
steps:

\list
* unpack source distribution |therion-0.2.*.tar.gz|
* |cd therion/src|
* |make config-macosx| or |make config-win32|, if you use MacOS~X or Windows, 
  respectively
* |make|
* |make xtherion|
* |make thpdf|
\endlist

This compilation should work on Unix/Linux systems---if it doesn't then read 
the next section for clues. On Win32, |make xtherion|
and |make thpdf| will not work. You have to download them compiled.

\subsubchapter Environment variables.

To be able to run therion you will probably need to set up following 
environment variables:

\list
* set environment variable `|THERION|' to `|therion/lib|' directory

  This environment variable is used to specify the
  Therion search path. If not defined (also not specified by |-p| option),
  \path|$HOME/.therion:/usr/share/therion:/usr/local/share/therion| is used.
  Notice, that on Windows you must use `|;|' as a path separator. The
  default directory on Windows is |C:/therion|. This default setting may not 
  reflect the real location of Therion.

* add `|therion/bin|' directory to `|PATH|' environment variable

* add `|therion/lib/tex|' and `|therion/lib/mpost|' to \TeX\ and \MP\ searching
   paths 
   
   See the documentation included in your \TeX\ distribution 
   (for web2c based \TeX\ distribution you may set these paths 
   in the file `|texmf/web2c/texmf.cnf|'
   or through environment variables |TEXINPUTS| and |MPINPUTS|).
\endlist

If you're not very experienced in environment variables etc., try the 
appropriate example below:

On UNIX add the following lines to |.bashrc| (or |.bash_profile|) file in your 
home directory:

|export PATH=$PATH:therion/bin
export THERION=therion/lib
export MPINPUTS=.:$THERION/mpost:
export TEXINPUTS=.:$THERION/tex:|

On WINDOWS, add the following lines to your |autoexec.bat| file:

|set PATH=%PATH%;C:\therion\bin
set THERION=C:\therion\lib
set MPINPUTS=.;%THERION%\mpost;
set TEXINPUTS=.;%THERION%\tex;|

Note, that on Win 2000 or XP, you have to set environment variables through 
the dialog {\it Control Panel} $\to$ {\it System} $\to$ {\it Advanced} $\to$ 
{\it Environment Variables}.

After setting up environment variables you should restart the computer.

The temporary directory used is named |th$PID$| where |$PID$| is a 
process pid. It is placed in the system temporary directory, which is
read from the |TEMP| or |TMP| environment variable. 

\subsubchapter Initialization file.

The correct paths are set in the initialization file.
This file is used to specify Therion's system dependent settings. 
It's named therion.ini and must be placed in the Therion 
search path (default location is `|therion/lib/therion.ini|'). If no
file is found therion uses its default settings. If you want to list
them, use |--print-init-file| option. The initialization file is read 
like any other therion file. (Empty lines or lines starting with `|#|' are 
ignored; lines ending with a backslash continue on next line.) Currently 
supported initialization commands follow.

\list
* |encoding_default <encoding-name>|

        Set the default output encoding. For \TeX\ ISO8859-2 is currently
        hardcoded.
        
* |path_cavern <file-path>|

        Set the full path to a Survex ``|cavern|'' executable. (``|cavern|'' is the default).
        
* |path_3dtopos <file-path>|

        Set the full path to a Survex ``|3dtopos|'' executable. (``|3dtopos|'' is the default).

* |path_mpost <file-path>|

        Set the full path to a \MP\ executable (``|mpost|'' is the default).

* |path_pdftex <file-path>|

        Set the full path to a pdf\TeX\ executable (``|pdfetex|'' is the default).
\endlist

\subsubchapter Special setup for Windows.

If you want to create maps you need to be able to run |thpdf| from the command
line. Therefore you need to edit |thpdf.bat| in |therion/bin| directory and
set the correct Perl path. The same has to be done with |xtherion|,
if you want to run it from command line. 

\subsubchapter Speed optimization.

[Optionally creating \MP\ and \TeX\ format files.]


\subchapter Map symbols library.

[Gallery of map symbols predefined by Therion.]

%\subchapter Debugging mode.

\subchapter History.

\list
* 1999

  Oct: first concrete ideas

  Nov: start of programming: scripts, macros \&c.

  Dec 27: Therion compiles simple map for the first time
       (32 kB of source code). This first release had some interesting features
       such as {\it transformation functions}, which alowed user-specification
       of the input format for survey data.
				
* 2000

  Jan: xthedit, a graphical front-end for Therion

  Aug: experiments with PDF and pdf\TeX, script for \MP\ $\to$ PDF conversion and
       atlas building, some \MP\ macros

* 2001

  Nov: start of reimplementation from scratch: 
       Therion \& interactive 2D map editor ThEdit
       (as a replacement of xthedit)

  Dec: ThEdit exports simple map for the first time

* 2002

  Mar: Therion 0.1 ---
       Therion is able to process survey data (centerline) of Dead Bats Cave.
       XTherion, text editor designed for Therion.

  Jul 27: Therion 0.2 ---
       Therion compiles simple map (consisting of two scraps)
       for the first time (800 kB of source code)

  Aug: XTherion extended to 2D map editor (as a replacement of ThEdit)

  Sep: Therion compiles first real and complex map of a cave. XTherion
       extended to compiler.

%  Oct 5: Public presentation of Therion, held on Trango\v{s}ka

* 2003

  Mar: First version of The Therion Book finished

  Apr: Therion included in Debian GNU/Linux
\endlist


\subchapter Future.

Although Therion is already used for map production, there are a lot of 
new features to be implemented:

\subsubchapter General.

\list
* comprehensive information on centerline processing and loop closure
\endlist

\subsubchapter 2D maps.

\list
* real font selection for PDF output
* map symbol sets
* debugging mode
* SVG support
\endlist


\subsubchapter 3D models.

\list
* passage walls modelling
* OpenGL 3D viewer as a part of XTherion
\endlist

\endinput
