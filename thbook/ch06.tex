\eject
\vbox{\hsize=93.2mm\baselineskip=12pt\leftskip3.5mm\parskip0pt
\eightit
\leavevmode\llap{1. }When a distinguished but elderly scientist states that something is possible, he is almost certainly right. When he states that something is impossible, he is very probably wrong.

\leavevmode\llap{2. }The only way of discovering the limits of the possible is to venture a little way past them into the impossible.

\leavevmode\llap{3. }Any sufficiently advanced technology is indistinguishable from magic.
\vskip 6pt
\rightline{\eightrm ---Arthur C. Clarke, {\eightmit 1973}}
}

\chapter Appendix.

\subchapter Compilation.

If you want to compile Therion from source code and run it, you need 
(first three are required only during compilation):
\list
* GNU C/C++ compiler
* GNU make
* Perl
* Tcl/Tk 8.4.3 and newer (\www{http://www.tcl.tk}) with 
  {\it BWidget} widget set  \hfil\break
  (\www{http://sourceforge.net/projects/tcllib})
  and optionally {\it tkImg} extension \hfil\break
  (\www{http://sourceforge.net/projects/tkimg}).   
%  {\it Tom} OpenGL extension (improved version is included in Therion source 
%  distribution)
%  and 
* \TeX\ distribution with at least \TeX\ with Plain format, 
  recent pdf\TeX, and \MP\ (\www{http://www.tug.org}). 
* LCDF Typetools package (\www{http://www.lcdf.org/type/})
* ImageMagick distribution with {\it convert} and {\it identify} utilities, 
  if you want to use warping of survey sketches.
* {\it ghostscript} if you want to create calibrated images from georeferenced
  PDF maps.
\endlist

To compile Loch, you need

\list
* freetype 2 and newer; freetype-config must work
* wxWidgets 2.6 and newer; wx-config must work
* VTK 5.0 and newer
* libjpeg, libpng, zlib
\endlist

All programs (with the exception of BWidget and tkImg package) are usually 
included in Linux, Unix or MacOS\,X distributions.
For Windows consider using MinGW and MSYS (\www{http://www.mingw.org}).
It's a distribution of GNU utilities with GNU make and GCC.
(BTW, why not to use precompiled Windows version?)

\subsubchapter Quick start.

\list
* unpack the source distribution |therion-5.*.tar.gz|
* |cd therion|
* |make config-macosx| or |make config-win32|, if you use MacOS~X or Windows, 
  respectively
* |make|
* |sudo make install|
\endlist

%Installing Tom:
%
%\list
%* if you use Windows, download a Tcl/Tk source distribution, |make| and
%  |make install| it under MSYS
%* |cd therion/thtom/linux| or |cd therion/thtom/win|
%* |make|
%* copy |Tom0.2| directory (which should contain |pkgIndex.tcl| and one of 
%  |libtom.so| or |libtom.dll|) 
%  to the |lib| subdirectory of your 
%  Tcl/Tk distribution.
%\endlist

\subsubchapter Hacker's guide.

{\it Make parameters}

Therion's {\it makefile} may take some optional parameters.

\list
* |config-linux|, |config-macosx|, |config-win32| = configure Therion for a 
  specific platform. Linux is a default.
* |config-release|, |config-oxygen|, |config-ozone| = set optimization level 
  for C++ compiler (none, |-O2| and |-O3|)  
* |config-debug| = useful before debugging the program  
* |install| = install Therion
* |clean| = delete all temporary files
\endlist

{\it Adding new translations}

Therion supports translation of map labels. 
Suppose you want to add a new language |xx|. 

\list
* run `|perl process.pl export xx|' in the `thlang' Therion source subdirectory. 
  This creates a file |texts_xx.txt|. This file is UTF-8 encoded.
* edit the |texts_xx.txt| file. Add your translations at lines beginning with
  `|xx:|'.
* run |make update|
* compile Therion
\endlist


{\it Adding new encodings}

Although UTF-8 Unicode encoding covers all characters which Therion is able to 
process, it may be inconvenient to use it. In that case it's possible to add 
support for any 8-bit encoding for text input files. Copy a translation file to 
the |thchencdata| directory; add its name to `ifiles' hash in the beginning of 
the Perl script |generate.pl|; run it and recompile Therion.

The translation file should contain two hexadecimal values of a character 
(first one in the 8-bit encoding, second one in Unicode) in each line. Possible 
comments follow the `|#|' character. 

{\it Adding new \TeX\ encodings}

It's easy to add new encodings for 2D map output.%
\[\NEW{5.3}This section applies to old-style font selection using |tex-fonts|
command in the initialization file and is obsolete when using |pdf-fonts| 
command.]
Copy an appropriate encoding 
mapping file with an |*.enc| extension to the |texenc/encodings|, run the Perl 
script |mktexenc.pl| located in the |texenc| directory and compile Therion.

Therion uses the same encoding files as |afm2tfm| program from the \TeX\ 
distribution, which has the same format as an encoding vector in a PostScript 
font. You may find more details in the chapter {\it 6.3.1.5 Encoding file 
format} in the documentation to Dvips program.


{\it Generating new \TeX\ and \MP\ headers}

Therion uses \TeX\ and \MP\ for 2D map visualization and typesetting. 
Predefined macros are compiled into the Therion executable and are copied to 
the working directory just before running \MP\ and \TeX\ (unless the 
|--use-extern-libs| option is used). Layout command makes it possible to modify 
some macros in the configuration file at the run-time. 

However, it's possible to make permanent changes to the macro files. After 
modifying the files in the |mpost| and |tex| directories it's necessary to run 
Perl scripts |genmpost.pl| and |gentex.pl|, which generate C++ header files, 
and compile Therion executable again.

\subchapter Environment variables.

Therion reads following environment variables:

\list
* |THERION| = [not required] search path for (x)therion.ini file(s)
* |HOME| (|HOMEDRIVE| + |HOMEPATH| on WinXP) = 
  [not required, but usually present on your system] search path 
  for (x)therion.ini file(s)
* |TEMP|, |TMP| = system temporary directory, where Therion stores temporary 
  files (in a directory named |th$PID$|, where |$PID$| is a process ID),
  unless |tmp-path| is specified in the initialization file.
\endlist

Consult the documentation of your OS how to set them.

\subchapter Initialization files.

Therion's and XTherion's system dependent settings are specified in the 
file therion.ini or xtherion.ini, respectively.
They are searched for in the following directories:

\list
* on UNIX: 
  |.|, |$THERION|, |$HOME/.therion|, |/etc|, |/usr/etc|, |/usr/local/etc|
* on Windows:
  |.|, |$THERION|, |$HOME\.therion|, |<Therion-installation-directory>|, 
  |C:\WINDOWS|, |C:\WINNT|, |C:\Program Files\Therion|
\endlist
 
\subsubchapter Therion.

If no file is found Therion uses its default settings. If you want to list
them, use |--print-init-file| option. The initialization file is read 
like any other therion file. (Empty lines or lines starting with `|#|' are 
ignored; lines ending with a backslash continue on next line.) Currently 
supported initialization commands follow.

\list
* |loop-closure <therion/survex>|

  By default, survex is used if present, otherwise therion.

* |encoding-default <encoding-name>|

  Set the default output encoding (currently unused).

* |encoding-sql <encoding-name>|

  Set the default output encoding for SQL export.
        
* |language <xx[_YY]>|  

  Default output language. See the copyright page for
  the list of available languages.

* |units <metric/imperial>| 

  Set default units.

* |mpost-path <file-path>|

  Set the full path to a \MP\ executable if Therion can't find it 
  (``|mpost|'' is the default).

* |mpost-options <string>|

  Set \MP\ options.

* |pdftex-path <file-path>|

  Set the full path to a pdf\TeX\ executable if Therion can't find it 
  (``|pdfetex|'' is the default).

* |identify-path <file-path>|

  Set the full path to ImageMagick's identify executable if Therion
  can't find it (``|identify|'' is the default).
  
* |convert-path <file-path>|

  Set the full path to ImageMagick's convert executable if Therion
  can't find it (``|convert|'' is the default).

* |source-path <directory>| 

  Path to data and configuration files. Used mostly for system-wide grades and 
  layout definitions.

* |tmp-path <directory>| 

  Path where temporary directory should be created.
  
* |tmp-remove <OS command>| 

  System command to delete files from the temporary directory.

* |tex-env <on/off>| 

  [Works on Windows only.]
  When set to |off| (default), Therion temporarily clears all environment 
  variables related to \TeX. Useful if there is other \TeX distribution
  installed on your system which had set-up any environment variables,
  which could confuse \TeX\ and \MP\ programs supplied in Therion for Windows
  distribution. 
  
  Set to |on| if you use other \TeX\ distribution for maps processing.

* |text <language ID> <therion text> <my text>|

  Using this option you can change any default therion text translation in output.
  For list of therion texts and available translations, see |thlang/texts.txt| file.

* |cs-def <id> <proj4def>|\NEW{5.4}  %[other options]

  Define a new coordinate system |<id>| using Proj4 syntax.

* |pdf-fonts <rm> <it> <bf> <ss> <si>|\NEW{5.3}

  Set-up fonts to be used in PDF maps. 
  The command has to be followed by paths specifying where regular, italic,
  bold, sans-serif and sans-serif oblique fonts are located in your system.
  TrueType and OpenType fonts are supported. 
  
  Therion requires LCDF Typetools to be installed on your system to use this
  command. Example:
  
  |pdf-fonts  "/usr/share/fonts/Serif.ttf" \
           "/usr/share/fonts/Serif-Italic.ttf" \
           "/usr/share/fonts/Serif-Bold.ttf" \
           "/usr/share/fonts/Sans.ttf" \
           "/usr/share/fonts/Sans-Oblique.ttf"|

* |otf2pfb <on/off>|\NEW{5.3}

  When set to |on| (default), OpenType fonts used in |pdf-fonts| are
  converted to PFB fonts, if they are PostScript-based. Some information
  is lost in the PFB format, but there is advantage that pdf\TeX\ can embed 
  subset of PFB fonts (in contrast with OpenType fonts which must be fully 
  embedded).

* |tex-fonts <encoding> <rm> <it> <bf> <ss> <si>|
        
  Original and more complicated way to set-up fonts for PDF maps. You need
  to explicitly specify encoding (maximum 256 characters from the font
  that will be used). The list of currently supported 
  encodings gives the |--print-tex-encodings| command line option.   
  The same encoding must be used while generating \TeX\ metrics (|*.tfm| files) 
  for those fonts (e.g.~with the afm2tfm program) and this encoding must be
  explicitly given also in the pdf\TeX's map file. The only exception is the 
  base set of Computer Modern fonts, which use `raw' encoding. This encoding
  doesn't need to be specified in the pdf\TeX's map file.
  
  Encoding has to be followed by five font specifications for regular, italic,
  bold, sans-serif and sans-serif oblique styles.
  Default setting is |tex-fonts raw cmr10 cmti10 cmbx10 cmss10 cmssi10|
  
  Example how to use other fonts (e.g.~TrueType Palatino in xl2 (an encoding 
  derived from ISO8859-2) encoding). Run:
  
  |ttf2afm -e xl2.enc -o palatino.afm palatino.ttf|
  
  |afm2tfm palatino.afm -u -v vpalatino -T xl2.enc|

  |vptovf vpalatino.vpl vpalatino.vf vpalatino.tfm|
  
  You get files |vpalatino.vf|, |vpalatino.tfm| and |palatino.tfm|. Add the line 
  
  |palatino <xl2.enc <palatino.ttf|
  
  to the pdf\TeX's map file. The same should be done for the italic and bold
  faces and corresponding sans-serif and sans-serif-oblique fonts. If you're lazy
  try 
  
  |tex-fonts xl2 palatino palatino palatino palatino palatino|
  
  (We should use actually virtual font |vpalatino| instead of |palatino|,
  which contains no kerning or ligatures, but
  pdf\TeX\ doesn't support |\pdfincludechars| command on virtual fonts.
  To be improved.)
    
  If you want to add some unsupported encodings, 
  read the chapter {\it Compilation / Hacker's guide}. 
\endlist


\subsubchapter XTherion.

Initialization file for XTherion is actually a Tcl script evaluated when 
XTherion starts. The file is commented; see the comments for details.


%\subsubchapter Speed optimization.
%
%[Optionally creating \MP\ and \TeX\ format files.]

\subchapter Limitations.

\list
*  scrap size = $\approx 2.8 \times 2.8$ m in the output scale (\MP\ limit)
*  page size = 

   PDF map or atlas: $\approx 5 \times 5$ m (pdf\TeX\ limit)
   
   SVG map: unlimited 
*  scraps count = approx. 500--6000, depending on frequency of cross-sections
   
   current \MP\ limit: $4 (scraps + sections) < 4096$ (may be arbitrarily increased)

   pdf\TeX\ limit: $2\times pages + images + patterns +
                            6 (scraps + sections) < 32500$
\endlist


\subchapter Example data.

Following simple example illustrates basic usage of Therion commands:

|encoding  utf-8

survey main -title "Test cave"
  
  survey first
    centreline
      units compass grad
      data normal from to compass clino length
                  1    2  100     -5    10
    endcentreline
  endsurvey

  survey second -declination [3 deg]
    centreline
      calibrate length 0 0.96
      data normal from to compass length clino
                  1    2  0       10     +10
    endcentreline
  endsurvey
 
  centreline
    equate 2@first 1@second
  endcentreline
 
  # scraps are usually in separate *.th2 files
  scrap s1 -author 2004 "Therion team"

    point 763 746 station -name 2@second
    point 702 430 station -name 2@first
    point 352 469 station -name 1@first
    point 675 585 air-draught -orientation 240 -scale large

    line wall -close on
      287 475
      281 354 687 331 755 367
      981 486 846 879 683 739
      476 561 293 611 287 475
    endline

  endscrap

  map m1 -title "Test map"
    s1
  endmap
 
endsurvey|

Corresponding configuration file could be:

|encoding  utf-8
source test

layout l1
  scale 1 100
  layers off
endlayout

select m1@main

export model -fmt survex
export map -layout l1|

If you save data file as `test.th' and configuration file as `thconfig' you may 
process them with Therion.


%\subchapter Map symbols library.

%\input sym/symlib

\subchapter History.

\list
\everypar{\hangindent16pt\hangafter1}
* {\bf 1999}

  Oct: first concrete ideas

  Nov: start of programming (Perl scripts and \MP\ macros)

  Dec 27: Therion compiles simple map in PostScript format 
       for the first time (32 kB of Perl and 7 kB of \MP\ and \TeX\ source code).
       The map warping model was substantially different
       from the current one (positions of features were relative to 
       a particular survey shot, not to positions of all stations in a scrap).
       This version already included some interesting features
       such as {\it transformation functions} which allowed user specification
       of the input format for survey data, or splitting large maps to
       multiple sheets.
				
  Dec 30: the first web page (with data examples but without source
       code)
  
* {\bf 2000}\nobreak\par\nobreak
  Jan: xthedit (Tcl/Tk), a graphical front-end for Therion

  Feb 18: start of reprogramming (Perl)

  Apr 1: the first hyperlinked PDF cave map / atlas

  Aug: experiments with PDF, pdf\TeX\ and \MP

* {\bf 2001}

  Nov: start of reimplementation from scratch: 
       Therion (C++ with some Perl scripts inherited from the previous version); 
       notion of a scrap;
       interactive 2D map editor ThEdit as a replacement of xthedit (Delphi) 

  Dec: ThEdit exports simple map for the first time

* {\bf 2002}

  Mar: Therion 0.1 ---
       Therion is able to process survey data (centreline) of the Cave of Dead Bats.
       XTherion, text editor designed for Therion (Tcl/Tk).

  Jul 27: Therion 0.2 ---
       Therion compiles simple map (consisting of two scraps)
       for the first time (800 kB of source code)

  Aug: XTherion extended to 2D map editor (as a replacement of ThEdit)

  Sep: Therion compiles first real and complex map of a cave. XTherion
       extended to compiler.

%  Oct 5: Public presentation of Therion, held on Trango\v{s}ka

* {\bf 2003}\nobreak

  Mar: the first version of The Therion Book finished

  Apr: Therion included in Debian GNU/Linux

  Jun: all Perl scripts rewritten in C++, Therion is one executable program
       now (although using Survex and \TeX)
       
%  Nov 10: first idea of the Labyrinth project

* {\bf 2004}\Nobreak

  Mar: Therion 0.3 --- Therion exports 3D model created from 2D maps.
  Loop closure algorithm included into Therion.

* {\bf 2006}\Nobreak

  Oct: Therion 0.4 --- New 3D viewer (Loch).

* {\bf 2007}\Nobreak

  Feb: Therion 0.5 --- Support for bitmap sketches morphing.
\endlist


\subchapter Future.

Although Therion is already used for map production, there are a lot of 
new features to be implemented:

\subsubchapter General.

\list
* loop closure information in SQL
\endlist

\subsubchapter 2D maps.

\list
* complete the predefined symbol sets
* generate registers for atlas
* use MPlib instead of \MP
\endlist


\subsubchapter 3D models.

\list
* improve passage walls modeling
\endlist

\subsubchapter XTherion.

\list
* improve 2D editing capabilities
\endlist

\subsubchapter Loch.

\list
* colour schemes
* survey tree for selecting sub-surveys to display
* spatial filtering (e.g.~clipping by planes)
* support for multiple surfaces
\endlist

\subsubchapter Labyrinth.

\list
* completely new GUI in the far future (see \www{http://labyrinth.speleo.sk})
\endlist



\endinput
