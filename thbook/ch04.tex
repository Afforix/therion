\eject
\chapter What do we get.

\subchapter Information files.

\subsubchapter Log file.

Besides the messages from Therion and other programs used, the log file
contains information about computed values of magnetic declination and meridian
convergence, loop errors, scrap distortions and transformations between
coordinate reference systems chosen by the Proj library.

Absolute loop error is $\sqrt{\Delta x^2+\Delta y^2+\Delta z^2}$, where
$\Delta x$ is the difference between the identical start and end points of
the loop before the error distribution measured along the $x$ coordinate axis;
similarly for $y$ and
$z$. Percentage loop error is calculated as {\it absolute error / loop length}.
Average error is the simple arithmetic average of all loop errors.

Scrap distortion is calculated using the distortion measure defined for
all pairs of points (point symbols, points and control points of line symbols)
in the scrap. The measure is calculated as $\left\vert d_a-d_b\right\vert\over d_b$, where $d_b$ is
the distance of points before warping and $d_a$ is the distance of points after
warping. The maximal and average scrap distortions are calculated as a maximum
or average of such measures applied to all pairs of points.

The log file also provides information about transformations between coordinate
systems, either user-specified pipelines or the transformations chosen by the Proj library.


\subsubchapter XTherion.

Therion provides some basic facts about each survey (length, vertical range,
N--S range, E--W range, number of shots and stations) if |-x| option is given.
This information is displayed in XTherion, {\it Compiler} window,
{\it Survey info} menu, when some survey from the {\it Survey structure} menu
is selected.

\subsubchapter SQL export.

SQL export makes it easy to get very detailed and subtle information about
the centreline. It is a text file starting with a tables declaration (where `?'
stands in the following listing for a maximal value required by the column
data)

|create table SURVEY (ID integer, PARENT_ID integer,
  NAME varchar(?), FULL_NAME varchar(?), TITLE varchar(?));
create table CENTRELINE (ID integer, SURVEY_ID integer,
  TITLE varchar(?), TOPO_DATE date, EXPLO_DATE date,
  LENGTH real, SURFACE_LENGTH real, DUPLICATE_LENGTH real);
create table PERSON (ID integer, NAME varchar(?), SURNAME varchar(?));
create table EXPLO (PERSON_ID integer, CENTRELINE_ID integer);
create table TOPO (PERSON_ID integer, CENTRELINE_ID integer);
create table STATION (ID integer, NAME varchar(?),
  SURVEY_ID integer, X real, Y real, Z real);
create table STATION_FLAG (STATION_ID integer, FLAG char(3));
create table SHOT (ID integer, FROM_ID integer, TO_ID integer,
  CENTRELINE_ID integer, LENGTH real, BEARING real, GRADIENT real,
  ADJ_LENGTH real, ADJ_BEARING real, ADJ_GRADIENT real,
  ERR_LENGTH real, ERR_BEARING real, ERR_GRADIENT real);
create table SHOT_FLAG (SHOT_ID integer, FLAG char(3));|

which is followed by a mass of SQL insert commands. This file may be loaded
into any SQL database (after some database-dependent initialization, which may
include running an SQL server and connecting to it, creating a database and
connecting to it. A good idea is to start a transaction before loading this
file, if the database doesn't start a transaction automatically.)
It's important to set-up database encoding to match the one
specified in Therion |export database| command.

\iffulloutput
\midinsert
    \centerline{\pic{database.pdf}}%
\endinsert
\fi

Table and column names are self-explanatory; for undefined or non-existing
values |NULL| is used. |FLAG| in |SHOT_FLAG| table is |dpl| or |srf| for duplicated
or surface shots; in |STATION_FLAG| table |ent|, |con|, |fix|,
|spr|, |sin|, |dol|, |dig|, |air|, |ove|, |arc| for stations
with entrance, continuation, fixed, spring, sink, doline, dig, air-draught,
overhang or arch attributes, respectively.

Examples of simple queries follow:

\penalty-100{\it List of survey team members with information about how
much each of them has surveyed:}

|select sum(LENGTH), sum(SURFACE_LENGTH), NAME, SURNAME
  from CENTRELINE, TOPO, PERSON
  where CENTRELINE.ID = TOPO.CENTRELINE_ID and PERSON.ID = PERSON_ID
  group by NAME, SURNAME order by 1 desc, 4 asc;|

{\it Which parts of the cave were surveyed in the year 1998?}

|select TITLE from SURVEY where ID in
  (select SURVEY_ID from CENTRELINE
  where TOPO_DATE between '1998-01-01' and '1998-12-31');|

{\it How long are the passages lying between 1500 and 1550 m a.s.l.?}

|select sum(LENGTH) from SHOT, STATION S1, STATION S2
  where (S1.Z+S2.Z)/2 between 1500 and 1550 and
  SHOT.FROM_ID = S1.ID and SHOT.TO_ID = S2.ID;|

\subsubchapter Lists---caves, surveys, continuations.

Using |export continuation-list| you get an overview of all points in the
centreline and scraps marked \[Using |station| attribute for centreline points
and |point continuation| in scraps.] as a possible continuation.

|export cave-list| gives you a tabular information about surveyed caves (you need
to specify |entrance| flags in your data) including length, depth and entrance(s)
location.

Detailed information about each sub-survey gives |export survey-list| command.
The length includes shots with |approximate| flags, but not |explored|,
|duplicate| or |surface|.

\subchapter 2D maps.

\subsubchapter Maps for printing.

Maps are produced in PDF and SVG formats, which may be viewed or printed in a wide
variety of viewers. Be sure to uncheck {\it Fit page to paper} or similar
option if you want to print in the exact scale.

In atlas mode some additional information is put on each page: page
number, map name, and page label.

The numbers of neighbouring pages in N, S, E and W
directions, as well as in upper and lower levels are especially useful. There are also hyperlinks at
the border of the map if the cave continues on the next page and on the
appropriate cells of the Navigator.

PDF files are highly optimized---scraps are stored in XObject forms only once
in the document and then referenced on appropriate pages.
Therion uses advanced PDF features like transparency and layers.

Created PDF files may be optionally post-processed in applications like
pdf\TeX\ or Adobe Acrobat---it's possible to extract or change some pages, add
comments or encryption, etc.

\NEW{5.3}If the map was produced using georeferenced data then it also contains
georeferencing information. This can be extracted by XTherion to produce
georeferenced raster images (see {\it XTherion/Additional tools} for
details).


\subsubchapter Maps for GIS.

Maps produced in DXF, ESRI or KML formats may be further processed in
appropriate software. These maps do not contain visualized map symbols


\subsubchapter Special-purpose maps.

Map in XVI format contains centreline with LRUD information (and optionally morphed
sketches) and can be imported in XTherion to serve as a background for
digitization.

Map in Survex format is intended for a quick preview in Aven.


\subchapter 3D models.

Therion may export 3D models in various formats besides its native format. These
may be loaded in appropriate viewing, editing or raytracing programs to be
printed or further processed. If the format doesn't support arbitrary passage
shape definition, only the centreline is included.

\subsubchapter Loch.

Loch is a 3D model viewer included in the Therion distribution. It supports
e.g.~high-resolution rendering to file and stereo view using 3D-glasses.

\endinput
