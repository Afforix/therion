\chapter What we get?.

\subchapter Information files.

Therion provides some basic facts about each survey (length, vertical range,
N--S range, E--W range, number of shots and stations) if |-x| option is given. 
This information is dispayed in XTherion, {\it Compiler} window,
{\it Survey info} menu, when some survey from the {\it Survey structure} menu
is selected.

More advanced statistics should be implemented in the near future.


\subchapter 2D maps.

Maps are produced in PDF format, which may be viewed or printed in a wide 
variety of viewers.

In atlas mode some additional information is put on each page: page 
number, map name, and page label.

Especially useful are the numbers of neighbouring pages in N, S, E and W 
directions, as well as in upper and lower levels. There are also hyperlinks at 
the border of the map if the cave continues on the next page and on the 
appropriate cells of the Navigator.

PDF files are highly optimized---scraps are stored in XObject forms only once 
in the document and than referenced on appropriate pages. 
Therion uses most advanced PDF features like transparency and layers.

Created PDF files may be optionally post-processed in applications like 
pdf\TeX\ or Adobe Acrobat---it's possible to extract or change some pages, add 
comments or encryption, etc.


\subchapter 3D models.

Therion exports centreline 3D models in Survex and Compass formats, which may 
be viewed in these programs. (Aven, xcaverot, or caverot for *.3d Survex files.)
You may also use |print*| programs from the Survex distribution to print the 
3D model.

In the future there should be full passage modelling in Therion. In order to 
display these models, XTherion will be extended to include an OpenGL 3D viewer.

\endinput
