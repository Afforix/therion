\chapter What we get?.

\subchapter Information files.

\subsubchapter Log file.

Besides the messages from Therion and other programs used, the log file 
contains information about average and maximal distortion for each scrap. 

\subsubchapter XTherion.

Therion provides some basic facts about each survey (length, vertical range,
N--S range, E--W range, number of shots and stations) if |-x| option is given. 
This information is displayed in XTherion, {\it Compiler} window,
{\it Survey info} menu, when some survey from the {\it Survey structure} menu
is selected.

\subsubchapter SQL export.

SQL export makes it easy to get very detailed and subtle informations about
centreline. It is a text file starting with tables declaration (where `?' 
stands in the following listing for a minimal value required by the column 
data)

|create table SURVEY (ID integer, PARENT_ID integer, 
  NAME varchar(?), FULL_NAME varchar(?), TITLE varchar(?));
create table CENTRELINE (ID integer, SURVEY_ID integer, 
  TITLE varchar(?), TOPO_DATE date, EXPLO_DATE date, 
  LENGTH real, SURFACE_LENGTH real, DUPLICATE_LENGTH real);
create table PERSON (ID integer, NAME varchar(?), SURNAME varchar(?));
create table EXPLO (PERSON_ID integer, CENTRELINE_ID integer);
create table TOPO (PERSON_ID integer, CENTRELINE_ID integer);
create table STATION (ID integer, NAME varchar(?), 
  SURVEY_ID integer, X real, Y real, Z real);
create table STATION_FLAG (STATION_ID integer, FLAG char(3));
create table SHOT (ID integer, FROM_ID integer, TO_ID integer, 
  CENTRELINE_ID integer, LENGTH real, BEARING real, GRADIENT real, 
  ADJ_LENGTH real, ADJ_BEARING real, ADJ_GRADIENT real, 
  ERR_LENGTH real, ERR_BEARING real, ERR_GRADIENT real);
create table SHOT_FLAG (SHOT_ID integer, FLAG char(3));|

which is followed by a mass of SQL insert commands. This file may be loaded 
into any SQL database (after some database-dependent initialization, which may 
include running a SQL server and connecting to it, creating a database and 
connecting to it). It's important to set-up database encoding to match the one 
specified in Therion |export database| command.

Table and column names are self-explaining; for undefined or non-existing 
values |NULL| is used. Examples of simple queries follow:

\penalty-100{\it List of survey team members with an information how
much has each of them surveyed:}

|select sum(LENGTH), sum(SURFACE_LENGTH), NAME, SURNAME 
  from CENTRELINE, TOPO, PERSON 
  where CENTRELINE.ID = TOPO.CENTRELINE_ID and PERSON.ID = PERSON_ID 
  group by NAME, SURNAME order by 1 desc, 4 asc;|

{\it Which parts of the cave were surveyed in the year 1998?}

|select TITLE from survey where ID in 
  (select SURVEY_ID from CENTRELINE 
  where TOPO_DATE between '1998-01-01' and '1998-12-31');|

{\it How long are pasages lying between 1500 and 1550 m a.s.l.?}

|select sum(LENGTH) from SHOT, STATION S1, STATION S2 
  where (S1.Z+S2.Z)/2 between 1500 and 1550 and 
  SHOT.FROM_ID = S1.ID and SHOT.TO_ID = S2.ID;|


\subchapter 2D maps.

Maps are produced in PDF format, which may be viewed or printed in a wide 
variety of viewers. Be sure to uncheck {\it Fit page to paper} or similar 
option if you want to print in the exact scale.

In atlas mode some additional information is put on each page: page 
number, map name, and page label.

Especially useful are the numbers of neighbouring pages in N, S, E and W 
directions, as well as in upper and lower levels. There are also hyperlinks at 
the border of the map if the cave continues on the next page and on the 
appropriate cells of the Navigator.

PDF files are highly optimized---scraps are stored in XObject forms only once 
in the document and than referenced on appropriate pages. 
Therion uses most advanced PDF features like transparency and layers.

Created PDF files may be optionally post-processed in applications like 
pdf\TeX\ or Adobe Acrobat---it's possible to extract or change some pages, add 
comments or encryption, etc.


\subchapter 3D models.

Therion exports centreline 3D models in Survex and Compass formats, which may 
be viewed in these programs. (Aven, xcaverot, or caverot for *.3d Survex files.)
You may also use |print*| programs from the Survex distribution to print the 
3D model.

In the future there should be full passage modelling in Therion. In order to 
display these models, XTherion will be extended to include an OpenGL 3D viewer.

\endinput
