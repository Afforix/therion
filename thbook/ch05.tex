\chapter Changing layout.

This chapter is extremely useful if you're not satisfied with the predefined 
layout of map symbols and maps provided, and want to adapt them to your needs.
However, you need to know how to write plain \TeX\ and \MP\ macros to do this.


\subchapter Page layout in the atlas mode.

The |layout| command allows basic page setup in the atlas mode. This is done 
through its options such as |page-setup| or |overlap|. But there are no 
options which would specify the position of map, navigator and other elements 
inside the area defined by |page-width| and |page-height| dimensions; e.g., 
why is the navigator below the map and not on its right or left side?

There are many possible arrangements for a page. Rather than offer even more
options for the |layout| command, Therion uses the \TeX\ language to 
describe other page layouts. 
%(See the section `layout' where to put \TeX\ commands.) 
This approach has the advantage that the user has 
direct access to the advanced typesetting engine without making the language of 
Therion overcomplex.

Therion uses pdf\TeX\ with the {\it plain} format for typesetting. So you 
should be familiar with the plain \TeX\ if you wish to define new layouts.

The ultimate reference for plain \TeX\ is
\list
  Knuth, D. E.: {\it The \TeX book}, 
                Reading, Massachusetts, Addison-Wesley $^1$1984 %[...]
\endlist

For pdf\TeX's extensions there is a short manual
\list
  Th\`anh, H. T.---Rahtz, S.---Hagen, H.: {\it The pdf\TeX\ user manual}, 
                available at \path|http://www.pdftex.org|
\endlist

The \TeX\ macros are used inside of |layout| command (see the chapter 
{\it Processing data} for details). The basic one predefined by Therion is the

|\dopage|

macro. The idea is simple: for each page Therion defines \TeX\ variables 
(count, token, and box registers) which contain the page elements (map, 
navigator, page name etc.). At the end of each page macro |\dopage| is invoked. 
This defines the position of each element on the page. By redefining this macro 
you'll get desired page layout. Without this redefinition you'll get the 
standard layout.

Here is the list of variables defined for each page:

\list
  {\it Boxes:}

  * |\mapbox| = The box containing the map. Its width (height) is set 
    according to the |size| and |overlap| options of the |layout| command to 

    |size_width + 2*overlap| or 
    
    |size_height + 2*overlap|, respectively

  * |\navbox| = The box containing the navigator, with dimensions

    |size_width * (2*nav_size_x+1) / nav_factor| or

    |size_height * (2*nav_size_y+1) / nav_factor|, respectively

    Both |\mapbox| and |\navbox| also contain hyperlinks.

  {\it Count registers:}

  * |\pointerE|, |\pointerW|, |\pointerN|, |\pointerS| contain the page number 
    of the neighbouring pages in the E, W, N and S directions. If there is no 
    such a page its page number is set to 0.

  * |\pagenum| current page number

  {\it Token registers:}

  * |\pointerU|, |\pointerD| contain information about pages above and below
    the current page. It consists of one or more concatenated records. Each 
    record has a special format
    
    {\tt|page-name|\char124|page-number|\char124|destination|\char124\char124}
    
    If there are no such pages, the value is set to |notdef|.

    See the description of the |\processpointeritem| macro below for how to 
    extract and use this information.

  * |\pagename| = name of the current map according to options of the |map|
    command.

  * |\pagelabel| = the page label as specified by |origin| and |origin-label|
    options of the |layout| command.
\endlist


The following variables are set at the beginning of the document:

\list
  * |\hsize|, |\vsize| = \TeX\ page dimensions, set according to |page-width|
    and |page-height| parameters of the |page-setup| option of the |layout|
    command. They determine our playground when defining page layout using 
    the |\dopage| macro.

  * |\ifpagenumbering| = This conditional is set true or false according to
    the |page-numbers| option of the |layout| command.
\endlist

There are also some predefined macros which help with the processing of 
|\pointer*| variables:

\list
  * |\showpointer| with one of the |\pointerE|, |\pointerW|, |\pointerN| or 
    |\pointerS| as an argument displays the value of the argument. If the value
    is 0 it doesn't display anything. This is useful because
    the zero value (no neighbouring page) shouldn't be displayed.
    
  * |\showpointerlist| with one of the |\pointerU| or |\pointerD| as an argument 
    presents the content of this argument. (Which contains 
    |\pointerU| or |\pointerD|, see above.) For each record it calls the macro 
    |\processpointeritem|, which is responsible for data formating. 
    
    Macro |\showpointerlist| should be used without redefinition in the place
    where you want to display the content of its argument; for custom data
    formating redefine |\processpointeritem| macro.
         
  * |\processpointeritem| has three arguments (page-name, page-number, 
    destination) and visualizes these data. The arguments are delimited as 
    follows
    
    {\tt|\def\processpointeritem#1|\char124|#2|\char124|#3\endarg{...}|}

    An example definition may be

    {\tt|\def\processpointeritem#1|\char124|#2|\char124}|#3\endarg{%
  \hbox{\pdfstartlink attr {/Border [0 0 0]}% 
        goto name {#3} #2 (#1)\pdfendlink}%
}|

   (note how to use the {\it destination} argument), 
   or much simpler (if we don't need hyperlink features):

    {\tt|\def\processpointeritem#1|\char124|#2|\char124}|#3\endarg{%
  \hbox{#2 (#1)}%
}|
\endlist


Now we're ready to define the |\dopage| macro. You may choose which of the 
predefined elements to use. A very simple example would be

|layout my_layout
  scale 1 200
  page-setup 29.7 21 27.7 19 1 1 cm
  size 26.7 18 cm
  overlap 0.5 cm

  \def\dopage{\box\mapbox}

  \insertmaps 

endlayout|

which defines the landscape A4 layout without the navigator nor any texts. There 
is only a map on the page.

Note the |\insertmaps| macro. Map pages are inserted at its position. [Similar 
macro |\insertlegend| will insert the map legend when Therion supports it.]
This is not done automatically because you may wish to insert some other pages 
before the first map page. (In this case something like |\input myfile.tex| 
should preceed the |\insertmaps| macro.) Without the |\insertmaps| invocation 
no map pages will be included!

More advanced is the default definition of the |\dopage| macro:

|\def\dopage{%
  \vbox{\line{\hfil\box\mapbox\hfil}
    \bigskip
    \line{%
     \vbox to \ht\navbox{
      \hbox{\big\the\pagelabel
        \ifpagenumbering\space(\the\pagenum)\fi
        \space\middle\the\pagename}
      \ifpagenumbering
        \medskip
        \hbox{\qquad\qquad
          \vtop{%
  	    \hbox to 0pt{\hss\showpointer\pointerN\hss}
            \hbox to 0pt{\llap{\showpointer\pointerW\hskip0.7em}%
	           \raise1pt\hbox to 0pt{\hss$\updownarrow$\hss}%
	           \raise1pt\hbox to 0pt{\hss$\leftrightarrow$\hss}%
		   \rlap{\hskip0.7em\showpointer\pointerE}}
  	    \hbox to 0pt{\hss\showpointer\pointerS\hss}
	  }\qquad\qquad
          \vtop{
	    \def\arr{$\uparrow$}
	    \showpointerlist\pointerU
	    \def\arr{$\downarrow$}
	    \showpointerlist\pointerD
	  }
        }
      \fi
      \vss}\hss
      \box\navbox
    }
  }
}|

{\it Note:} |\big| and |\middle| are font switches

Using other plain \TeX\ macros or \TeX\ primitives it's possible to add other 
features, e.g.~a different layout for odd and even pages; headers and footers;
or adding a logo to each page. 

\subchapter New map symbols.

[This chapter will explain how to define new map symbols in \MP. Currently 
Therion doesn't support custom map symbol sets.]

%\subchapter More about maps.

%\subchapter Dirty tricks.

\endinput
