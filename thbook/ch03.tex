\eject
\iffulloutput
\vbox{\hsize=93.2mm\baselineskip=12pt
\rightline{\eightit We both step and do not step in the same rivers.}
\pic{herakl.pdf}
\vskip1pt
\eightrm
\rightline{---Heraclitus of Ephesus, {\eightmit 6}th/{\eightmit 5}th century BC}
}
\fi

\chapter Processing data.

Besides data files, which contain survey data, Therion uses a configuration file,
which contains instructions on how the data should be presented.

\subchapter Configuration file.

The configuration filename can be given as an
argument to therion. By default Therion searches for file named |thconfig|
in the current working directory. It is read like any other therion file
(i.e.~one command per line; empty lines or lines starting with `|#|' are ignored;
lines ended with a backslash continue on the next line.) A list of currently
supported commands follow.


\subsubchapter `system'.
Allows to execute system commands during therion compilation.\[E.g. to open or
refresh external PDF viewer.] Normally Therion waits until the subprocess is
finished. If you want to continue compilation without break, use |<command> &|
syntax on Linux and |start <command>| syntax on Windows.


\subsubchapter `encoding'.

Works like the |encoding| command in data files---specifies character sets.

\subsubchapter `language'.
\syntax
  \list
    * |language <xx_[YY]>|
  \endlist
\endsyntax

\NEW{5.3}Sets the output language for translatable texts.

\subsubchapter `cs'.

\syntax
  \list
    * |cs <coordinate system>|
  \endlist
\endsyntax

\NEW{5.3}Outside of |layout| command specifies the coordinate system for output.
It is not possible to specify more coordinate systems for different outputs (the
last occurrence of |cs| is used for all output files).

If no |cs| is defined in the configuration file, the first |cs| therion
encounters in the data files is used as an output |cs|.

Inside
the |layout| specifies coordinate system for subsequent location data (|origin|,
|grid-origin|).

\subsubchapter `sketch-warp'.

\syntax
  \list
    * |sketch-warp <algorithm>|
  \endlist
\endsyntax

Specifies which scrap warping (morphing) algorithm to use.
Possible algorithms are |line|---the default; |plaquette|---invented by
Marco Corvi.

\subsubchapter `input'.

Works like |input| command in data files---includes other files.

\subsubchapter `source'.

\description
   Specifies which source (data) files Therion should read. You can
   specify several files here; one per line.
   You can also specify them using the |-s| command
   line option (see below).

   It is also possible to type (some small snippets of) code directly in
   configuration file using the multi-line syntax.
\enddescription

\syntax
  \hfill\break|source <file-name>|\hfill\break
  or\hfill\break
  |source|\hfill\break
  {\qquad\it\dots therion commands\dots}\hfill\break
  |endsource|
\endsyntax

\arguments
*  |<file-name>|
\endarguments

\subsubchapter `select'.

\description
  selects objects (surveys and maps) for export. By default, all survey objects
  are selected. If there is no map selected, all scraps belonging to
  selected surveys are selected by default for map export.

  If there are no scraps or maps in the data, centreline from all surveys is
  exported in the map.

  When exporting maps in different projections, you need to select
  them for each projection separately.
\enddescription

\syntax
  |select <object> [OPTIONS]|
\endsyntax

\arguments
*  |<object>| = any survey or map, identified by its ID.
\endarguments

\options
  * |recursive <on/off>| = valid only when a survey is selected. If set
    on (by default) all subsurveys of the given survey are recursively
    selected/unselected.

  * |map-level <number>| = valid only when a map is selected. Determines
    the level at which map expansion for atlas export is stopped.
    By default 0 is used; if `basic' is specified,
    expansion is done up to the basic maps.
    {\it Note:} Map previews are displayed only as specified in maps in the
    current |map-level|.

  * |chapter-level <number>| = valid only when a map is selected. Determines
    the level at which chapter expansion for atlas export is stopped.
    By default 0 is used, if `|-|' or `|.|' is specified, no chapter is
    exported for this map. If |title-pages| option in |layout| is on,
    each chapter starts with a title page.
\endoptions



\subsubchapter `unselect'.

\description
  Unselects objects from export.
\enddescription

\syntax
  |unselect <object> [OPTIONS]|
\endsyntax

\arguments
  The same as the |select| command.
\endarguments

\options
  The same as the |select| command.
\endoptions

\subsubchapter `maps'.

\description
  Turns processing of maps on (default) or off. If you turn if off,
  all scraps from selected surveys will be used in the output, no map
  definitions are taken into account. Usefull for debugging map definitions.
\enddescription

\syntax
  |maps <on/off>|
\endsyntax

\subsubchapter `maps-offset'.

\description
  Turns drawing maps in offset on (default) or off. If you turn if off,
  all cave passages will be displayed in detail in their actual position.
  All offset specifications will be completely ignored.
\enddescription

\syntax
  |maps-offset <on/off>|
\endsyntax


\subsubchapter `log'.

\description
  Turn on logging of various info. Currently only extended elevation
  processing log is supported.
\enddescription

\syntax
  |log extend|
\endsyntax


\subsubchapter `text'.

\description
  Specifies translation of any default therion text in output.
\enddescription

\syntax
  |text <language ID> <therion text> <my text>|
\endsyntax

\arguments
*  |<language ID>| = standard ISO language identifier (e.g. |en| or |en_GB|)
*  |<therion text>| = therion text to translate. For list of therion texts and
available translations, see |thlang/texts.txt| file.
\endarguments


\subsubchapter `layout'.

\description
  Specifies layout for 2D maps. Settings which apply to atlas mode
  are marked `A'; map mode `M'.
\enddescription

\syntax
|layout <id> [OPTIONS]
    copy <source layout id>
    cs <coordinate system>
    north <true/grid>
    scale <picture length> <real length>
    base-scale <picture length> <real length>
    units <metric/imperial>
    rotate <number>
    symbol-set <symbol-set>
    symbol-assign <point/line/area/group/special> <symbol-type> \
                                                  <symbol-set>
    symbol-hide <point/line/area/group/special> <symbol-type>
    symbol-show <point/line/area/group/special> <symbol-type>
    symbol-colour <point/line/area/group/special> <symbol-type> <colour>
    min-symbol-scale <scale>
    fonts-setup <tinysize> <smallsize> <normalsize> <largesize> <hugesize>
    size <width> <height> <units>
    overlap <value> <units>
    page-setup <dimensions> <units>
    page-numbers <on/off>
    exclude-pages <on/off> <list>
    title-pages <on/off>
    nav-factor <factor>
    nav-size <x-size> <y-size>
    transparency <on/off>
    opacity <value>
    surface <top/bottom/off>
    surface-opacity <value>
    sketches <on/off>
    layers <on/off>
    grid <off/top/bottom>
    grid-origin <x> <y> <x> <units>
    grid-size <x> <y> <z> <units>
    grid-coords <off/border/all>
    origin <x> <y> <z> <units>
    origin-label <x-label> <y-label>
    own-pages <number>
    page-grid <on/off>
    legend <on/off/all>
    legend-columns <number>
    legend-width <n> <units>
    map-comment <string>
    map-header <x> <y> <off/n/s/e/w/ne/nw/se/sw/center>
    map-header-bg <on/off>
    map-image <x> <y> <n/s/e/w/ne/nw/se/sw/center> <filename>
    statistics <explo/topo/carto/copyright all/off/number>
               <explo/topo-length on/hide/off>
               <carto/copyright-count on/hide/off>
    scale-bar <length> <units>
    survey-level <N/all>
    language <xx[_YY]>
    colo[u]r-model <cmyk/rgb/grayscale>
    colo[u]r <item> <colour>
    smooth-shading <off/quick>
    debug <on/all/first/second/scrap-names/station-names/off>
    doc-author <string>
    doc-keywords <string>
    doc-subject <string>
    doc-title <string>
    code <metapost/tex-map/tex-atlas>
    endcode
endlayout|
\endsyntax

\arguments
  |<id>| = layout identifier (to be used in the |export| command)
\endarguments

\comopt
  * |copy <source layout id>| = set properties here that are not
    modified based on the given source layout.

  {\it map presentation-related:}\Nobreak

  * |scale <picture length> <real length>| = set scale of
    output map or map atlas (M, A; default: |1 200|)
  * |base-scale <picture length> <real length>| = if set, Therion will
    optically scale the map by a (|scale/base-scale|) factor.
    This has the same effect as if the map printed in |base-scale| would be
    photo-reduced to the |scale|. (M, A)
  * |rotate <value>| = rotates the map (M, A; default: |0|)
  * |units <metric/imperial>| = set output units (M, A; default: |metric|)
  * |symbol-set <symbol-set>| = use |symbol-set| for all map symbols,
    if available. Be aware, that symbol set name is case sensitive. (M, A)

    Therion uses following predefined symbol sets:\par
    UIS (International Union of Speleology)\par
    ASF (Australian Speleological Federation)\par
    AUT (Austrian Speleological Association)\par
    BCRA (British Cave Research Association )\par
    NSS (National Speleological Society/USA)\par
    NZSS (New Zealand Symbol Set)\NEW{5.4}\par
    SBE (Brazilian Speleological Society -{\it Sociedade Brasileira de Espeleologia})\NEW{5.4.5}\par
    SKBB (Speleoklub Bansk\'a Bystrica)

  * {\rightskip 0cm minus 4pt
    |symbol-assign <point/line/area/group/special> <symbol-type> <symbol-set>| =
    display a
    particular symbol in the given symbol-set. This option overrides
    |symbol-set| option.\par}

    If the symbol has a subtype, |<symbol-type>| argument may have one
    of the following forms: |type:subtype| or simply |type|, which
    assigns new symbol set to all subtypes of a given symbol.

    Following symbols may not be used with this option: point {\it section}
    (which isn't rendered at all) and all point and line labels ({\it label},
    {\it remark}, {\it altitude}, {\it height}, {\it passage-height},
    {\it station-name}, {\it date}). See the chapter
    {\it Changing layout/Customizing text labels} for details how to change
    labels' appearance. (M, A)

    Group may be one of the following: all, centerline, sections, \NEW{5.3}water,
    speleothems, passage-fills, \NEW{5.4}ice, sediments, equipment.

    There are two special symbols: north-arrow, scale-bar.

  * |symbol-hide <point/line/area/group/special>  <symbol-type>| = don't display
    particular symbol or group of symbols.

    You may use |group cave-centerline|, |group surface-centerline|,
    |point cave-station|, |point surface-station|
    and \NEW{5.4}|group text|
    in |symbol-hide| and |symbol-show| commands.

    Use |flag:<entrance/continuation/sink/spring/doline/dig>| as a
    |<symbol-type>| to hide stations with particular flags
    (e.g. |symbol-hide point flag:entrance|).

    May be combined with |symbol-show|.(M, A)
  * |symbol-show <point/line/area/group/special> <symbol-type>| = display particular
    symbol or group of symbols. May be combined with |symbol-hide|. (M, A)

  * \NEW{5.3}|symbol-colo[u]r <point/line/area/group/special> <symbol-type> <colour>| =
    change colour of particular symbol or group of symbols.\[Note: colour change
    currently applies to pattern fills only if (1) output format is PDF and
    (2)~\MP\ version is at least 1.000] (M, A)

  * \NEW{5.4.1}|min-symbol-scale <scale>| =
    define minimal |<scale>|, from which points and lines are displayed on the map. E.g. for
    |min-symbol-scale M|, no points or lines scaled |S| and |XS| will be shown on the map.
    |<scale>| has the same format, as |scale| option for points and lines.

  * \NEW{5.4.1}|fonts-setup <tinysize> <smallsize> <normalsize> <largesize> <hugesize>| =
    specify size of the text in points.
    |<normalsize>| applies to point label, |<smallsize>| applies to
    remark and all other point labels. Each of them may apply to line label
    according to its |-size| option.

    The defaults are
    |8 10 12 16 24| for scales upto 1:100;
    |7 8 10 14 20| for scales upto 1:200;
    |6 7 8 10 14| for scales upto 1:500 and
    |5 6 7 8 10| for scales smaller than 1:500.


\iffulloutput
\midinsert
%  \ifx\pdfoutput\undefined\else
%    \pdfximage {pic/page.pdf}%
%  \fi
%  \vbox to 482bp{\centerline{\hbox to 400bp{%
%    \ifx\pdfoutput\undefined
%      \epsfbox{mp/page.1}%
%    \else
%      \rlap{\pdfrefximage\pdflastximage}%
%      \convertMPtoPDF{mp/page.1}{1}{1}
%    \fi
%    \hss}}\vss
%  }
  \vbox to 482bp{\centerline{\hbox to 400bp{%
      \rlap{\pic{page.pdf}}%
      \MPpic{page.1}
    \hss}}\vss
  }
\endinsert
\fi

  {\it page layout related:}\Nobreak

  * |size <width> <height> <units>| = set map size in the atlas mode. If
    not specified, it will be calculated from |page-setup| and |overlap|.
    In map mode applies iff |page-grid| is |on| (M, A; default: |18 22.2 cm|)
  * |overlap <value> <units>| = set overlap size in paper units in the atlas
    mode or map margin in the map mode (M, A; default: |1 cm|)
  * |page-setup <dimensions> <units>| = set page dimensions in
    this order: paper-width, paper-height, page-width, page-height,
    left-margin and top-margin. If not specified, it will be computed
    from |size| and |overlap| (A; default: |21 29.7 20 28.7 0.5 0.5 cm|)
  * |page-numbers <on/off>| = turn on/off page numbering (A; default: |true|)
  * |exclude-pages <on/off> <list>| = exclude specified pages from cave
    atlas. The list may contain page numbers separated by a comma or dash
    (for intervals) e.g.~|2,4-7,9,23| means, that pages 2, 4, 5, 6, 7, 9 and 23
    should be omitted. Only the map pages should be counted. (Set |own-pages 0|
    and |title-pages off| to get the correct page numbers to be excluded.)
    Changes of |own-pages| or |title-pages| options don't affect page
    excluding. (A)
  * |title-pages <on/off>| = turn on/off title pages before each atlas chapter
    (A; default: |off|)
  * |nav-factor <factor>| = set atlas navigator zoom factor (A; default: |30|)
  * |nav-size <x-size> <y-size>| = set number of atlas pages in
    both directions of navigator (A; default: |2 2|)
  * |transparency <on/off>| = set transparency for the passages (underlying
    passages are also visible) (M, A; default: |on|)
  * |opacity <value>| = set opacity value (used if |transparency| is |on|).
    Value range is 0--100. (M, A; default: |70|)
  * |surface-opacity <value>| = set the surface bitmap opacity (used if
    |transparency| is |on|).  Value range is 0--100. (M, A; default: |70|)
  * |surface <top/bottom/off>| = set the position of the surface bitmap
    above/below the map. (M, A; default: |off|)
  * |sketches <on/off>| = turn on/off displaying of morphed sketch bitmaps.
    (M, A; default: |off|)
  * |layers <on/off>| = enable/disable PDF~1.5 layers (M, A; default: |on|)
  * |grid <off/bottom/top>| = enable/disable grid (optionally coordinates'
    values may be also displayed)
    (M, A; default: |off|)
  * |cs <coordinate system>| = coordinate system for |origin| and
    |grid-origin|
  * |north <true/grid>| = specify default orientation of the map.
    By default, true (astronomical) north is used. It is ignored when used with
    local coordinate system.
  * |grid-origin <x> <y> <x> <units>| = set coordinates of grid
    origin (M, A)
  * |grid-size <x> <y> <z> <units>| = set grid size in real
    units (M, A; default is equal to scalebar size)
  * |grid-coords <off/border/all>| = specify where to label grid with
    coordinates. (M, A; default: |off|)
  * |origin <x> <y> <z> <units>| = set origin of atlas pages (M, A)
  * |origin-label <x-label> <y-label>| = set label for atlas page
    which has the lower left corner at the given origin coordinates.
    May be either a number or a character. (M, A; default: |0 0|)
  * |own-pages <number>| = set number of own pages added before
    the first page of automatically generated pages in atlas mode
    (currently required for correct page numbering) (A; default: |0|)
  * |page-grid <on/off>| = show pages key plan (M; default: |off|)

  {\it map legend related:}\Nobreak

  * |map-header <x> <y> <off/n/s/e/w/ne/nw/se/sw/center>| =
    print map header at location specified by |<x> <y>|.
    Predefined map header contains some basic information about
    cave: name, scale, north arrow, list of surveyors etc. It is fully
    customizable (see the chapter {\it Changing layout} for details).
    |<x>| is easting (left-right on page). |<y>| is northing (up/down page).
    Ranges for |<x>| and |<y>| are -100--200. Lower-left corner of the map
    is |0 0|,
    upper-right corner is |100 100|. The header is aligned with the specified
    corner or side to this anchor point.
    (M; default: |0 100 nw|)
  * |map-header-bg <on/off>| = when on, background of map
     header is filled with background color (e.g.~to hide map grid).
     (M; default: off)
  * |map-image <x> <y> <n/s/e/w/ne/nw/se/sw/center> <filename>| =
    include image\[Note that you can include PDF too, which may be used to
    combine plan and extended elevation into one nice looking PDF file.] specified
    by |<filename>| into map at location specified by |<x> <y>|. For coordinates
    and alignment details, see |map-header| specification.
  * |legend-width <n> <units>| = legend width (M, A; default: |14 cm|)
  * |legend <on/off/all>| = display list of used map symbols in the map header.
    If set to |all|, all symbols from the current symbol set are displayed.
    (M, A; default: |off|)
  * |colo[u]r-legend <on/off>| = turn on/off legend of map-fg colours
    when map-fg is set to altitude, scrap or map (M, A)
  * |legend-columns <number>| = adjusts the number of legend columns
    (M, A; default: |2|)
  * |map-comment <string>| = optional comment displayed at the map header (M)
  * |statistics <explo/topo/carto/copyright all/off/number>| or
  * |statistics <explo/topo-length on/hide/off>| = display some basic
    statistics; if set to off, team members are sorted \NEW{5.4}alphabetically;
    otherwise according to their contribution to exploration and surveying
    (M, A; default: |hide|)
  * |statistics <carto/copyright-count on/hide/off>| = \NEW{5.6} display number of scraps
    for given author/copyright string; if set to off, items are sorted 
    alphabetically; otherwise by number of scraps
    (M, A; default: |hide|)
  * |scale-bar <length> <units>| = set the length of the scale-bar (M, A)
  * |language <xx[_YY]>| = set output language.
    Available languages are listed on the copyright page.
    See the {\it Appendix} if
    you want to  add or customize translations. (M, A)
  * |colo[u]r-model <cmyk/rgb/grayscale>| = select the output colour model (M, A;
    default: |cmyk|).

    The CMYK colour model is intended for printing. Black overprint is used
    for the best appearance of black lines and texts. The hue of other colours may
    vary depending on the type or settings of the RIP or printer driver.

    The RGB colour model is intended for screen or multimedia projector presentation.
    If used for printing, the result will depend on the colour management
    settings and also on particular colour, as not all RGB colours can be represented
    on printer.

    The grayscale colour model is meant for printing on B\&W printers.

    If input colours (|lookup|, |colour| in the |layout|) do not match the output
    colour model (e.g. only RGB is defined in the |lookup| and CMYK selected as the |colour-model|),
    they will be converted to the output model.
  * |colo[u]r-profile <cmyk/rgb/grayscale> <filename>| = assign an ICC profile to colours
    specified in the given colour model. ICC profiles are applied only to drawings,
    not to embedded raster images like sketches and surface bitmaps. (M, A)
  * |colo[u]r <item> <colour>| = customize colour for special map
    items (|map-fg|, |map-bg|, |preview-above|, |preview-below|, |labels|).
    Colour range is 0--100 for grayscale, [0--100 0--100 0--100] triplet
    for RGB colours and [0--100 0--100 0--100 0--100] quadruplet
    for CMYK colours.\[Note, that not all colour combinations are valid;
    e.g. no printer will print CMYK [100 100 100 100\char93.
    The maximum ink coverage or limit ($C+M+Y+K$) may be around 240 or 300,
    depending on the printer.]

    For |map-fg|, you can use |altitude|, |scrap| or |map| as colours.
    In this case the map is coloured according to altitude, scraps or maps.

    For |map-bg|, you can use |transparent| to omit page background
    completely.

    For |labels|, you can switch colour |on/off|. If |on|, labels are coloured
    using the colour of associated scrap.

  * \NEW{5.6}|smooth-shading <off/quick>| =
    set the mode of smooth scrap backgroud shading. By default, altidute and depth
    colour is interpolated across the scrap the |quick| way. Some issues are present if
    transparent symbol colours are used. More precise modes should be added in the future.
    If |off|, scrap is filled with single colour.    
  * |debug <on/all/first/second/scrap-names/station-names/off>| =
    draw scrap in different stages
    of transformation in different colours to see how Therion distorts
    map data. See the description of |scrap| command for details.
    The points with distance changed most during transformation are displayed
    orange. If |scrap-names| is specified, scrap names are shown for each scrap,
    |station-names| displays name of each survey station.
  * |survey-level <N/all>| = |N| is the number of survey levels displayed
                             next to the station name (M, A; default: |0|).

  {\it PDF related:}\Nobreak

  * |doc-author <string>| = set document author (M, A)
  * |doc-keywords <string>| = set document keywords (M, A)
  * |doc-subject <string>| = set document subject (M, A)
  * |doc-title <string>| = set document title (M, A)

  {\it customization:}\Nobreak

  * |code <metapost/tex-map/tex-atlas>| = Add/redefine \TeX\ and \MP\
    macros here. This allows user to configure various things
    (like user defined symbols, map and atlas layout at one place \&c.)
    See the chapter {\it Changing layout} for details.
  * |endcode| = should end the TeX\ and \MP\ sections
\endcomopt


\subsubchapter `lookup'.

\description
  \NEW{5.4.4} Allows to define lookup maps. They are used for defining custom coloring
  of maps.\[see |colour| command.]
\enddescription

\syntax |lookup <type>[:<index>] [-title "custom title text"]
  <parameter(s)> [colour] ["text in legend"]
  ...
endlookup|
\endsyntax

\arguments
  * |<type>| = May be |altitude|, |explo-date|, |topo-date|, |map| or |scrap|.

  * |<index>| = the index is a user defined string that allows to distinguish
        different lookup maps for the same |type|.

  * |<parameter>| = the thing, you want to set the color for, e.g. an altitude
        with type |altitude|. May be distinct values or (if appropriate) bands.

  * |[colour]| = grayscale value, RGB triplet, CMYK quadruplet or a combination
    of eight values (RGB, grayscale, CMYK).
        Use empty brackets (|[]|) if you want Therion to
        use it’s default colour palette together with a text label.

  * |[text in legend]| = some optional text to show in the legend.
\endarguments

\example Banded altitudes. It should generate red-blue scale with desired values.

\ \ \ \ \ |lookup altitude -title "Altitude legend"
      700 [100 0 0] "700 m a.s.l."
      680
      660
      640
      620
      600 [0 0 100] "below 600 m"
    endlookup|
\endexample

\example Using banded altitude specification (with index ”banded”). Notice that if the
text is omitted, then the default will display the range, ie 1600 m - 1500 m

\ \ \ \ \ |lookup altitude:banded
      [1500 1600] []          # <- displays "1600 m - 1500 m"
      [1800 1900] [] "cave floor 2"
    endlookup|
\endexample

\subsubchapter `setup3d'.

\syntax
  \list
    * |setup3d <value>|
  \endlist
\endsyntax

\NEW{5.3}Temporary hack to set sampling distance in meters when generating
piecewise linear 3d model from passage walls made of B\'ezier curves.


\subsubchapter `sketch-colors'.

\syntax
  \list
    * |sketch-colors <number-of-colors>|
  \endlist
\endsyntax

\NEW{5.4} This option can be used to reduce size of sketch bitmap
  images in maps.


\subsubchapter `export'.

\description
  Exports selected surveys or maps.
\enddescription

\syntax
  \list
    * |export <type> [OPTIONS]|
  \endlist
\endsyntax

\arguments
  * |<type>| = The following export types are supported:

    |model| = 3D model of the cave

    |map| = one page 2D map

    |atlas| = 2D atlas in more pages

    |cave-list| = summary table of caves

    |survey-list| = summary table of surveys

    |continuation-list| = list of possible continuations

    |database| = SQL database with centreline
\endarguments

\penalty0

\options
  {\it common:}\Nobreak
  * |encoding/enc <encoding>| = set output encoding
  * |output/o <file>| = set output file name. If no file name is
    given the prefix ``|cave.|'' is used with an extension corresponding to
    output format.

    If the output filename is given and no output format is specified,
    the format is determined from the filename extension.


  {\it model:}\Nobreak

  * |format/fmt <format>| = set model output format. Currently the following
    output formats are supported: |loch| (native format; default),
    |compass| (plt file), |survex| (3d file), |dxf|,
    |esri| (3d shapefiles), |vrml|, |3dmf| and |kml| (Google Earth).
  * |enable <walls/[cave/surface-]centerline/splay-shots/surface/all>| and
  * |disable <walls/[cave/surface-]centerline/splay-shots/surface/all>| =
     \hfil\break
     selects which features to export, if the format supports it. Surface
     is currently exported in |therion| format only.
  * |wall-source <maps/centerline/all/splays>| = set source data for passage
    wall modeling.

  {\it map/atlas:}\Nobreak

  * |format/fmt <format>| = set map format. Currently |pdf|, |svg|,
    |xhtml|\[SVG embedded in XHTML file which contains also legend],
    |survex|, |dxf|, |esri|\[ESRI shapefiles. Multiple files are written to a
    directory with the specified filename.], |kml| (Google Earth),
    |xvi|\[Xtherion vector image. XVI images may be used in
    xtherion to draw in-scale maps. Scale (100\,DPI image resolution is
    assumed) and grid-size from layout are used in export.] and
    \NEW{5.3}|bbox|\[Text file containing geographic coordinates of
    lower-left and upper-right corners of the map area.]
    for map; |pdf| for atlas are supported.
  * |projection <id>| = unique identifier that specifies the map projection type.
    (See the |scrap| command for details.)

    If there is no map defined, all scraps in the given projection are
    exported.

    If there are no scraps with
    the specified projection then Therion will display centreline from selected
    surveys.
  * |layout <id>| = use predefined map or atlas layout.
  * |layout-xxx| = where |xxx| stands for other layout options. Using this
    you can change some layout properties directly within the export command.
  * |encoding/enc <encoding>| = set output encoding

  {\it common for lists:}\Nobreak

  * |format/fmt <format>| = set continuation output format. Currently the following
    output formats are supported: |html| (default), |txt|,
    \NEW{5.4}|kml|\[For cave-list and continuation-list.] and |dbf|.

  {\it continuation-list:}\Nobreak

  * |attributes <(on)/off>| = set whether to export user defined attributes
    in continuation list table.
  * \NEW{5.3}|filter <(on)/off>| = set whether continuations without comment/text
    should be filtered out.

  {\it cave-list:}\Nobreak

  * |location <on/(off)>| = set whether to export coordinates of cave entrances in
    the table.
  * \NEW{5.3}|surveys (on)/off| = exports raw list of caves when set |off|. Otherwise
    survey structure with aggregated statistics is also displayed.

  {\it database:}\Nobreak

  * |format/fmt <format>| = currently |sql| and |csv|
  * |encoding/enc <encoding>| = set output encoding
\endoptions

{\it File formats summary:}
\nobreak\medskip
\bgroup
\leavevmode\kern1em\vbox{\advance\hsize-1em\halign{#\hfil\quad&#\hfil\cr
{\it export type}&{\it available formats}\cr\noalign{\smallskip\hrule\smallskip}
model & loch, dxf, esri, compass, survex, vrml, 3dmf, kml\cr
map  & pdf, svg, xhtml, dxf, esri, survex, xvi, kml, \NEW{5.3}bbox\cr
atlas & pdf\cr
database & sql, \NEW{5.4}csv\cr
lists & html, txt, kml, dbf\cr}}\egroup


\subchapter Running Therion.

Now, after mastering data and configuration files, we're ready to run Therion.
Usually this is done from the command line in the data directory by typing

|therion|

The full syntax is

|therion [-q] [-L] [-l <log-file>]
        [-s <source-file>] [-p <search-path>]
        [-b/--bezier]
        [-d] [-x] [--use-extern-libs] [<cfg-file>]|

or

|therion [-h/--help]
        [-v/--version]
        [--print-encodings]
        [--print-environment]
        [--print-init-file]
        [--print-library-src]
        [--print-symbols]
        [--print-tex-encodings]
        [--print-xtherion-src]
        [--reproducible-output]
        [--generate-output-crc]
        [--verify-output-crc]|

\penalty-200
\arguments
  |<cfg-file>|
  Therion takes only one optional argument: the name of a configuration
  file. If no name is specified |thconfig| in the current directory is used.
  If there is no |thconfig| file (e.g.~current directory is not a data
  directory), Therion exits with an error message.
\endarguments

\options
* |-d| =
  Turn on debugging mode. The current implementation creates a
  temporary directory named |thTMPDIR| (in your system temporary
  directory) and does not delete any temporary files.

%* |-g| =
%  Using this option you can generate a new configuration file.
%  If |cfg-file| is not specified therion will use the |thconfig|
%  file. If the destination file exists, it'll be overwritten.
%  Generate a new configuration file. This will be the given
%  |<cfg-file>| if specified, or |thconfig| in the current directory if not.
%  If the file already exists, it will be overwritten.
%
* |-h, --help| =
        Display short help.

%* |-i| =
%        Ignore comments when writing |-g| or |-u| configuration file.
%
* |-L| =
        Do not create a log-file. Normally therion writes all the messages
        into a therion.log file in the current directory.

* |-l <log-file>| =
        Change the name of the log file.

* |-p <search-path>| =
        This option is used to set the search path (or list of
	colon-separated paths) which therion uses to find its source
        files (if it doesn't find them in the working directory).

* |-q| =
        Run therion in quiet mode. It will print only warning
        and error messages to STDERR.

* |--print-encodings| =
        Print a list of all supported input encodings.

* |--print-tex-encodings| =
        Print a list of all supported encodings for PDF output.

* |--print-init-file| =
        Print a default initialization file. For more details
        see the {\it Initialization} section in the {\it Appendix}.

* |--print-environment| =
        Print environment settings for therion.

* |--print-symbols| =
        Print a list of all therion supported map symbols in
        |symbols.xhtml| file.

* |--reproducible-output| =
        Create reproducible PDF and SVG files.
        No volatile information is included (e.g. creation date or software
        version is omitted from the metadata). \TeX\ variables |\thversion|
        and |\currentdate| are set to predefined constatnts.
        This option enforces the use of Therion loop closure.

        To get the same output files on different platforms, it's
        necessary to install the same set of basic fonts used by
        Therion\[CM, CS and/or CMCYR font families; the `.pfb' font files have to be exactly
        the same on all platforms as they are embedded into PDF files.]
        and use reasonably recent \TeX\ distribution on all platforms.

* |--generate-output-crc| =
        Generate a `.crc' file with CRC32 checksum for each output file.
        Implies |--reproducible-output|.

* |--verify-output-crc| =
        Verify that the output file has not been changed.
        Generate reproducible output, calculate CRC32 checksum and check
        it against one saved in `.crc' file.

* |-s <source-file>| =
        Set the name of the source file.

%* |-u| =
%        Upgrade the configuration file.
%
* |--use-extern-libs| =
  Don't copy \TeX\ and \MP\ macros to working directory. \TeX\ and \MP\
  should search for them on their own. Use with caution.

* |-v, --version| =
  Display version information.

* |-x| =
  Generate file `.xtherion.dat' with additional information for XTherion.
\endoptions

\subsubchapter XTherion---compiler.

XTherion makes it easier to run Therion especially on systems without a command
line prompt. Compiler window is the default window of XTherion. To run Therion
it's enough to open a configuration file and press `F9' or `Compile' button.

XTherion displays messages from Therion in the lower part of the screen.
Each error message is highlighted and is hyperlinked to the source file
where the error occurred.

After a first run there are activated additional menus {\it Survey
structure} and {\it Map structure}. User may comfortably select a survey or map
for export by double clicking on some of the items in the tree. Simple click in
the {\it Survey structure} tree displays some basic information about the
survey in the {\it Survey info} menu.

\endinput
