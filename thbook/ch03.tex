\chapter Processing data.

Besides data files, which contain survey data, Therion uses a configuration file, 
which contains instructions on how the data should be presented.

\subchapter Configuration file.

The configuration filename can be given as an
argument to therion. By default Therion searches for file named |thconfig|
in the current working directory. It is read like any other therion file 
(i.e.~one command per line; empty lines or lines starting with `|#|' are ignored;
lines ended with a backslash continue on the next line.) A list of currently 
supported commands follow.

\subsubchapter `encoding'.

Works like the |encoding| command in data files---specifies character sets.

\subsubchapter `input'.

Works like |input| command in data files---includes other files.

\subsubchapter `source'.

\description
   Specifies which source (data) files Therion should read. 
   You can specify several files here; one per line. 
   You can also specify them using the |-s| command line option (see below).
\enddescription

\syntax
  |source <file-name>|
\endsyntax

\arguments
*  |<file-name>| 
\endarguments



\subsubchapter `select'.

\description
  selects objects (surveys and maps) for export. By default, all survey objects 
  are selected. If there is no map selected, all maps belonging to
  selected surveys are selected by default for map export.
  If there are no such maps, centreline from the selected surveys is 
  exported in the map.
\enddescription

\syntax
  |select <object> [OPTIONS]|
\endsyntax

\arguments
*  |<object>| = any survey or map, identified by its ID.
\endarguments

\options
  * |recursive <on/off>| = valid only when a survey is selected. If set
    on (by default) all subsurveys of the given survey are recursively
    selected/unselected.
  
  * |map-level <number>| = valid only when a map is selected. Determines
    the level at which map expansion for atlas export is stopped. 
    By default 0 is used; if ``basic'' is specified, 
    expansion is done up to the basic maps.
    {\it Note:} Map previews are displayed only as specified in maps in the 
    current |map-level|.

  * |chapter-level <number>| = valid only when a map is selected. Determines
    the level at which chapter expansion for atlas export is stopped. 
    By default 0 is used, if ``|-|'' or ``|.|'' is specified, no chapter is
    exported for this map. If |title-pages| option in |layout| is on,
    each chapter starts with a title page.
\endoptions



\subsubchapter `unselect'.

\description
  Unselects objects from export.
\enddescription

\syntax
  |unselect <object> [OPTIONS]|
\endsyntax

\arguments
  The same as the |select| command.
\endarguments

\options
  The same as the |select| command. 
\endoptions


\subsubchapter `layout'.

\description
  Specifies layout for 2D maps. Settings which apply to atlas mode
  are marked `A'; map mode `M'.
\enddescription

\syntax
|layout <id> [OPTIONS]
    copy <source layout id>
    scale <picture length> <real length>
    base-scale <picture length> <real length>
    units <metric/imperial>
    rotate <number>
    symbol-set <symbol-set>
    symbol-assign <point/line/area/group/special> <symbol-type> \ 
                                                  <symbol-set>
    symbol-hide <point/line/area/group/special> <symbol-type>
    symbol-show <point/line/area/group/special> <symbol-type>
    size <width> <height> <units>
    overlap <value> <units>
    page-setup <dimensions> <units>
    page-numbers <on/off>
    exclude-pages <on/off> <list>
    title-pages <on/off>
    nav-factor <factor>
    nav-size <x-size> <y-size>
    transparency <on/off>
    opacity <value>
    surface <top/bottom/off>
    surface-opacity <value>
    layers <on/off>
    grid <off/top/bottom>
    grid-origin <x> <y> <x> <units>
    grid-size <x> <y> <z> <units>
    origin <x> <y> <z> <units>
    origin-label <x-label> <y-label>
    own-pages <number>
    page-grid <on/off>
    legend <on/off/all>
    legend-columns <number>
    legend-width <n> <units>
    map-comment <string>
    map-header <x> <y> <off/n/s/e/w/ne/nw/se/sw/center>
    map-header-bg <on/off>
    statistics <explo/topo/carto/copyright all/off/number>
               <explo/topo-length on/off>
    scale-bar <length> <units>
    language <xx[_YY]>
    colour/color <item> <colour>
    debug <on/all/first/second/scrap-names/off>
    doc-author <string>
    doc-keywords <string>
    doc-subject <string>
    doc-title <string>
    code <metapost/tex-map/tex-atlas>
    endcode
endlayout|
\endsyntax

\arguments
  |<id>| = layout identifier (to be used in the |export| command)
\endarguments

\comopt 
  * |copy <source layout id>| = set properties here that are not
    modified based on the given source layout.
  
  {\it map presentation-related:}\Nobreak

  * |scale <picture length> <real length>| = set scale of
    output map or map atlas (M, A; default: |1 200|)
  * |base-scale <picture length> <real length>| = if set, Therion will
    optically scale the map by a (|scale/base-scale|) factor.
    This has the same effect as if the map printed in |base-scale| would be
    photoreduced to the |scale|. (M, A)
  * |rotate <value>| = rotates the map (M, A; default: |0|)
  * |units <metric/imperial>| = set output units (M, A; default: |imperial|)
  * |symbol-set <symbol-set>| = use |symbol-set| for all map symbols,
    if available (M, A)
    
    Therion uses following predefined symbol sets:\par
    UIS (International Union of Speleology)\par
    ASF (Australian Speleological Federation)\par
    CCNP (Carlsbad Caverns National Park)\par
    SKBB (Speleoklub Bansk\'a Bystrica)
    
  * {\rightskip 0cm minus 4pt
    |symbol-assign <point/line/area/group/special> <symbol-type> <symbol-set>| = 
    display a
    particular symbol in the given symbol-set. This option overrides 
    |symbol-set| option.\par}
    
    If the symbol has a subtype, |<symbol-type>| argument may have one
    of the following forms: |type:subtype| or simply |type|, which
    assigns new symbol set to all subtypes of a given symbol.
    
    Following symbols may not be used with this option: point {\it section} 
    (which isn't rendered at all) and all point and line labels ({\it label}, 
    {\it remark}, {\it altitude}, {\it height}, {\it passage-height}, 
    {\it station-name}, {\it date}). See the chapter
    {\it Changing layout/Customizing text labels} for details how to change
    labels' appearance. (M, A)

    Group may be one of the following: all, centerline, sections.
    
    There are two special symbols: north-arrow, scale-bar.

  * |symbol-hide <point/line/area/group/special>  <symbol-type>| = don't display 
    particular symbol or group of symbols. 
    You may use |group cave-centerline|, |group surface-centerline|, 
    |point cave-station| and |point surface-station|
    in |symbol-hide| and |symbol-show| commands.
    May be combined with |symbol-show|.(M, A)
  * |symbol-show <point/line/area/group/special> <symbol-type>| = display particular 
    symbol or group of symbols. May be combined with |symbol-hide|. (M, A)


  {\it page layout related:}\Nobreak

  * |size <width> <height> <units>| = set map size in the atlas mode.
    In map mode applies iff |page-grid| is |on| (M, A; default: |18 22.2 cm|)
  * |overlap <value> <units>| = set overlap size in paper units in the atlas 
    mode or map margin in the map mode (M, A; default: |1 cm|)
  * |page-setup <dimensions> <units>| = set page dimensions in
    this order: paper-width, paper-height, page-width, page-height,
    left-margin and top-margin (A; default: |21 29.7 20 28.7 0.5 0.5 cm|)
  * |page-numbers <on/off>| = turn on/off page numbering (A; default: |true|)
  * |exclude-pages <on/off> <list>| = exclude specified pages from cave
    atlas. The list may contain page numbers separated by a comma or dash 
    (for intervals) e.g.~|2,4-7,9,23| means, that pages 2, 4, 5, 6, 7, 9 and 23 
    should be omitted. Only the map pages should be counted. (Set |own-pages 0| 
    and |title-pages off| to get the correct page numbers to be excluded.) 
    Changes of |own-pages| or |title-pages| options don't affect page 
    excluding. (A)
  * |title-pages <on/off>| = turn on/off title pages before each atlas chapter
    (A; default: |off|)
  * |nav-factor <factor>| = set atlas navigator zoom factor (A; default: |30|)
  * |nav-size <x-size> <y-size>| = set number of atlas pages in 
    both directions of navigator (A; default: |2 2|)
  * |transparency <on/off>| = set transparency for the passages (underlying 
    passages are also visible) (M, A; default: |on|)
  * |opacity <value>| = set opacity value (used if |transparency| is |on|). 
    Value range is 0--100. (M, A; default: |70|)
  * |surface-opacity <value>| = set the surface bitmap opacity (used if 
    |transparency| is |on|).  Value range is 0--100. (M, A; default: |70|)
  * |surface <top/bottom/off>| = set the position of the surface bitmap
    above/below the map. (M, A; default: |off|)
  * |layers <on/off>| = enable/disable PDF~1.5 layers (M, A; default: |on|)
  * |grid <off/bottom/top>| = enable/disable grid (M, A; default: |off|)
  * |grid-origin <x> <y> <x> <units>| = set coordinates of grid
    origin (M, A) [Grid is currently not supported.]
  * |grid-size <x> <y> <z> <units>| = set grid size in real
    units (M, A; default \hbox{|10 m|})
  * |origin <x> <y> <z> <units>| = set origin of atlas pages (M, A)
  * |origin-label <x-label> <y-label>| = set label for atlas page
    which has the lower left corner at the given origin coordinates
    (M, A; default: |0 0|)
  * |own-pages <number>| = set number of own pages added before 
    the first page of automatically generated pages in atlas mode
    (currently required for correct page numbering) (A; default: |0|)
  * |page-grid <on/off>| = show pages key plan (M; default: |off|)

  {\it map legend related:}\Nobreak

  * |map-header <x> <y> <off/n/s/e/w/ne/nw/se/sw/center>| = 
    print map header at location specified by |<x> <y>|. 
    Predefined map header contains some basic informations about
    cave: name, scale, north arrow, list of surveyors etc. It is fully 
    customizable (see the chapter {\it Changing layout} for details).
    Ranges for |<x>| and |<y>| are -100--200. Lower-left corner of the map 
    is |0 0|, 
    upper-right corner is |100 100|. The header is aligned with the specified 
    corner or side to this anchor point.
    (M; default: |0 100 nw|)
  * |map-header-bg <on/off>| = when on, background of map 
     header is filled with background color (e.g.~to hide map grid). 
     (M; default: off)
  * |legend-width <n> <units>| = legend width (M, A; default: |14 cm|)
  * |legend <on/off/all>| = display list of used map symbols in the map header. 
    If set to |all|, all symbols from the current symbol set are displayed. 
    (M, A; default: |off|)
  * |colo[u]r-legend <on/off>| = turn on/off colour legend (M, A)
  * |legend-columns <number>| = adjusts the number of legend columns 
    (M, A; default: |2|)
  * |map-comment <string>| = optional comment displayed at the map header (M)
  * |statistics <explo/topo/carto/copyright all/off/number>| or 
  * |statistics <explo/topo-length on/off>| = display some basic 
    statistics (M, A; default: |off|)
  * |scale-bar <length> <units>| = set the length of the scale-bar (M, A)
  * |language <xx[_YY]>| = set output language. 
    Available languages are listed on impressum page.
    See the {\it Appendix} if 
    you want to  add or customize translations. (M, A)
  * |colo[u]r <item> <colour>| = customize colour for special map
    items (map-fg, map-bg, preview-above, preview-below). 
    Colour range is 0--100 for grayscale, [0--100 0--100 0--100] triplet
    for RGB colours. Special colour specification for |map-fg| is
    |altitude| or |map|, in which case the map is coloured according to
    altitude or maps.
  * |debug <on/all/first/second/scrap-names/off>| = draw scrap in different stages
    of transformation in different colours to see how Therion distorts
    map data. See the description of |scrap| command for details.
    The points with distance changed most during transformation are displayed
    orange. If |scrap-names| is specified, scrap names are shown for each scrap.

  {\it PDF related:}\Nobreak

  * |doc-author <string>| = set document author (M, A)
  * |doc-keywords <string>| = set document keywords (M, A)
  * |doc-subject <string>| = set document subject (M, A)
  * |doc-title <string>| = set document title (M, A)

  {\it customization:}\Nobreak

  * |code <metapost/tex-map/tex-atlas>| = Add/redefine \TeX\ and \MP\
    macros here. This allows user to configure various things 
    (like user defined symbols, map and atlas layout at one place \&c.)
    See the chapter {\it Changing layout} for details.
  * |endcode| = should end the TeX\ and \MP\ sections
\endcomopt

\midinsert
  \ifx\pdfoutput\undefined\else
    \pdfximage {pic/page.pdf}%
  \fi
  \vbox to 482bp{\centerline{\hbox to 400bp{%
    \ifx\pdfoutput\undefined
      \epsfbox{mp/page.1}%
    \else
      \rlap{\pdfrefximage\pdflastximage}%
      \convertMPtoPDF{mp/page.1}{1}{1}
    \fi
    \hss}}\vss
  }
\endinsert


\subsubchapter `export'.

\description
  Exports selected surveys or maps. 
\enddescription

\syntax
  \list
    * |export <type> [OPTIONS]| 
  \endlist
\endsyntax

\arguments
  * |<type>| = The following export types are supported:

    |model| = 3D model of the cave

    |map| = one page 2D map

    |atlas| = 2D atlas in more pages

    |database| = SQL database with centreline
\endarguments

\penalty0

\options
  {\it common:}\Nobreak

  * |output/o <file>| = set output file name. If no file name is
    given the prefix ``|cave.|'' is used with an extension corresponding to
    output format.
    
    If the output filename is given and no output format is specified,
    the format is determined from the filename extension.

  {\it model:}\Nobreak

  * |format/fmt <format>| = set model output format. Currently the following
    output formats are supported: |therion| (native format; default),
    |compass| (plt file), |survex| (3d file), |dxf|, |vrml| and |3dmf|.
  * |enable <walls/[cave/surface-]centerline/surface/all>| and
  * |disable <walls/[cave/surface-]centerline/surface/all>| = 
     \hfil\break
     selects which features to export, if the format supports it. Surface
     is currently exported in |therion| format only.

  {\it map/atlas:}\Nobreak
  
  * |format/fmt <format>| = set map format. Currently |pdf|, |svg| and 
    |xvi|\[Xtherion vector image. XVI images may be used in
    xtherion to draw in-scale maps. Scale (100\,dpi image resolution is 
    assumed) and grid-size from layout are used in export.]
    for map; |pdf| for atlas are supported.
  * |projection <id>| = unique identifier that specifies the map projection type.
    (See the |scrap| command for details.) 
    
    If there is no map defined, all scraps in the given projection are 
    exported.
    
    If there are no scraps with 
    the specified projection then Therion will display centreline from selected
    surveys.
  * |layout <id>| = use predefined map or atlas layout.
  * |layout-xxx| = where |xxx| stands for other layout options. Using this
    you can change some layout properties directly within the export command.

  {\it database:}\Nobreak
  
  * |format/fmt <format>| = currently only |sql|
  * |encoding/enc <encoding>| = set output encoding
\endoptions

{\it File formats summary:}
\nobreak\medskip
\bgroup
\leavevmode\kern1em\vbox{\halign{#\hfil\quad&#\hfil\cr
{\it export type}&{\it available formats}\cr\noalign{\smallskip\hrule\smallskip}
model & therion, dxf, compass, survex, vrml, 3dmf\cr
map  & pdf, svg, xvi\cr
atlas & pdf\cr
database & sql\cr}}\egroup


\subchapter Running Therion.

Now, after mastering data and configuration files, we're ready to run Therion. 
Usually this is done from the command line in the data directory by typing

|therion|

The full syntax is

|therion [-q] [-L] [-l <log-file>]
        [-s <source-file>] [-p <search-path>]
        [-g/-u] [-i] [-d] [-x] [--use-extern-libs] [<cfg-file>]|

or

|therion [-h/--help]
        [-v/--version]
        [--print-encodings]
        [--print-tex-encodings]
        [--print-init-file]
        [--print-environment]|

\penalty-200
\arguments
  |<cfg-file>| 
  Therion takes only one optional argument: the name of a configuration
  file. If no name is specified |thconfig| in the current directory is used. 
  If there is no |thconfig| file (e.g.~current directory is not a data 
  directory), Therion exits with an error message.
\endarguments

\options
* |-d| =
  Turn on debugging mode. The current implementation creates a 
  temporary directory named |thTMPDIR| (in your system temporary 
  directory) and does not delete any temporary files. 

* |-g| =
%  Using this option you can generate a new configuration file.
%  If |cfg-file| is not specified therion will use the |thconfig|
%  file. If the destination file exists, it'll be overwritten.
  Generate a new configuration file. This will be the given 
  |<cfg-file>| if specified, or |thconfig| in the current directory if not. 
  If the file already exists, it will be overwritten.
        
* |-h, --help| =
        Display short help.

* |-i| =
        Ignore comments when writing |-g| or |-u| configuration file.

* |-L| =
        Do not create a log-file. Normally therion writes all the messages
        into a therion.log file in the current directory.
        
* |-l <log-file>| =
        Change the name of the log file.
        
* |-p <search-path>| =
        This option is used to set the search path (or list of 
	colon-separated paths) which therion uses to find its source
        files (if it doesn't find them in the working directory).

* |-q| =
        Run therion in quiet mode. It will print only warning
        and error messages to STDERR.

* |--print-encodings| =
        Print a list of all supported input encodings.
        
* |--print-tex-encodings| =
        Print a list of all supported encodings for PDF output.
        
* |--print-init-file| =
        Print a default initialization file. For more details
        see the {\it Initialization} section in the {\it Appendix}.
        
* |-s <source-file>| =
        Set the name of the source file.
        
* |-u| =
        Upgrade the configuration file.

* |--use-extern-libs| =
  Don't copy \TeX\ and \MP\ macros to working directory. \TeX\ and \MP\ 
  should search for them on their own. Use with caution.

* |-v, --version| =
  Display version information.        
        
* |-x| =
  Generate file `.xth-thconfig' with additional informations for XTherion.
\endoptions

\subsubchapter XTherion---compiler.

XTherion makes it easier to run Therion especially on systems without a command 
line prompt. Compiler window is the default window of XTherion. To run Therion 
it's enough to open a configuration file and press `F9' or `Compile' button. 

XTherion displays messages from Therion in the lower part of the screen.
Each error message is highlited and is hyperlinked to the source file
where the error occured.

After a first run there are activated additional menus {\it Survey 
structure} and {\it Map structure}. User may comfortably select a survey or map 
for export by double clicking on some of the items in the tree. Simple click in 
the {\it Survey structure} tree displays some basic informations about the 
survey in the {\it Survey info} menu.

\endinput
