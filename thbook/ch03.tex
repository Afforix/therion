\chapter Processing data.

Besides data files, which contain survey data, Therion uses a configuration file, 
which contains instructions on how the data should be presented.

\subchapter Configuration file.

The configuration filename can be given as an
argument to therion. By default Therion searches for file named |thconfig|
in the current working directory. It is read like any other therion file 
(i.e.~one command per line; empty lines or lines starting with `|#|' are ignored;
lines ended with a backslash continue on the next line.) A list of currently 
supported commands follow.

\subsubchapter `encoding'.

Works like the |encoding| command in data files - specifies character sets.

\subsubchapter `input'.

Works like |input| command in data files  - includes other files.

\subsubchapter `source'.

\description
   Specifies which source (data) files Therion should read. 
   You can specify several files here; one per line. 
   You can also specify them using the |-s| command line option (see below).
\enddescription

\syntax
  |source <file-name>|
\endsyntax

\arguments
*  |<file-name>| 
\endarguments



\subsubchapter `select'.

\description
  selects objects for export. By default, all survey objects 
  are selected. 
\enddescription

\syntax
  |select <object> [OPTIONS]|
\endsyntax

\arguments
*  |<object>| = any object from the database, such as a survey or 
   map, identified by its id.
\endarguments

\options
  * |recursive <on/off>| = valid only when a survey is selected. If set
    on (by default) all subsurveys of the given survey are recursively
    selected/unselected.
  
  * |map-level <number>| = valid only when a map is selected. Determines
    the level at which map expansion for atlas export is stopped. 
    By default 0 is used, if ``|-|'' or ``|.|'' is specified, 
    expansion is done up to basic maps.

  * |chapter-level <number>| = valid only when a map is selected. Determines
    the level at which chapter expansion for atlas export is stopped. 
    By default 0 is used, if ``|-|'' or ``|.|'' is specified, no chapter is
    exported for this map.
\endoptions



\subsubchapter `unselect'.

\description
  Unselects objects from export.
\enddescription

\syntax
  |unselect <object> [OPTIONS]|
\endsyntax

\arguments
  The same as the |select| command.
\endarguments

\options
  The same as the |select| command. 
\endoptions


\subsubchapter `layout'.

\description
  Specifies layout for exported 2D maps. Most settings apply 
  to atlas mode.
\enddescription

\syntax
  |layout <id> [OPTIONS]
          copy <source layout id>
          doc-author <string>
          doc-keywords <string>
          doc-subject <string>
          doc-title <string>
          exclude-pages <on/off> <list>
          grid-origin <x> <y> <x> <units>
          grid-size <width> <height> <units>
          nav-factor <factor>
          nav-size <x-size> <y-size>
          opacity <on/off> <value>
          origin <x> <y> <x> <units>
          origin-label <x-label> <y-label>
          overlap <value> <units>
          own-pages <number>
          page-numbers <on/off>
          page-setup <dimensions> <units>
          scale <picture length> <real length>
          size <width> <height> <units>
          title-pages <on/off>
          ... [atlas tex macros] ...
        endlayout|
\endsyntax

\arguments
  |<id>| = layout identifier (to be used in the |export| command)
\endarguments

\options
\endoptions

\comopt 
  * |copy <source layout id>| = set properties here that are not
    modified based on the given source layout.
  * |doc-author <string>| = set document author
  * |doc-keywords <string>| = set document keywords
  * |doc-subject <string>| = set document subject
  * |doc-title <string>| = set document title
  * |exclude-pages <on/off> <list>| = exclude specified pages from cave
    atlas. The list may contain page numbers separated by a comma or dash (for intervals) 
    e.g.~|2,4-7,9,23| means, that pages 2, 4, 5, 6, 7, 9 and 23 should be 
    omited. (Note: this applies only to the map and title pages. Set 
    |own-pages| to 0 to get the correct page numbers to be excluded.) 
    If you change |title-pages| setting, you have to
    adjust this list accordingly. (In the future there will be another form
    of the specification of which pages to omit, which won't require
    this redefinition.)
  * |grid-origin <x> <y> <x> <units>| = set coordinates of grid
    origin
  * |grid-size <width> <height> <units>| = set grid size in real
    units
  * |nav-factor <factor>| = set atlas navigator zoom factor
  * |nav-size <x-size> <y-size>| = set number of atlas pages in
    both directions of navigator
  * |opacity <value>| = set opacity value (used if |transparency| is on). 
    Value range is 0--100.
  * |origin <x> <y> <x> <units>| = set origin of atlas pages
  * |origin-label <x-label> <y-label>| = set label for atlas page
    which has the lower left corner at the given origin coordinates
  * |overlap <value> <units>| = set overlap size in paper units
  * |own-pages <number>| = set number of own pages added before
    the first page of automatically generated pages in atlas mode
    (currently required for correct page numbering)
  * |page-numbers <on/off>| = turn on/off page numbering in atlas mode
  * |page-setup <dimensions> <units>| = set page dimensions in
    this order: paper-width, paper-height, page-width, page-height,
    left-margin and top-margin.
  * |scale <picture length> <real length>| = set scale of
    output map or map atlas
  * |size <width> <height> <units>| = set map size in the atlas mode
  * |title-pages <on/off>| = turn on/off atlas title pages
  * |transparency <on/off>| = set transparency for the passages (underlying 
    passages are also visible)
  * all other lines can contain \TeX\ macros, which will produce
    the cave atlas. Here the page layout may be adjusted (positions
    of elements in the area framed by dashed blue line in the picture).
    See the chapter {\it Page layout} for details.
\endcomopt

\midinsert
  \ifx\pdfoutput\undefined\else
    \pdfximage {pic/page.pdf}%
  \fi
  \vbox to 482bp{\centerline{\hbox to 400bp{%
    \ifx\pdfoutput\undefined
      \epsfbox{mp/page.1}%
    \else
      \rlap{\pdfrefximage\pdflastximage}%
      \convertMPtoPDF{mp/page.1}{1}{1}
    \fi
    \hss}}\vss
  }
\endinsert


\subsubchapter `export'.

\description
  Exports specified data from database. 
\enddescription

\syntax
  \list
    * |export <type> [OPTIONS]| 
%    * |export map/atlas [PROJECTION [OPTIONS]]| 
  \endlist
\endsyntax

\arguments
  * |<type>| = The following export types are supported:

    |model| = 3D model of the cave

    |map| = one page 2D map

    |atlas| = 2D atlas in more pages
\endarguments

\options
  {\it common:}

  * |output/o <file>| = set output file name. If no file name is
    given the prefix ``|cave.|'' is used with an extension corresponding to
    output format.

  {\it model:}

  * |format/fmt <format>| = set model output format. Currently the following
    output formats are supported: Therion (native format), Compass 
    (plt file) and Survex (3d file).

  {\it map/atlas:}
  
  * |projection| is a unique identifier that specifies the map projection type.
    (See the |scrap| command for details.) If there are no scraps with 
    the specified projection then Therion will give an error. 
  * |layout| = map or atlas layout---see layout command for details.
  * |layout-xxx| = where |xxx| stands for other layout options. Using this
    you can change some layout properties directly within the export command.
%  * |map_level <level>| = defines the maximum level of recursion the program 
%     will perform as it selects maps for output.
  * |format/fmt <format>| = set map format. Currently only PDF format
    is supported.
\endoptions



\subchapter Running Therion.

Now, after mastering data and configuration files, we're ready to run Therion. 
Usually this is done from the command line as follows

|therion|

The full syntax is

|therion [-q] [-L] [-l <log-file>]
        [-s <source-file>] [-p <search-path>]
        [-g/-u] [-i] [-d] [<cfg-file>]

therion [-h/--help]
        [-v/--version]
        [--print-encodings]
        [--print-init-file]|

\arguments
  |<cfg-file>| 
  Therion takes only one optional argument: the name of a configuration
  file. If no name is specified |thconfig| in the current directory is used. 
\endarguments

\options
* |-g|

        Using this option you can generate a new configuration file.
        If |cfg-file| is not specified therion will use the |thconfig|
        file. If the destination file exists, it'll be overwritten.
        
* |-h, --help|

        Display short help.

* |-i|

        Ignore comments when writing |-g| or |-u| configuration file.

* |-L|

        Do not create a log-file. Normally therion writes all the messages
        into a therion.log file in the current directory.
        
* |-l <log-file>|

        Change the name of the log file.
        
* |-p <search-path>|

        This option is used to set the search path (or list of 
	colon-separated paths) which therion uses to find its source
        files (if it doesn't find them in the working directory).

* |-q|

        Run therion in quiet mode. It will print only warning
        and error messages to STDERR.

* |--print-encodings|

        Print a list of all supported encodings.
        
* |--print-init-file|

        Print a default initialization file. For more details
        see the {\it Initialization} section in the {\it Appendix}.
        
* |-s <source-file>|

        Set the name of the source file.
        
* |-u|

        Upgrade the configuration file.

* |-v, --version|

        Display version information.        
        
* |-d|

        Turn on debugging mode. The current implementation creates a 
	temporary directory named |thTMPDIR| (in your system temporary 
	directory) and does not delete any temporary files. 
	What to do with temporary files is explained in the chapter
	{\it Debugging mode} in the {\it Appendix}.
\endoptions

\subsubchapter XTherion---compiler.

[Compiler mode F3 and compilation F9]


\endinput
